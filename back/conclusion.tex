% !Mode:: "TeX:UTF-8" 
\begin{conclusions}

本文采用基于控制体积的多孔介质流动计算模型,对层流非稳态情形下的多孔圆柱绕流问题进行了数值模拟,主要探讨了达西数和雷诺数变化对流动的影响,其中雷诺数的范围是 40--200,达西数的范围是 $1\times 10^{-5}$--$0.01$。根据计算得到的结果,对流动特性进行分析,得出了以下结论。

(1)在适当划分计算区域的基础上,研究了网格密度对计算结果的影响。将达西数和雷诺数分别设为 0.0001 和 100,对比了不同网格密度下的平均阻力系数,确定了将网格尺寸设为 $80\times 80$ 时模拟结果较好。通过对比达西数等于 $1\times 10^{-5}$ 时斯特劳哈尔数、平均阻力系数和升力波动振幅与已有结果的差别,验证了求解程序的准确性。

(2)得到了不同达西数下非稳态层流流动的范围,主要是流动从稳态变化为非稳态的临界雷诺数,并且临界雷诺数随达西数的增大而增大。当达西数为 $1\times 10^{-5}$ 时,临界雷诺数为 40--45,达西数增加到 0.001,临界雷诺数仍在 40 到 45 之间,当达西数增大到 0.01 时,临界雷诺数随之增加到 140--160,达西数达到 0.1 时,雷诺数 40--200 范围内的流动已经全都是稳态流动了。

(3)为了说明雷诺数和达西数的影响,分析了几个重要的时间平均量。其中,衡量波动频率的斯特劳哈尔数随着雷诺数的增加而增加,但增速逐渐减小。达西数对该参数的影响较小。另外,几个物理量的能谱分析也佐证了这一点。达西数较小时平均阻力系数随雷诺数先增后减,达西数较大时则相反。压差阻力对总阻力形成所起的作用要大于摩擦阻力,雷诺数或达西数越大,这一趋势就越加明显,该比例从 71\% 到 92\% 不等。方均根升力系数随雷诺数的增加而增加,表明涡脱落越来越剧烈。当雷诺数较小时,方均根阻力系数与雷诺数的平方根成正比。压差波动在升力系数波动中所占的比例也要大于摩擦引起的作用,该比例从 80\% 变化到 96\%。%压强和摩擦引起的升力波动之间存在相位差,大约为 $30^\circ$。

圆柱等钝体绕流现象已经被广泛研究,而以往多孔圆柱绕流的研究主要关注稳态特性,随着雷诺数继续增大,流动逐渐变为非稳态,本文则考察了这种情形下的流动特性。在多孔流动中采用了更加通用的理论模型,应用了应力阶跃条件,得到了多孔介质的存在对流动的影响,为层流非稳态多孔圆柱绕流的研究提供了部分参考,同时为进一步的研究做出了准备。

由于斯特劳哈尔数、阻力系数、升力系数等参数在工程实践中具有一定的意义,本文得出了非稳态层流中这些量的特性,所以可以和实际应用相结合,得到具有价值的启发。此外,本文的研究工作存在局限性,还有许多遗留问题等待探讨。

(1)对一些流动特性未作出充足的解释。在某些方面,多孔介质的存在对流动的影响并不十分清晰。

(2)稳态多孔圆柱绕流时存在许多不同于普通圆柱绕流的特性,非稳态绕流情形下还没有做类似的探讨。

(3)流动中还有许多有意义的物理量未作分析。通过对那些物理量的分析,可以得到更多雷诺数和达西数的影响。除了雷诺数和达西数,还可以研究其他参数的影响。

\end{conclusions}
