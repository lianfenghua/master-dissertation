% !Mode:: "TeX:UTF-8"
\begin{acknowledgements}

又到毕业时。16 年秋天,我来到深圳,开始了南科大与哈工大的联培项目,时间匆匆,如今研究生的第二个年头都已经过去大半了。

在完成毕业论文的过程中,我得到了许多人的指导和帮助;整个研究生期间,我也受益于大家的鼓励和支持,故在这里致以谢意。

首先,感谢我的导师余鹏教授。我的毕设课题和老师之前的研究一脉相承,研究思路和求解方法也有老师过去的研究可作参考,所以省去了不少功夫。在过去的时间里,科研和生活中的心态曾经暂起暂落,在与老师的交流中获益良多,从而能够放下部分疑惑,逐渐调整自己。“云山苍苍,江水泱泱”,老师的教诲将成为我漫漫路途的一部分。

其次,感谢课题组的人们。在做毕设课题的过程中,从确定论文框架,到填充具体细节,都得到了汤婷婷师姐的帮助,论文的主要工作得以完成。同时感谢课题组中苏建、卢锐新、余世敏师姐等其他成员不时的建议和帮助。

感谢一起就读于此的同学,在学习生活中可以互相支持,共同前行。感谢父母家人的支持。感谢哈工大\LaTeX\ 论文模板\hithesis\ ,在写论文时不需要忙碌于繁杂琐碎的格式要求。

此外,还得到了许多其他人或多或少的帮助,无法一一列出名姓,在此一并表示感谢。

%细细回忆,过去的许多想象都变得不切实际,%在太史公自序中,司马迁立志继承春秋时曾说“小子何敢让焉”,

秦吴绝国,燕宋千里;春苔始生,秋风暂起。从此告别了研究生的生活。
%别离的气氛总令人惆怅,空间和时间的距离。收拾旧行囊,继续向前走。这样的字还是不要写,一点都不好。

\end{acknowledgements}
