% !Mode:: "TeX:UTF-8"

\hitsetup{
  %******************************
  % 注意:
  %   1. 配置里面不要出现空行
  %   2. 不需要的配置信息可以删除
  %******************************
  %
  %=====
  % 秘级
  %=====
  statesecrets={公开},
  natclassifiedindex={O351.2},
  intclassifiedindex={532},
  %
  %=========
  % 中文信息
  %=========
  ctitle={多孔介质圆柱非稳态绕流现象的数值研究},
  cxueke={工学},
  csubject={力学},
  caffil={南方科技大学},
  cauthor={廉丰华},
  csupervisor={余鹏 副教授},
  % 日期自动使用当前时间,若需指定按如下方式修改:
  cdate={2018 年 6 月},
  cstudentid={11649093},
  %
  %
  %=========
  % 英文信息
  %=========
  etitle={Numerical investigation of unsteady flow through and around porous circular cylinders},
  exueke={Engineering},
  esubject={Mechanics},
  eaffil={\emultiline[t]{Southern University of Science \\ and Technology}},
  eauthor={Lian Fenghua},
  esupervisor={Prof. Yu Peng},
  % 日期自动生成,若需指定按如下方式修改:
  edate={June, 2018},
  estudenttype={Master of Engineering},
  %
  % 关键词用“英文逗号”分割
  ckeywords={计算流体力学,圆柱绕流,多孔介质,数值模拟,斯特劳哈尔数,阻力和升力系数},
  ekeywords={computational fluid dynamics, flow around circular cylinders, porous media, numerical simulation, Strouhal number, drag and lift coefficient},
}

\begin{cabstract}
圆柱绕流和多孔介质流动现象在自然界和工程中都很常见,例如,在传统工程领域,桥梁、建筑物、地下水、岩溶地貌中都充斥着类似现象;近年来随着生物、医药科技的快速发展,与之相关的多孔介质流动现象也受到了很大的关注。因此,对该现象的研究具有重要意义。

本文对一般情形下的多孔圆柱绕流进行了数值模拟,研究主要集中在层流非稳态情形,采用了控制体积法,应用了更加通用的理论模型和应力阶跃条件,探讨了达西数和雷诺数对流动的影响,得到了流动的特性。

根据计算数据,通过对不同雷诺数下物理量波动大小的分析,得到了从稳态转变为非稳态的临界雷诺数。随着达西数的增加,临界雷诺数逐渐向增大的方向移动,从一开始的 40--45 增大到了 200 以上。通过分析几个重要的物理量,探讨了雷诺数以及多孔介质的存在对流动产生的影响。通过时变曲线和能谱分析均得出,衡量波动频率的斯特劳哈尔数随着雷诺数的增加而增加。平均阻力系数随雷诺数的变化取决于达西数的大小,达西数较小时先增后减,达西数较大时则相反。在总阻力中,压差阻力所占的比例大于摩擦阻力,该比例从 71\% 变化到 92\%,说明尾迹区的耗散对阻力起主要作用。雷诺数或达西数越大,这一趋势就越加明显。方均根升力系数随雷诺数的增加而增加,表明涡脱落越来越剧烈。当雷诺数较小时,方均根阻力系数与雷诺数的平方根成正比。在升力系数波动中,压差波动所起的作用占主要部分,该比例从 80\% 变化到 96\%。随着达西数的增大,升力的波动减小,即涡脱落的程度有所下降。
\end{cabstract}

\begin{eabstract}
Both flow around circular cylinders and flow through porous media are common in nature and engineering. For example, similar phenomena appear in the traditional engineering, including the bridge, buildings, groundwater, and karst topography; in recent years, with the rapid development of biotechnology and medical science, the related flow phenomenon has attached a great deal of attention. Therefore, the study on this phenomenon is of great significance.

In this article, numerical simulations of the flow around porous circular cylinders were carried out. The study was mainly focused on the unsteady state of laminar flow. The control volume method was used, and a more general theoretical model and stress jump conditions were applied. It was discussed how the Darcy number and Reynolds number have an effect on the flow, and the flow characteristics were acquired.

Based on the calculated data, the critical Reynolds numbers that represents the flow's change from steady state to unsteady state were obtained, by analyzing the fluctuations of the physical quantities at different Reynolds numbers. As the increase of Darcy number, the critical Reynolds number gradually moves toward the direction of increase, from 40--45 initially to more than 200. By analyzing several important physical quantities, the influence of Reynolds number and the presence of porous media on the flow was investigated. Both the time-varying curve and the energy spectrum analysis show that the Strouhal number, which measures the frequency of fluctuations, increases with increasing Reynolds number. The change of the time-average drag coefficient with the Reynolds number depends on the Darcy number. When Darcy number is small, it increases first and then decrease, and when Darcy number become large, the opposite is true. In the total drag, the proportion of pressure difference, from 71\% to 92\%, is larger than the frictional resistance, indicating that the dissipation in the wake plays a major role in the generation of drag. The larger the Reynolds or Darcy number is, the more obvious this tendency is. The root mean square (rms) of lift increases with Reynolds number, meaning the stronger vortex shedding. When the Reynolds number is small, rms of lift coefficient is proportional to the square root of the Reynolds number. In the fluctuations of lift coefficient, the effect of pressure difference is the main part, with the ratio from 80\% to 96\%. With the increase of the Darcy number, the fluctuations of lift decreases, demonstrating the weaker vortex shedding.
\end{eabstract}
