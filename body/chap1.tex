% !Mode:: "TeX:UTF-8"
% !TEX root = ../main.tex
\chapter{绪论}

\section{研究课题的来源、背景及意义}

如今,多孔圆柱绕流问题成为了人们研究的重要领域,尤其是近年来生物、医学及化学领域的多孔流动问题越来越多,等待着人们更进一步的研究。

半页……

\section{国内外的研究进展及成果}

\subsection{圆柱绕流}

在自然界和工程实践中,流体绕过钝体的流动是一种常见的现象。船只在水面行驶,飞行器在空中航行,河水流过桥梁,海水绕过岛屿,风吹过高大的建筑物,这些都是钝体绕流的实例,它出现在我们世界的方方面面,与我们的生活息息相关。人们很早就对这种现象进行了思索和研究,随着流体力学基本理论的建立,人们对此类现象的认识也逐渐深刻起来。

圆柱绕流是流体力学中的一个经典问题,按照雷诺数的划分,整个流动状态可以被划分为许多阶段 \cite{zdravkovich1997flow}。在远离圆柱的区域,流动可以按照势流理论处理。在靠近圆柱的区域,按流动特性可以将区域划分为四部分,参见图~\ref{fig: flow area}\cite{demartino2017aerodynamics}。这四个区域可分别称为缓流区、边界层区、边界区和尾迹区。当流体横掠平板时,从平板的前缘开始形成边界层,起初流态为层流状态,随着流体向下游的移动,层流逐渐转变为湍流状态。随着雷诺数的增加,层流转变为湍流的转变区也向上游移动。同样地,当流体流过圆柱时,在下游会出现流态的转变,随着雷诺数的增加,转变点也逐渐向上游移动,并且依次经过划分出来的几个区域,相应的流态称为尾迹区转捩(TrW)、剪切层转捩(TrSL)、边界层转捩(TrBL)。

\begin{figure}
	\centering
	\includegraphics[width=.8\textwidth]{figs/Regions-of-disturbed-flow-around-a-perfect-circular-cylinder}
	\caption{流动区域的划分 \cite{demartino2017aerodynamics}}
	\label{fig: flow area}
\end{figure}

层流状态(L)。当 0 < $Re$ < 4--5 时,流体沿着圆柱的轮廓流动,圆柱左右的流线呈现出对称的特性,此时还没有出现流动分离。当 4--5 < $Re$ < 30--48 时,流体从圆柱表面分离,圆柱的背面开始出现封闭的附着涡,附着涡没有从圆柱表面脱落,此时尚处于稳定状态。当 30--48 < $Re$ < 180--200 时,圆柱背面的涡从圆柱表面脱落,并由近而远地向远处缓缓飘去。其中在 $Re$ > 30--48 时,尾迹区逐渐由稳态向非稳态过渡,尾迹末端的流线开始出现类似正弦曲线的摆动。随后,大约在 $Re$ > 45--65 时,剪切层卷起了一排排的波峰和波谷,接着出现了一列列的涡,即为涡街。

尾迹区转捩(TrW)。这一范围称为亚临界状态。当 180--200 < $Re$ < 220--250 时,首先在远离圆柱的尾迹区出现层流向湍流的转变(远尾迹转捩,TrW1),随着雷诺数的增大,转变点向上游移动。当 220--250 < $Re$ < 350--400 时,转变发生在接近圆柱的地方,称为近尾迹转捩(TrW2),最终,当涡一开始产生就已经处于湍流状态。TrW1和TrW2的一个重要现象是非稳态涡产生和脱落的层流模式逐渐被湍流模式所取代 \cite{zdravkovich1997flow}。

自由剪切层转捩(TrSL)。雷诺数处于这一范围的流动被称为亚临界状态。当 350--400 < $Re$ < 1k--2k 时,过渡波首先出现在近壁面剪切层,这一区域称为下亚临界区(TrSL1)。1k--2k < $Re$ < 20k--40k 时,过度涡出现在剪切层,在转变为湍流之前,过渡波沿着剪切层卷为分散的涡,之后则继续变成交替的涡,这一阶段为中亚临界区(TrSL2)。雷诺数范围为 20k--40k < $Re$ < 100k--200k 时是下亚临界区(TrSL3),此时近壁面处已全部转变为湍流状态,近壁面的自由剪切层突然转变为湍流,涡出现在圆柱的背面。

边界层转捩(TrBL)。这一状态也称为临界状态。100k--200k < $Re$ < 300k--340k 为前临界区(TrSL0),沿着分离线的剪切层开始转捩,阻力系数开始下降。300k--340k < $Re$ < 380k--400k 为单泡临界区(TrBL1),圆柱一侧的自由剪切层重附着形成分离泡。380k--400k < $Re$ < 0.5M--1M 为双泡临界区(TrBL2),圆柱两侧形成一对对称的分离泡。0.5M--1M < $Re$ < 3.4M--6M 为超临界区(TrBL3),此前出现的分离泡破碎,脱落涡丧失了周期性。3.5M--6M < $Re$ < unknown 为过临界区(TrBL4),此时边界层已完全转变为湍流,脱落涡周期性重现。这一阶段雷诺数的上界仍不为人所知。

完全湍流状态(T)。当雷诺数继续增大,整个流场变成了湍流状态。

在圆柱绕流的几个阶段中,层流流动已经被较多地研究过 \cite{},非稳态层流。(?)在非稳态区域,当雷诺数大于某个临界值 $Re_\text{osc}$ 时,原本处于稳态的层流就开始变得不稳定,圆柱尾部封闭的尾迹涡开始轻微振荡。临界雷诺数 $Re_\text{osc}$ 体现了尾迹的不稳定性,不同研究所揭示出的这一参数的具体数值有所不同 \cite{zdravkovich1997flow}。如果雷诺数小于临界值,尾迹也可以由外界激发而产生波动(例如弹性丝线的振动),随着外界干扰的平息,尾迹又会重新回归稳态。在实验研究中,人为施加的激励可以使临界雷诺数的大小降至 20 甚至 10,从而可以测出波动的频率,即 Strouhal 数。实验表明,由此得出的频率曲线可以和 $Re>Re_\text{osc}$ 时的频率曲线光滑衔接。

当雷诺数大于临界值时,近尾迹的不稳定将产生一个波浪形的痕迹,尾波开始卷起。Bénard 做了在糖水溶液中拖动圆柱的开创性实验,描述了尾迹处形成的旋涡。Phillips \cite{} 通过实验发现,当 $40<Re<80$ 时,周期性的尾迹是二维结构,当 $80<Re<100$ 时,尾迹对扰动很敏感并可能变成三维的,当 $100<Re<160$ 时,尾迹就总是呈现三维结构了。也有其他一些实验的结果与此不太相同。

%许多努力都试图寻找为什么层流周期性尾迹会存在两种流动模式。

Strouhal 第一次利用金属丝和杆在空气中产生声音并测量了它的频率,提出了无量纲参数 $fD/V$, 后来被称为 Strouhal 数。他的研究发现,对细金属丝($400<Re<1k$)而言,$St=0.185$;对金属杆($1.8k<Re<5.4k$)而言,$St=0.195$。当 $40<Re<160$ 时,关系 $Re=\text{常数}$ 并不成立,此时,$St-Re$ 的关系是非线性的。Roshko \cite{Roshko1953} 得到的 $St-Re$ 关系为
\begin{equation}
	St = 0.212 (1-\frac{21.2}{Re})
\end{equation}
Roshko \cite{Roshko1953} 提出了一个新的无量纲数 $Ro=fD^2/\nu$ 从而使得 $Ro-Re$ 具有线性关系:
\begin{equation}
	Ro = 0.212 (Re - 21.2)
\end{equation}

Tritton 第一次使用了线性的 $Ro-Re$ 关系。在空气和水中,当 $Re>80$ 时,涡脱落的频率会有一个不连续的下降,并且发现,涡脱落频率的下降并不会对阻力产生影响。Tritton 注意到热金属丝产生的信号在不连续点以上和以下都很有规律,但处在不连续点附近的一个小范围内时,这些信号就变得不再那么规律。这种从一个模式到另一个模式的改变不一定发生在一个特定的 $Re$ 值,实际上,它可以发生在 $80<Re<105$ 的范围内。Tritton 认为,这种不连续是由封闭尾迹的突然消失和紧接着涡生成机制的改变引起的。这两种现象都会对阻力产生影响。Berger 发现,在 $Re=125$ 以上,$Ro$ 也有类似的不连续现象,并发现在 $125<Re<160$ 范围内存在两种亚稳态模式:一种是 Tritton 的高速模式,另一种低速模式被 Berger 称为“基础模式”,因为信号的幅值和相位都很有规律。这很可能是由圆柱的振动引起的。许多研究者都试图找到脱落的不连续点所对应的临界雷诺数 $Re_d$。Gaster 通过减小圆柱的长径比抑制了高速模式。Nishioka 和 Sato 将长径比减小到 $L/D=6$,不但抑制了原来存在的不连续性,而且将涡脱落时的临界雷诺数推移到了 $Re_\text{osc}=85$。Friehe 确信,脱落频率和 $Re_d$ 都严重依赖于 $L/D$。

Teissié-Solier 等人于 1937 年发现,一个固定直径的圆柱在一个测试段中产生了两个不同的频率。??他们测量了沿着中心间隔的高频和沿着尾迹的低频。??低频部分的范围是 $6D-8D$,并且在 $Re>214$ 时消失。Gerrard 测出 $Re=85$ 从尾端数 $7D$ 的距离时频率低了 16\%,证明了壁面涡街的存在。??Gerich 和 Eckelmann 随后的研究确认了沿着圆柱展向的不同脱落频率的存在以及它们的范围。

\subsection{多孔介质内部的流动}

流体通过多孔介质的流动是地下水水文学、石油工程、土壤学及化学工程等领域的常见现象 \cite{Bear2013}。地下水中的含水层,石油工程中的储油层(见图~\ref{fig: Aquifers and oil reservoirs}),以及土壤、多孔岩石、陶瓷、过滤纸和沙过滤器都是多孔介质的实例。此外,流化床、生物过滤器、森林中的降雨、生物工程中的微载体和支架以及多孔换热器也都是多孔介质流动的重要研究领域。

\begin{figure}
	\centering
	\includegraphics[height=.2\textheight]{figs/Aquifers}
	\includegraphics[height=.2\textheight]{figs/Oil-reservoirs}
	\caption{含水层和储油层}
	(图片来源: Colorado Geological Society; Alan Sherwood and Jock Phillips, 'Coal and coal mining - The future of coal', Te Ara - the Encyclopedia of New Zealand)
	\label{fig: Aquifers and oil reservoirs}
\end{figure}

多孔介质可以简单描述为“含有孔洞的固体”,从流体流动的角度来说,它是多相物质占据的空间,且有一相需为固体;固相分布在整个多孔区域;一部分孔隙必须互相连通,使得气液相的流体可以在其中流动 \cite{Bear2013}。

流体的连续介质假设将对流体的描述从分子尺度转移到流体质点尺度(微观水平)。类比对流体的描述,我们引入多孔介质的连续介质假设,这样就将描述水平进一步转到表征体积单元(REV)的尺度上(宏观水平)。在这个假设的基础上,我们可以用连续函数来描述多孔介质内的流动,并由此建立起相应的质量方程和动量方程。

在描述多孔介质流动方面,Darcy 定律和 Brinkman 方程是两个主要的模型。这两个模型都不含时间的导数项和非线性的对流性。Wang L 等人 \cite{Wang2015} 通过采用体积平均法,从具有孔隙尺度的宏观方程中严格推导出了包含时间导数项和非线性对流项的控制方程,对此方程应用蠕动流条件可以推导Darcy定律和Brinkman方程。该方程对内部项平均速度和相平均速度均满足伽利略不变性,具有现实的物理意义。为了验证得到的结果,该文献采LB方法求解了多孔圆柱沿槽道中心线运动的问题,并且模拟了多孔通道中的Poiseuille流动,与有限差分的结果相一致。在此基础上建议使用内部相平均速度而放弃使用通常用于多孔粒子系统的相平均速度。



\subsection{绕过及穿过多孔钝体的流动}

由于在自然界和工程中的普遍性,钝体绕流现象一直是人们研究的重要内容。

到目前为止,人们针对钝体绕流现象已经做出了大量研究,得到了许多有意义的结果。Joseph 和Tao \cite{joseph1964effect} 采用 Darcy 定律与 Stokes 渐进方程研究了低雷诺数下流体绕过多孔球体的粘性不可压缩流动,得到了速度场、压力场和阻力系数的解析解。Neale 等 \cite{neale1973creeping} 探究了 Brinkman 项对多孔球体绕流的影响,结果发现,在低雷诺数条件下,多孔圆柱受到的阻力要低于相同条件下的实心圆柱。Masliyah 和 Polikar \cite{masliyah1980terminal} 通过实验验证了 Neale 等人的结果。接着,Masliyah 和 Polikar \cite{masliyah1980terminal} 进一步发现,在高雷诺数条件下(7 < $Re$ < 120),多孔球体受到的阻力可能会高于相同条件下的实心球体。Nandakumar 和 Masliyah \cite{nandakumar1982laminar} 通过研究发现,Masliyah 和 Polikar 预测的阻力数值比他们的数值模拟结果高出大约 10\%。

Hsu 等人 \cite{hsu2004re} 采用 Brinkman 模型研究了低雷诺数下流体绕过多孔球壳的流动。Bhattacharyya 和 Raja Sekhar \cite{bhattacharyya2004viscous} 研究的多孔球体模型有一个不允许流体通过的中心,他们研究了穿过这个多孔球体模型的 Stokes 流动,其中多孔介质内的流动采用了 Brinkman 模型,多孔介质和纯流体之间的界面上采用了应力阶跃条件 \cite{ochoa1995momentum1,ochoa1995momentum2}。结果发现,应力阶跃系数对作用在该球体上的阻力和转矩都有着显著的影响。

Jue \cite{jue2004numerical} 探究了当 $Re$ = 100, 200 和 250 时多孔圆柱绕流中圆柱背面涡的脱落情况。他的研究基于有限单元法,采用了通用的非 Darcy 多孔介质模型,并在处理纯流体和多孔介质界面处的流动特性时使用了调和平均数。研究发现,除雷诺数外,达西数对流动也有着显著的影响,但孔隙率的影响却无足轻重。Chen 等人 \cite{chen2008numerical} 研究了应力阶跃条件对多孔圆柱绕流的影响,结果显示,当达西数增大时,涡脱落现象发生时所对应的雷诺数也会随之增大。同时,界面应力阶跃参数对多孔圆柱绕流的稳定性有着重要影响,其中第一参数有明显的影响而第二参数影响很小。

Noymer 等人 \cite{noymer1998drag} 使用商业软件 PHOENICS 研究了圆柱绕流现象,Darcy 方程和 Navier-Stokes 方程分别用来描述多孔区和纯流体区的流动,在两种流动的界面上,压力和质量流量匹配在一起。结果显示,当 $Re$ = 100 和 1000 时,作用在多孔圆柱上的阻力要比无渗流的实心圆柱高得多。这一结论也被他们的风洞试验所证实。Bhattacharyya 等人 \cite{bhattacharyya2006fluid} 通过数值分析的方式研究了绕过及穿过多孔圆柱的稳定流动,多孔区域的流动采用了包含 Brinkman 项、Forcheimmer 项和非线性对流项的通用模型。结论显示,阻力系数、尾迹长度和流动分离的角度都会随着达西数的增大而减小。当达西数减小时,再循环尾迹产生时对应的临界雷诺数也会单调减小,最终减小到一个渐进值,即为绕实心圆柱流动产生涡时所对应的雷诺数。

Fransson 等人 \cite{fransson2004flow} 研究了连续抽气和吸气时的圆柱绕流情况,研究的雷诺数范围为 $10^4$ 量级,即处于亚临界区。研究发现即使中等程度的吸气或吹起对流场都有着明显的影响。吸气可以延迟流动分离的发生,缩小尾迹的宽度,减小阻力,吹气具有相反的效果。

Yu 等人 \cite{Yu2010} 研究了绕过及穿过多孔圆柱流动的稳定状态,分析了雷诺数和达西数对流动的影响。结果发现圆柱背面的循环尾迹或穿入圆柱之内,或和圆柱相分离。多孔圆柱的尾迹从内部或下游开始发展,而不会像实心圆柱那样从表面开始。有限雷诺数下尾迹的形态是涡积累的结果,而非逆压梯度下的边界层分离所致。尾迹出现的临界雷诺数是达西数的函数;穿透深度是雷诺数和达西数的函数。

王亚玲 \cite{王亚玲2001圆柱绕流的三维数值模拟} 等人对圆柱绕流进行了三维数值模拟,采用有限体积法模拟了亚临界区内的绕流流动,计算结果表明,高雷诺数时圆柱周围的流动具有明显的三维特性,且沿柱长方向不同断面的升力和阻力系数并不相同,三维模拟的升力和阻力系数均小于二维模拟。而何鸿涛 \cite{何鸿涛2009圆柱绕流及其控制的数值模拟研究} 通过数值模拟的方法来研究圆柱绕流的基本特性,探究了三种控制方法对于圆柱绕流尾部涡脱落的控制效果。

\subsection{存在的不足或有待深入研究的问题}

起初人们对这一问题的研究主要采用理论分析和解析计算的方法,随着计算机的发展,数值计算技术得到了迅速提高,与之相关的理论也层出不穷,极大地丰富了研究的手段。之后人们便主要利用数值模拟来研究此类问题了。同时也和实验相配合,验证结果的可信度,更深刻地理解其中的原理。另一方面,对流动基本原理的探讨也一直在进行中,包括对基本控制方程的改进和新的解释。

针对多孔圆柱绕流或与之相似的问题,人们通常研究的是低雷诺数绕钝体的流动,相应的钝体主要是几何形状简单的物体,例如球体、圆柱或是球壳、环形。一开始采用的模型主要是 Darcy 定律和 Brinkman 模型。后来,随着对流动控制方程认识的深入,更多的物理含义被发掘出来,于是,包含有 Brinkman 项、Forcheimmer 项和非线性对流项的通用模型被越来越多的人所采用。应力阶跃条件被用在对纯流体区域和多孔介质区域界面条件的处理上,但有研究指出应力阶跃条件对作用在球体上的阻力有着明显的影响。在实际的研究中,人们发现,相比于普通的绕流问题,绕流物体内部多孔介质的存在对流动有着重要的影响,主要是达西数和孔隙率的影响。当达西数增大时,涡脱落现象发生时所对应的雷诺数也会随之增大。而更有研究指出阻力系数、尾迹长度和流动分离的角度都会随着达西数的增大而减小。现有研究表明,在稳态流动时,多孔圆柱背面的尾迹回流区和普通圆柱有所不同。普通圆柱的尾迹区附着在圆柱表面,而多孔圆柱的回流区并没有附着在表面,而是和圆柱分离了一段距离,或者穿入了圆柱内部,甚至在雷诺数增加时还会消失,这些现象指示出稳态条件下多孔介质流动不同于普通圆柱的诸多特性。因此很有必要研究非稳态时,多孔介质和普通圆柱绕流的不同点。即流态发展到非稳态时多孔介质的存在又会对流动产生怎样的影响,以及多孔参数是如何对流动状态发生具体影响的,仍是需要进一步研究的问题。

\section{本文的主要研究内容}

本课题的研究内容主要是

多孔圆柱流动中流动状态随雷诺数的变化和非多孔圆柱是相似的,当雷诺数不大时,垂直于圆柱轴线任一截面上的流态是相同的,此时的流动是二维问题。雷诺数较小时流动处于稳态,随着雷诺数的增大,圆柱的背面会出现漩涡。雷诺数继续增大,流态也渐渐变成非稳态,圆柱背面会出现两列对称的涡街,速度等流动参数也以一定的规律变化着。

对多孔圆柱绕流的研究通常关注稳定状态的流动,当雷诺数增大时,流动将变得不稳定,逐渐进入非稳定状态。对多孔圆柱绕流非稳态特性的研究就构成了本课题的主要内容。进一步地,课题的研究内容可以分为以下两个方面。

(1)	层流非稳态情况下多孔圆柱绕流的数值模拟。

首先明确描述研究区域内流体流动的控制方程。由于控制方程是复杂的非线性偏微分方程,无法得到解析解,所以接下来要对对方程进行离散化处理,利用计算流体力学的理论,采用合适的离散格式,运用数值求解技术来求解对应的方程,从而解决方程所描述的物理问题。最终得到所研究的流场内各个点上的数据,例如速度、涡量、压力等,以及不同条件下圆柱所受到的阻力和升力。

(2)	流动特性的分析。

利用第一步计算得到的数据,经过数据处理和分析,得到对流动形态准确而清晰的描述。通过和圆柱绕流的结果作对比,发现它们之间的相同和不同之处,分析多孔介质的存在是如何影响到流动状态的,具体而言则是分析与多孔介质相关的参数,例如达西数与孔隙率对流动的影响。同时,当这些参数取极限值时,流动状态应该退回到普通圆柱绕流时的情况。另一方面,通过和雷诺数更小时稳态情形下的多孔圆柱绕流作比较,分析流动是如何由稳态演化到非稳态的,将它们综合起来,以期待能够形成流动状态演化的更加清晰的图景。

通过对结果的分析,在初步明了参数对流动的影响之后,再返回第一步,通过设定特定的参数值,得到更多不同参数下的数据。在合理的分析方法下,继续对所得的数据进行分析处理,以得到更加具有说服力的结果,还可以对结果做进一步的处理。和已有的文献结果作比较,得到最终的结论。
