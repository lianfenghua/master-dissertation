% \chapter{Analysis of Flow Characteristics}
\chapter{Results and Discussion}
\section{Flow Patterns and Vorticity}
\subsection{Steady state (L2) and unsteady state (L3)}
For each Darcy number, the flow is at steady state when Reynolds number is small. As Reynolds number increases, the flow state will become unsteady. For each Darcy number, there is a critical Reynolds number that means the flow state change from steady to unsteady. Figure \ref{fig: Cl_t} shows three cases of lift coefficient with time. The critical Reynolds number can be estimated from Figure \ref{fig: Cl_t}, Figure \ref{fig: resd} and Figure \ref{fig: error}.  These critical Reynolds numbers are shown in Table \ref{tab: critical Re}. Figure \ref{fig: resd} shows the error of different Reynolds number and Darcy number.

\begin{table}[]
	\centering
	\caption{The critical Reynolds numbers.}\label{tab: critical Re}
	$\begin{array}{c|c}
	\hline
	Da & \mathrm{Critical}\, Re \\ \hline
	0.00001& 40-45   \\
	0.0001 & 40-45 \\
	0.001  & 40-45 \\
	0.005  & ?   \\
	0.01   & 140-160 \\
	0.1    & >200\\
	\hline
	\end{array}$
\end{table}

\begin{figure}[!h]
	\setlength{\subfigcapskip}{-1bp}
	\centering
	\begin{minipage}{\textwidth}
		\centering
		\subfigure[$Re=40$]{\includegraphics[width=0.8\textwidth]{../figs/0.0001_40/Cl_t}}
	\end{minipage}
	\centering
	\begin{minipage}{\textwidth}
		\centering
		\subfigure[$Re=45$]{\includegraphics[width=0.8\textwidth]{../figs/0.0001_45/Cl_t}}
	\end{minipage}
	\vspace{0.2em}
	\caption{Temporal variation of lift coefficient with time at different Reynolds number. $Da=0.0001$.}
	\label{fig: Cl_t}
\end{figure}

\begin{figure}[!h]
	\centering
	\begin{minipage}{\textwidth}
		\centering
		\subfigure[$Re=40$]{\includegraphics[width=0.8\textwidth]{../figs/0.0001_40/resd}}
	\end{minipage}
	\centering
	\begin{minipage}{\textwidth}
		\centering
		\subfigure[$Re=45$]{\includegraphics[width=0.8\textwidth]{../figs/0.0001_45/resd}}
	\end{minipage}
	\caption{The error with time steps. $Da=0.0001$}
	\label{fig: resd}
\end{figure}

\begin{figure}[]
	\centering
	\includegraphics[width=0.8\textwidth]{../analysis/meanError_Re}
	\caption{Variation of the mean error with $Re$.}
	\label{fig: error}
\end{figure}

\subsection{Vorticity contours}
According to the measurement data, we can get instantaneous spanwise vorticity contours for different Darcy number and Reynolds number. Furthermore, we can get the vortex's shedding during a whole period for unsteady flow state.

\begin{figure}[]
	\centering
	\begin{minipage}{\textwidth}
		\centering
		\subfigure[$Re=40$]{\includegraphics[width=0.4\textwidth]{../figs/0.0001_40/flow}}
		\subfigure[$Re=45$]{\includegraphics[width=0.4\textwidth]{../figs/0.0001_45/flow}}
	\end{minipage}
	\centering
	\begin{minipage}{\textwidth}
		\centering
		\subfigure[$Re=50$]{\includegraphics[width=0.4\textwidth]{../figs/0.0001_50/flow}}
		\subfigure[$Re=90$]{\includegraphics[width=0.4\textwidth]{../figs/0.0001_90/flow}}
	\end{minipage}
	\centering
	\begin{minipage}{\textwidth}
		\centering
		\subfigure[$Re=140$]{\includegraphics[width=0.4\textwidth]{../figs/0.0001_140/flow}}
		\subfigure[$Re=200$]{\includegraphics[width=0.4\textwidth]{../figs/0.0001_200/flow}}
	\end{minipage}
	\caption{Vorticity contours for different Reynolds numbers.
		$Da=0.0001$.}
\end{figure}


\section{Drag and Lift Coefficients, and Strouhal Number}
\subsection{Drag and Lift Coefficients}
Figure \ref{fig: meanCd} shows the variation of mean drag coefficient. For $Da=0.0001$ and $Da=0.001$, the value of the mean drag coefficient decrease with the increase of Reynold number and then have an increase with Reynold number.

\begin{figure}[]
	\centering
	\includegraphics[width=0.8\textwidth]{../analysis/meanCd_Re}
	\caption{Variation of mean drag coefficients with $Re$.}
	\label{fig: meanCd}
\end{figure}

\subsection{Strouhal Number}
For flow at unsteady state, the period of flow fluctuation is calculated according to the measurement data such as the horizontal velocity of Point 3. Then we can get the frequencies from calculated periods and obtain Strouhaul numbers further according to Equation \ref{eq: St}.
\begin{equation}\label{eq: St}
St = \frac{Df}{U} = f = \frac{1}{T}
\end{equation}
These periods and Strouhaul numbers are recorded in \texttt{record.xlsx}, and the curves that describe the change of periods and Strouhaul numbers along with Reynolds numbers are also shown.

\begin{figure}[]
	\centering
	\includegraphics[width=0.8\textwidth]{../analysis/St_Re}
	\caption{Variation of Strouhal number with $Re$.}
	\label{fig: St}
\end{figure}

For unsteady flow, vorticity contours during the whole period can be obtained. Figure \ref{fig: 4*vortex}

\begin{figure}[]
	\centering
	\begin{minipage}{\textwidth}
		\centering
		\subfigure[0]{\includegraphics[width=0.4\textwidth]{../figs/0.0001_100/T0}}
		\subfigure[1/4]{\includegraphics[width=0.4\textwidth]{../figs/0.0001_100/T1}}
	\end{minipage}
	\centering
	\begin{minipage}{\textwidth}
		\centering
		\subfigure[2/4]{\includegraphics[width=0.4\textwidth]{../figs/0.0001_100/T2}}
		\subfigure[3/4]{\includegraphics[width=0.4\textwidth]{../figs/0.0001_100/T3}}
	\end{minipage}
	\caption{The vortex's shedding at 0, 1/4, 2/4, and 3/4 period.
		$Da=0.0001$, $Re=100$.}
	\label{fig: 4*vortex}
\end{figure}


\section{Energy Spectrum Analysis} % FFT
\begin{figure}[]
	\centering
	\begin{minipage}{\textwidth}
		\centering
		\includegraphics[width=0.8\textwidth]{../figs/0.0001_50/psd_U}
	\end{minipage}
	\centering
	\begin{minipage}{\textwidth}
		\centering
		\includegraphics[width=0.8\textwidth]{../figs/0.0001_50/psd_V}
	\end{minipage}
	\caption{Velocities and their discrete Fourier transform. $Da=0.0001$, $Re=50$.}
\end{figure}

\begin{figure}
	\centering
	\includegraphics[width=0.8\textwidth]{../figs/0.0001_50/psd_Cl}
	\caption{Lift Coefficient and its discrete Fourier transform. $Da=0.0001$, $Re=50$.}
\end{figure}


\section{The Influence of Porous Media on Flow State\\
	---Comparison with Flow Around Solid Circular Cylinders}


\section{The Transition of Flow Pattern from Steady State to Unsteady State\\
	---Comparison with Steady Flow with Lower Reynolds Number}


\section{Summary}
