% !Mode:: "TeX:UTF-8"
\chapter{结果和讨论}

\section{流动实例和涡量图}

\subsection{稳态和非稳态}

对于每个达西数,当雷诺数较小时,流动处于稳态。随着雷诺数的增加,流动逐渐转变为非稳态。所以,对于每个达西数,都存在一个标志着流动状态转变的临界雷诺数。图~\ref{fig: Cl_t}~显示了升力系数随时间变化的三种情形。临界雷诺数可从图~\ref{fig: Cl_t}、图~\ref{fig: resd}~和图~\ref{fig: error}~上得到。这些临界雷诺数列在表~\ref{tab: critical Re}~中。图~\ref{fig: resd}~展示了不同雷诺数和达西数下的误差。

\begin{table}[]
	\centering
	\caption{临界雷诺数}\label{tab: critical Re}
	$\begin{array}{c|c}
	\hline
	Da & \mathrm{Critical}\, Re \\ \hline
	0.00001& 40-45   \\
	0.0001 & 40-45 \\
	0.001  & 40-45 \\
	0.005  & ?   \\
	0.01   & 140-160 \\
	0.1    & >200\\
	\hline
	\end{array}$
\end{table}

\begin{figure}[!h]
	\setlength{\subfigcapskip}{-1bp}
	\centering
	\begin{minipage}{\textwidth}
		\centering
		\subfigure[$Re=40$]{\includegraphics[width=0.8\textwidth]{../figs/0.0001_40/Cl_t}}
	\end{minipage}
	\centering
	\begin{minipage}{\textwidth}
		\centering
		\subfigure[$Re=45$]{\includegraphics[width=0.8\textwidth]{../figs/0.0001_45/Cl_t}}
	\end{minipage}
	\vspace{0.2em}
	\caption{不同雷诺数下升力系数随时间的变化 $Da=0.0001$.}
	\label{fig: Cl_t}
\end{figure}

\begin{figure}[!h]
	\centering
	\begin{minipage}{\textwidth}
		\centering
		\subfigure[$Re=40$]{\includegraphics[width=0.8\textwidth]{../figs/0.0001_40/resd}}
	\end{minipage}
	\centering
	\begin{minipage}{\textwidth}
		\centering
		\subfigure[$Re=45$]{\includegraphics[width=0.8\textwidth]{../figs/0.0001_45/resd}}
	\end{minipage}
	\caption{误差 $Da=0.0001$}
	\label{fig: resd}
\end{figure}

\begin{figure}[]
	\centering
	\includegraphics[width=0.8\textwidth]{../analysis/meanError_Re}
	\caption{平均误差随$Re$的变化}
	\label{fig: error}
\end{figure}

\subsection{等涡线图}

根据测量数据,可以得到不同达西数和雷诺数下的涡的等值线图。还可以得到非稳态情形下一个周期内涡的脱落情形。

\begin{figure}[]
	\centering
	\begin{minipage}{\textwidth}
		\centering
		\subfigure[$Re=40$]{\includegraphics[width=0.4\textwidth]{../figs/0.0001_40/flow}}
		\subfigure[$Re=45$]{\includegraphics[width=0.4\textwidth]{../figs/0.0001_45/flow}}
	\end{minipage}
	\centering
	\begin{minipage}{\textwidth}
		\centering
		\subfigure[$Re=50$]{\includegraphics[width=0.4\textwidth]{../figs/0.0001_50/flow}}
		\subfigure[$Re=90$]{\includegraphics[width=0.4\textwidth]{../figs/0.0001_90/flow}}
	\end{minipage}
	\centering
	\begin{minipage}{\textwidth}
		\centering
		\subfigure[$Re=140$]{\includegraphics[width=0.4\textwidth]{../figs/0.0001_140/flow}}
		\subfigure[$Re=200$]{\includegraphics[width=0.4\textwidth]{../figs/0.0001_200/flow}}
	\end{minipage}
	\caption{不同雷诺数下涡量的等值线图 $Da=0.0001$.}
\end{figure}

\section{阻力、升力系数,Strouhal数}

\subsection{阻力和升力系数}

图~\ref{fig: meanCd}~展示了平均阻力系数的变化。对于$Da=0.0001$、$Da=0.001$,平均阻力系数随着雷诺数的增加而减小,然后随雷诺数的增加而增加。

\begin{figure}[]
	\centering
	\includegraphics[width=0.8\textwidth]{../analysis/meanCd_Re}
	\caption{平均阻力系数随$Re$的变化}
	\label{fig: meanCd}
\end{figure}

\subsection{Strouhal数}

对于非稳态流动,流动波动的周期可以根据测量数据(例如Point 3的水平速度)求出。继而根据方程(\ref{eq: St})可以得到Strouhaul数。
\begin{equation}\label{eq: St}
	St = \frac{Df}{U} = f = \frac{1}{T}
\end{equation}
得到的周期和Strouhaul数记录在\texttt{record.xlsx}中。

\begin{figure}[]
	\centering
	\includegraphics[width=0.8\textwidth]{../analysis/St_Re}
	\caption{Strouhal数随$Re$的变化}
	\label{fig: St}
\end{figure}

对于非稳态流动,一个周期内的涡量图如~\ref{fig: 4*vortex}~所示。

\begin{figure}[]
	\centering
	\begin{minipage}{\textwidth}
		\centering
		\subfigure[0]{\includegraphics[width=0.4\textwidth]{../figs/0.0001_100/T0}}
		\subfigure[1/4]{\includegraphics[width=0.4\textwidth]{../figs/0.0001_100/T1}}
	\end{minipage}
	\centering
	\begin{minipage}{\textwidth}
		\centering
		\subfigure[2/4]{\includegraphics[width=0.4\textwidth]{../figs/0.0001_100/T2}}
		\subfigure[3/4]{\includegraphics[width=0.4\textwidth]{../figs/0.0001_100/T3}}
	\end{minipage}
	\caption{在0,1/4,2/4,和3/4周期时涡的脱落图 $Da=0.0001$,$Re=100$}
	\label{fig: 4*vortex}
\end{figure}

\section{能谱分析} % FFT

\begin{figure}[]
	\centering
	\begin{minipage}{\textwidth}
		\centering
		\includegraphics[width=0.8\textwidth]{../figs/0.0001_50/psd_U}
	\end{minipage}
	\centering
	\begin{minipage}{\textwidth}
		\centering
		\includegraphics[width=0.8\textwidth]{../figs/0.0001_50/psd_V}
	\end{minipage}
	\caption{Fourier变换 $Da=0.0001$, $Re=50$.}
\end{figure}

\begin{figure}
	\centering
	\includegraphics[width=0.8\textwidth]{../figs/0.0001_50/psd_Cl}
	\caption{升力系数和Fourier变换 $Da=0.0001$, $Re=50$.}
\end{figure}

\section{多孔介质对流动状态的影响---与固体圆柱绕流相比}

\section{流动状态从稳态向非稳态的转变---与稳态圆柱绕流相比}

\section{本章小结}
