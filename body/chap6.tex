% !Mode:: "TeX:UTF-8"
% !TEX root = ../main.tex
\chapter{从稳态流动到非稳态流动}\label{chap: transform}

\section{判断方法}\label{sec: judgement}

在稳定流动状态(L2),流动在圆柱背面分离,形成了上下对称的一对漩涡,随着雷诺数的增加,圆柱背面尾迹的长度随之增加,分离点的角度则逐渐减小(从更靠前的位置开始分离)。这些量的变化情况可以参考文献 \inlinecite{Rajani2009,Qu2013}。Waqas Sarwar Abbasi\cite{Abbasi2018} 研究了多孔方柱绕流时流动从稳态向非稳态的转变,并分析了雷诺数的影响。

固定达西数,当雷诺数较小时,流动处于稳态;随着雷诺数的增加,流动逐渐转变为非稳态,尾迹出现振荡,然后上下两个漩涡从圆柱表面交替脱落,各变量随时间做周期性变化。对于每个达西数,都存在一个临界雷诺数,该雷诺数标志着流动状态的转变。为了知晓某一达西数所对应的临界雷诺数,从已有数据计算出不同雷诺数下某一物理量随时间的变化图,通过观察物理量是否随时间波动,从而判断流动是否处于非稳态。由于计算具有一定误差,所以稳态时物理量也会随时间有微小的波动,实际当波动小于某一特定精度时就任务流动处于稳态,波动大于此精度则认为流动处于非稳态。下面将通过升力系数随时间的变化图($C_L$-$t$)来判断流动状态。

以 $Da=10^{-4}$ 为例,图~\ref{fig: Cl_t-1e-4} 显示了不同雷诺数下升力系数随时间的变化。当 $Re=40$ 时,升力系数随着时间逐渐减小,并最终减小到一个极小的值($<10^{-6}$),因此 $Re=40$ 时流动尚处于稳态。当 $Re=45$ 时,升力系数不会随时间减小,而是几乎稳定地波动,因此 $Re=45$ 时流动已经处于非稳态。由此可知,$Da=0.0001$ 所对应的临界雷诺数处于 40 和 45 之间。$Da=0.1$ 时整个雷诺数范围都处于稳态。

\begin{figure}
	\setlength{\subfigcapskip}{-1bp}
	\centering
	\begin{minipage}{\textwidth}
		\centering
		\subfigure[$Re=40$]{\includegraphics[width=0.48\textwidth]{../figs/0.0001_40/Cl_t}}
		\subfigure[$Re=45$]{\includegraphics[width=0.48\textwidth]{../figs/0.0001_45/Cl_t}}
	\end{minipage}
	\vspace{0.2em}
	\caption{$Da=0.0001$ 时不同雷诺数下升力系数随时间的变化}
	\label{fig: Cl_t-1e-4}
\end{figure}

\begin{figure}
	\setlength{\subfigcapskip}{-1bp}
	\centering
	\begin{minipage}{\textwidth}
		\centering
		\subfigure[$Re=40$]{\includegraphics[width=0.48\textwidth]{../figs/0.001_40/Cl_t}}
		\subfigure[$Re=45$]{\includegraphics[width=0.48\textwidth]{../figs/0.001_45/Cl_t}}
	\end{minipage}
	\vspace{0.2em}
	\caption{$Da=0.001$ 时不同雷诺数下升力系数随时间的变化}
	\label{fig: Cl_t-1e-3}
\end{figure}

\begin{figure}
	\setlength{\subfigcapskip}{-1bp}
	\centering
	\begin{minipage}{\textwidth}
		\centering
		\subfigure[$Re=40$]{\includegraphics[width=0.48\textwidth]{../figs/0.001_40/Cl_t}}
		\subfigure[$Re=45$]{\includegraphics[width=0.48\textwidth]{../figs/0.001_45/Cl_t}}
	\end{minipage}
	\vspace{0.2em}
	\caption{$Da=0.01$ 时不同雷诺数下升力系数随时间的变化(尚无)}
	\label{fig: Cl_t-1e-2}
\end{figure}

同时,通过直接分析误差随时间的变化也可以判断流动状态。图~\ref{fig: resd}显示了 $Da=0.0001$ 时不同雷诺数下在计算过程中误差随时间步数的变化(包括 Resor0、Resor1、Resor2)。从图中可以看出,当 $Re=40$ 时,随着计算的进行,误差越来越小,并最终达到一个接近零的极小值,于是可以认为流动已经处于稳定状态;当 $Re=45$ 时,随着计算的进行,误差早早达到一个稳定波动的状态,但不会接近零,所以此时流动处于非稳态。由此可知临界雷诺数大于 40 而小于 45,与通过升力得到的结果一致。

\begin{figure}
	\centering
	\begin{minipage}{\textwidth}
		\centering
		\subfigure[$Re=40$]{\includegraphics[width=0.8\textwidth]{../figs/0.0001_40/resd}}
	\end{minipage}
	\centering
	\begin{minipage}{\textwidth}
		\centering
		\subfigure[$Re=45$]{\includegraphics[width=0.8\textwidth]{../figs/0.0001_45/resd}}
	\end{minipage}
	\caption{$Da=0.0001$ 时不同雷诺数下误差随时间的变化}
	\label{fig: resd}
\end{figure}

当流动处于稳态时,误差基本不再变化,当流动处于非稳态时,误差随时间呈周期性变化,计算出误差在一个周期内的平均值,作为该达西数和雷诺数下的平均误差,画出不同达西数下平均误差随雷诺数的变化,如图~\ref{fig: error}所示。从图中可以看出,不同达西数下误差的变化趋势基本一致。Resor2 都接近零,Resor0、Resor1都随着雷诺数的增大而增大,伴随着流动由稳态转变为非稳态,可以设定一个精度,当误差大于改值时流动状态发生改变,由此可以得到临界雷诺数。

\begin{figure}
	\centering
	\includegraphics[width=0.8\textwidth]{../analysis/meanError_Re}
	\caption{平均误差随 $Re$ 和 $Da$ 的变化}
	\label{fig: error}
\end{figure}

同理,对 $Da=1\times 10^{-5},\,0.0005\,0.001,\,0.005,\,0.01$,也可以得到相应的临界雷诺数。最终得到的临界雷诺数见表~\ref{tab: critical Re}。从表中可以看到,达西数越大,临界雷诺数也越大,当达西数小于 $10^{-3}$ 时,临界雷诺数小于 45,当达西数达到 $10^{-2}$ 时,临界雷诺数已在 150 附近,当达西数等于 0.1 时,雷诺数 200 以下已经完全处于稳态,由于 雷诺数 200 以上具有一定的三维效应,所以不再计算。达西数为无量纲的渗透率,不同达西数下临界雷诺数具有不同的数值,这反映了达西数对流动的影响。当达西数很小时,多孔区几乎为固态,只有极少量的流体可以穿过圆柱,此时,临界雷诺数等于固体圆柱绕流的值。随着达西数的增大,多孔介质的影响体现出来,流体可以从圆柱内部穿过,一定程度上减弱了原来的不稳定。达西数很大时,多孔介质接近消失,相当于整个区域都是纯流体的流动,没有了圆柱内部的渗流以及多孔介质的阻碍,流动更接近稳态,需要更大的雷诺数才能由稳态转变为非稳态。

\begin{table}[h]
	\caption{不同达西数下的临界雷诺数}\label{tab: critical Re}
	\vspace{.5em}\centering\wuhao
	\begin{tabular}{cc}
		\toprule[1.5pt]
		$Da$ & 临界 $Re$ \\
		\midrule[1pt]
		$1\times 10^{-5}$ & 40--45 \\
		$0.0001$ & 40--45 \\
		$0.001$ & 40--45 \\
		$0.005$ & -- \\
		$0.01$ & 140--160 \\
		$0.1$ & >200 \\
	\bottomrule[1.5pt]
	\end{tabular}
\end{table}

\section{达西数的影响}\label{sec: Da}

\section{本章小结}

本章通过计算结果判断流动从稳态(L2)转变为非稳态(L3)的临界雷诺数,不同达西数下确定非稳态流动的具体范围。达西数较小时,雷诺数位于 40 到 45 之间,随着达西数的增大,临界雷诺数也逐渐增大,当达西数增大到 0.1 时,临界雷诺数已在 200 之上。另外,随着雷诺数的增大,尾迹的长度逐渐变小。同一雷诺数下,更大的达西数则对应更缓慢的波动。
