% !Mode:: "TeX:UTF-8"
% !TEX root = ../main.tex
\chapter{多孔圆柱受力分析}\label{chap: force}

非稳态层流流动中的几个重要变量是 Strouhal 数、阻力和升力系数,Strouhal 数体现了流体的非稳态流动所具有的频率,通过阻力和升力系数则可以知晓圆柱受到的阻力和升力,在工程中的一些多孔绕流问题中具有一定意义。

表~\ref{tab: results-1e-5} $\sim$ \ref{tab: results-1e-3} 列出了达西数 $10^{-5}$--$10^{-3}$、雷诺数 45--200 范围内二维流动的计算结果,其他参数的设置同 \ref{subsec: grid-independent} 小节中的表~\ref{tab: parameters} 保持一致。表中第二列数据表示非稳态流动的无量纲频率 $St$,第三到第五列表示圆柱所受的阻力系数,其中阻力系数等于压差阻力和摩擦阻力系数之和,$C_D=C_{Dp}+C_{Df}$,第六到第八列表示圆柱所受的方均根升力系数,其中升力波动由压差和摩擦共同引起,但不等于二者之和。

\begin{table}[ht]
	\caption{不同雷诺数下的计算结果($Da=1\times 10^{-5}$)}\label{tab: results-1e-5}
	\vspace{.5em}\centering\wuhao
	\begin{tabular}{*{8}{c}}
		\toprule[1.5pt]
		$Re$ & $St$ & $C_D$ & $C_{Dp}$ & $C_{Df}$ & $C_{L'}$ & $C_{L'p}$ & $C_{L'f}$ \\
		\midrule[1pt]
		45  & 0.1160 & 1.3671 & 0.9695 & 0.3977 & 0.0006 & 0.0006 & 0.0001 \\
		50  & 0.1239 & 1.3495 & 0.9702 & 0.3793 & 0.0358 & 0.0309 &	0.0054 \\
		60  & 0.1351 & 1.3150 & 0.9675 & 0.3474 & 0.0786 & 0.0687 & 0.0112 \\
		70  & 0.1460 & 1.3041 & 0.9789 & 0.3252 & 0.1314 & 0.1163 &	0.0172 \\
		80  & 0.1531 & 1.2894 &	0.9840 & 0.3054 & 0.1616 & 0.1442 &	0.0198 \\
		90  & 0.1592 & 1.2772 &	0.9887 & 0.2885 & 0.1922 & 0.1728 &	0.0222 \\
		100 & 0.1650 & 1.2724 &	0.9976 & 0.2749 & 0.2231 & 0.2018 &	0.0244 \\
		120 & 0.1733 & 1.2646 &	1.0127 & 0.2520 & 0.2792 & 0.2550 &	0.0278 \\
		140 & 0.1805 & 1.2676 &	1.0328 & 0.2348 & 0.3322 & 0.3056 &	0.0305 \\
		160 & 0.1866 & 1.2739 &	1.0532 & 0.2207 & 0.3825 & 0.3540 &	0.0327 \\
		180 & 0.1916 & 1.2836 &	1.0745 & 0.2091 & 0.4296 & 0.3996 &	0.0345 \\
		200 & 0.1957 & 1.2938 &	1.0947 & 0.1991 & 0.4727 & 0.4415 &	0.0358 \\
		\bottomrule[1.5pt]
	\end{tabular}
\end{table}

\begin{table}
	\caption{不同雷诺数下的计算结果($Da=0.0001$)}\label{tab: results-1e-4}
	\vspace{.2em}\centering\wuhao
	\begin{tabular}{*{8}{c}}
		\toprule[1.5pt]
		$Re$ & $St$ & $C_D$ & $C_{Dp}$ & $C_{Df}$ & $C_{L'}$ & $C_{L'p}$ & $C_{L'f}$ \\
		\midrule[1pt]
		45  & 0.1164 & 1.3849 & 0.9814 & 0.4035 & 0.0030 & 0.0026 & 0.0004 \\
		50  & 0.1241 & 1.3681 & 0.9846 & 0.3835 & 0.0367 & 0.0317 & 0.0056 \\
		60  & 0.1361 & 1.3393 & 0.9898 & 0.3495 & 0.0810 & 0.0710 & 0.0116 \\
		70  & 0.1468 & 1.3331 & 1.0082 & 0.3250 & 0.1334 & 0.1185 & 0.0174 \\
		80  & 0.1541 & 1.3236 & 1.0205 & 0.3031 & 0.1656 & 0.1484 & 0.0202 \\
		90  & 0.1603 & 1.3192 & 1.0345 & 0.2848 & 0.1988 & 0.1795 & 0.0228 \\
		100 & 0.1656 & 1.3198 & 1.0505 & 0.2693 & 0.2322 & 0.2110 & 0.0251 \\
		120 & 0.1742 & 1.3287 & 1.0849 & 0.2438 & 0.2934 & 0.2694 & 0.0285 \\
		140 & 0.1808 & 1.3457 & 1.1219 & 0.2238 & 0.3512 & 0.3250 & 0.0313 \\
		160 & 0.1866 & 1.3683 & 1.1608 & 0.2074 & 0.4056 & 0.3778 & 0.0333 \\
		180 & 0.1916 & 1.3932 & 1.1996 & 0.1936 & 0.4543 & 0.4254 & 0.0346 \\
		200 & 0.1957 & 1.4192 & 1.2375 & 0.1817 & 0.4987 & 0.4692 & 0.0355 \\
		\bottomrule[1.5pt]
	\end{tabular}
\end{table}

\begin{table}
	\caption{不同雷诺数下的计算结果($Da=0.0005$)}\label{tab: results-5e-4}
	\vspace{.2em}\centering\wuhao
	\begin{tabular}{*{8}{c}}
		\toprule[1.5pt]
		$Re$ & $St$ & $C_D$ & $C_{Dp}$ & $C_{Df}$ & $C_{L'}$ & $C_{L'p}$ & $C_{L'f}$ \\
		\midrule[1pt]
		45  & 0.1164 & 1.3992 & 1.0265 & 0.3727 & 0.0006 & 0.0005 & 0.0001 \\
		50	& 0.1211 & 1.3697 & 1.0223 & 0.3473 & 0.0208 & 0.0182 & 0.0030 \\
		60	& 0.1377 & 1.3847 & 1.0681 & 0.3166 & 0.0857 & 0.0763 & 0.0115 \\
		70	& 0.1479 & 1.3945 &	1.1047 & 0.2897 & 0.1325 & 0.1195 & 0.0162 \\
		80	& 0.1550 & 1.3999 &	1.1337 & 0.2662 & 0.1639 & 0.1492 &	0.0186 \\
		90	& 0.1610 & 1.4105 &	1.1639 & 0.2466 & 0.1953 & 0.1791 &	0.0206 \\
		100	& 0.1661 & 1.4240 &	1.1940 & 0.2299 & 0.2237 & 0.2065 &	0.0221 \\
		120	& 0.1739 & 1.4523 &	1.2495 & 0.2028 & 0.2703 & 0.2520 &	0.0237 \\
		140	& 0.1802 & 1.4788 &	1.2972 & 0.1817 & 0.3051 & 0.2866 &	0.0241 \\
		160	& 0.1852 & 1.4992 &	1.3346 & 0.1646 & 0.3286 & 0.3104 &	0.0236 \\
		180	& 0.1883 & 1.5117 &	1.3615 & 0.1503 & 0.3435 & 0.3260 &	0.0226 \\
		200 & 0.1912 & 1.5162 &	1.3780 & 0.1381 & 0.3520 & 0.3354 &	0.0214 \\
		\bottomrule[1.5pt]
	\end{tabular}
\end{table}

\begin{table}
	\caption{不同雷诺数下的计算结果($Da=0.001$)}\label{tab: results-1e-3}
	\vspace{.2em}\centering\wuhao
	\begin{tabular}{*{8}{c}}
		\toprule[1.5pt]
		$Re$ & $St$ & $C_D$ & $C_{Dp}$ & $C_{Df}$ & $C_{L'}$ & $C_{L'p}$ & $C_{L'f}$ \\
		\midrule[1pt]
		45  & 0.1171 & 1.4137 & 1.0682 & 0.3455 & 0.0063 & 0.0055 & 0.0009 \\
		50  & 0.1253 & 1.4114 & 1.0872 & 0.3242 & 0.0346 & 0.0307 & 0.0049 \\
		60  & 0.1376 & 1.4089 & 1.1208 & 0.2881 & 0.0765 & 0.0689 & 0.0099 \\
		70  & 0.1477 & 1.4228 & 1.1618 & 0.2609 & 0.1126 & 0.1028 & 0.0132 \\
		80  & 0.1548 & 1.4285 & 1.1910 & 0.2374 & 0.1343 & 0.1235 & 0.0145 \\
		90  & 0.1603 & 1.4361 & 1.2181 & 0.2180 & 0.1531 & 0.1418 & 0.0153 \\
		100 & 0.1653 & 1.4433 & 1.2416 & 0.2017 & 0.1676 & 0.1562 & 0.0156 \\
		120 & 0.1724 & 1.4522 & 1.2768 & 0.1754 & 0.1847 & 0.1736 & 0.0152 \\
		140 & 0.1783 & 1.4544 & 1.2992 & 0.1552 & 0.1922 & 0.1818 & 0.0142 \\
		160 & 0.1821 & 1.4502 & 1.3109 & 0.1392 & 0.1921 & 0.1827 & 0.0128 \\
		180 & 0.1855 & 1.4415 & 1.3153 & 0.1262 & 0.1884 & 0.1798 & 0.0115 \\
		200 & 0.1887 & 1.4299 & 1.3144 & 0.1155 & 0.1821 & 0.1744 & 0.0102 \\
		\bottomrule[1.5pt]
	\end{tabular}
\end{table}

\section{阻力}\label{sec: drag}

图~\ref{fig: ClCd_t-1e-4} 显示了非稳态层流中在特定达西数与雷诺数下阻力和升力系数随时间的变化。对于每一个雷诺数和达西数,经过一段时间,流动逐渐进入周期性平稳状态。设定的计算时长普遍为 300 秒。计算过程中有时无法在 300 秒内达到平稳状态,例如,对 $Da=1\times 10^{-5}$,当 $40<Re<60$ 时在 300 秒处波动的振幅还在变化,未达到平稳状态;$Re=60$ 时可以在 150 秒稳定下来,此后所需的时间逐渐减少,$Re=100$ 时所需时间为 70 秒。

\begin{figure}
	\setlength{\subfigcapskip}{-1bp}
	\centering
	\begin{minipage}{\textwidth}
		\centering
		\subfigure[$Re=50$]{\includegraphics[width=0.47\textwidth]{../analysis/Cl/old/{0.0001_50}.pdf}\includegraphics[width=0.47\textwidth]{../analysis/Cd/old/{0.0001_50}.pdf}}
	\end{minipage}
	\centering
	\begin{minipage}{\textwidth}
		\centering
		\subfigure[$Re=70$]{\includegraphics[width=0.47\textwidth]{../analysis/Cl/old/{0.0001_70}.pdf}\includegraphics[width=0.47\textwidth]{../analysis/Cd/old/{0.0001_70}.pdf}}
	\end{minipage}
	\centering
	\begin{minipage}{\textwidth}
		\centering
		\subfigure[$Re=100$]{\includegraphics[width=0.48\textwidth]{../analysis/Cl/old/{0.0001_100}.pdf}\includegraphics[width=0.48\textwidth]{../analysis/Cd/old/{0.0001_100}.pdf}}
	\end{minipage}
	\centering
	\begin{minipage}{\textwidth}
		\centering
		\subfigure[$Re=200$]{\includegraphics[width=0.48\textwidth]{../analysis/Cl/old/{0.0001_200}.pdf}\includegraphics[width=0.48\textwidth]{../analysis/Cd/old/{0.0001_200}.pdf}}
	\end{minipage}
	\vspace{0.2em}
	\caption{$Da=0.0001$ 时升力系数和阻力系数随时间的变化}
	\label{fig: ClCd_t-1e-4}
\end{figure}

%非稳态情形下阻力和升力系数都随时间作周期性变化,所以此处取一个周期内的平均值。平均阻力系数随雷诺数的变化如图~\ref{fig: meanCd} 所示。当 $Da=1\times 10^{-5}$ 时,阻力系数先随雷诺数的增大而减小,在大约 $Re=120$ 时达到最小值,之后 $Re=120$--$200$ 的范围内,阻力系数随着雷诺数的增加而缓慢地增大。当 $Da=0.0001$ 时,从 $Re=45$ 开始,阻力系数下降得较快,与 $Da=1\times 10^{-5}$ 的下降速率相当,当 $Re=90$ 时,阻力系数达到了极小值,此时阻力系数的值比 $Re=45$ 降低了 4.74\%,随后在 $Re=90$--$200$ 的范围内,阻力系数不断增大,增速比 $Da=1\times 10^{-5}$ 的情形快许多,并且最终超过了 $Re=45$ 处的值。当 $Da=0.001$ 时,阻力系数的变化有所不同。在 $Re=60$ 之前,阻力系数随着雷诺数的增大而减小,在 $Re=60$ 之后,阻力系数突然开始增大,并于 $Re=140$ 时达到了最大值,相比 $Re=60$ 时增大了 3.23\%,随后才逐渐下降。所以,在该达西数下,阻力在 $Re=60$--$200$ 区间内的变化趋势和其他情形完全相反,可能是多孔介质渗流增大所导致的。%变化曲线可能是错的,和之前一样,不知道原因。计算得出的原始数据里就是如此:同一雷诺数下,达西数越大,阻力也越大(将阻力都乘以-2),不知道计算方法是否有误。

%观察同一雷诺数下不同达西数对应的阻力系数,可以得知多孔介质的存在对流体流动的影响。在 $Re=45$--$200$ 范围内,固定雷诺数,达西数越大则阻力系数越大。$Re=45$ 时,$Da=1\times 10^{-5},\,0.0001,\,0.001$ 的阻力系数之比为 $1:1.013:1.034$,此时雷诺数很小,多孔介质的影响并不明显。随着雷诺数的增大,这一差距也逐渐拉大,$Re=120$ 时,三者之比达到了 $1:1.05:1.15$,$Re=200$ 时的比例为 $1:1.10:1.11$,由于 $Da=0.0001$ 和 0.001 变化趋势相反,所以二者逐渐接近。L.-C. Hsu\cite{Hsu2016} 利用其他模拟方法研究了二维圆柱绕流,得到了各个参数的变化情况。%?

% \begin{figure}
% 	\centering
% 	\includegraphics[width=0.8\textwidth]{../analysis/Cd_Re}
% 	\caption{平均阻力系数 $C_D$ 随 $Re$ 的变化}
% 	\label{fig: meanCd}
% \end{figure}

% 阻力系数定义
% \begin{equation}
% 	C_D = \frac{D}{\frac12\rho U_{\infty}^2 A}
% \end{equation}
当流体绕过钝体流动时,流体的粘性作用形成了摩擦阻力 $C_{Df}$,物体前后的压强差形成了压差阻力 $C_{Dp}$。摩擦阻力为作用在物体表面的切向力在来流方向的分量之和,压差阻力是作用在物体表面的法向力在来流方向的分量之和,二者共同形成了物体所受的阻力 $C_{D}$。表示如下:
\begin{equation}
	C_D = C_{Dp} + C_{Df}
\end{equation}
压力系数定义为 $C_p = \frac{p-p_{\infty}}{\frac12\rho U_{\infty}^2}$。
图~\ref{fig: meanCdpf} 反映了不同达西数下压差阻力系数和摩擦阻力系数随雷诺数的变化。当流体在物体背面形成尾迹时,流体的能量不断耗散在尾迹的漩涡中,使得物体背面的压强较低,物体前后产生压强差,形成了压差阻力。尾迹区域越大则耗散越强,压差阻力也就越大。对于流线型物体,流体几乎顺着物体表面流动,在物体末端才出现分离现象,所以它的阻力主要来源于摩擦阻力。钝体受到的压差阻力则相对较大,流体分离点的位置越靠前,尾迹区越大,压差阻力也越大。对于圆柱绕流,从图中可以看出,在 $Re=45$--$200$ 区间内,压差阻力显著大于摩擦阻力,说明阻力主要由流动的尾迹造成。随着雷诺数的增大,$Da=1\times 10^{-5}$ 时压差阻力占总阻力的比例从 $Re=45$ 时的 71\% 增加到了 $Re=200$ 时的 85\%;$Da=0.0001$ 下从 71\% 增加到了 87\%;$Da=0.001$ 下更是从 76\% 增加到了 92\%。随着雷诺数的增大,分离点向上游移动,尾迹区扩大,压差阻力增大。%与图~\ref{fig: meanCdpf} 中 $C_{Dp}$ 的变化一致。摩擦阻力则随着雷诺数的增大而减小。%此图中还是达西数越大,阻力越大,可能不相符。

%对于不同的达西数,多孔圆柱受到的阻力不尽相同。同一雷诺数下,达西数越大,则压差阻力也越大。同一雷诺数下,达西数越大,摩擦阻力越小。%图有问题

\begin{figure}
	\centering
	\includegraphics[width=0.8\textwidth]{../analysis/Cdpf_Re}
	\caption{平均压差阻力系数 $C_{Dp}$ 和平均摩擦阻力系数 $C_{Df}$ 随 $Re$ 的变化}
	\label{fig: meanCdpf}
\end{figure}

\section{升力}\label{sec: lift}

% 升力系数定义
% \begin{equation}
% 	C_L = \frac{L}{\frac12\rho U_{\infty}^2 A}
% \end{equation}
升力系数的平均值为零,此处取其方均根值。方均根升力系数 $C_{L'}$ 随达西数和雷诺数的变化如图~\ref{fig: rmsCl} 所示。$C_{L'}$ 反映了升力系数波动的振幅,亦即涡脱落的剧烈程度,由图可知,$C_{L'}$ 随雷诺数的增加而增加,雷诺数增大时涡脱落也变得更加剧烈。Lu Lin 等人 \cite{Lu2011} 发现通过流量控制可以减小圆柱绕流时的这一升力波动。文献 \inlinecite{Qu2013} 得出,当 $Re=50,\,60$ 时 $C_{L'}$ 的大小正比于 $\sqrt{Re}$。当雷诺数较小时(50--60),$C_{L'}$ 与雷诺数的关系为 $0.17\sqrt{\epsilon}$,其中 $\epsilon=Re/Re_c-1$,$Re_c$ 取值 47.4 \cite{Norberg1994,Kumar2006}。在整个雷诺数范围内,升力系数的波动主要由壁面上的压力引起,壁面摩擦力导致的波动只占一小部分。根据计算结果,$Da=1\times 10^{-5}$ 时,比值 $C_{L'p}/C_{L'}$ 从 $Re=45$ 时的 80\% 增加到了 $Re=200$ 时的 93\%;$Da=0.0001$ 时从 $Re=45$ 的 87\% 增加到了 $Re=200$ 时的 94\%;$Da=0.001$ 时从 $Re=45$ 的 87\% 增加到了 $Re=200$ 时的 96\%。与之相反,比值 $C_{L'f}/C_{L'}$ 随着雷诺数下降,并和雷诺数的平方根成反比。

Park 等人 \cite{Park1998} 判断,在由压强和摩擦引起的升力波动之间存在一个相位的转变。随时间变化的三个升力系数具有如下关系(此处表示瞬时值):
\begin{equation}\label{eq: C_L}
	C_L = C_{Lp} + C_{Lf}
\end{equation}
$C_{L'}$ 为 $C_L$ 的方均根:
\begin{equation}\label{eq: C_L'}
	C_{L'} = \sqrt{\frac{1}{T}\int_{t_0}^{t_0+T}C_L^2\diff t} \approx
	\sqrt{\frac{1}{N}\sum_{n=1}^N C_{L,n}^2}
\end{equation}
其中区间 [$t_0$, $t_0+T$] 表示一个振荡周期,在实际计算中,一个完整周期内的积分变成了该周期内 $N$ 个离散数值点的求和(第二个等号后的式子)。
由式 \eqref{eq: C_L} 和 \eqref{eq: C_L'} 可得
\begin{equation}\label{eq: C_L'-C_L}
	C_{L'}^2 = C_{L'p}^2 + C_{L'f}^2 + 2\cdot\frac{1}{T}\int_{t_0}^{t_0+T}C_{Lp}C_{Lf}\diff t
\end{equation}
如果升力的波动可以看作正弦曲线,并且压强和摩擦之间的相位差是一个常数 $\phi$ 的话,式 \eqref{eq: C_L'-C_L} 可进一步写为三个方均根升力系数之间的关系
\begin{equation}
	C_{L'}^2 = C_{L'p}^2 + C_{L'f}^2 + 2C_{L'p}C_{L'f}\cos\phi
\end{equation}
文献 \inlinecite{Qu2013} 从数据计算得出的 $\phi$ 大约为 $30^\circ$。

\begin{figure}
	\centering
	\includegraphics[width=0.8\textwidth]{../analysis/Clpf_Re}
	\caption{方均根升力系数 $C_{L'}$、$C_{L'p}$ 和 $C_{L'f}$ 随 $Re$ 的变化}
	\label{fig: rmsCl}
\end{figure}

\section{Strouhal 数}\label{sec: St}

对于非稳态流动,流动波动的频率是一个重要参数,在工程中具有重要价值。作为表述流体波动的无量纲频率,Strouhal 数是流动振荡不稳定性的度量:
\begin{equation}\label{eq: St}
	St = \frac{Df}{U_{\infty}}
\end{equation}
\begin{tabularx}{\textwidth}{@{}l@{\quad}r@{——}X@{}}
	式中 & $St$ & Strouhal 数;\\
		& $D$ & 圆柱的直径;\\
		& $f$ & 波动的频率;\\
		& $U_{\infty}$ & 远方的来流速度。 
\end{tabularx}\vspace{3.15bp}
由于计算时设定 $D=1\,\mathrm{m}$,$U=1\,\mathrm{m/s}$,于是 $St=f=1/T$,
频率与周期互为倒数,周期可以根据某一物理量(例如 Point 3 的水平速度)随时间的变化曲线得到。继而根据式 \eqref{eq: St} 可以得到 Strouhal 数。

图~\ref{fig: St} 显示了 Strouhal 数随着雷诺数和达西数的变化。从图中可以看出,$Da$ 固定时,$St$ 随 $Re$ 的增大而增大,且增长得越来越缓慢。在指定圆柱大小和流体种类的前提下,雷诺数由来流速度决定,来流速度越大,流体在圆柱背面形成的漩涡产生及脱落的频率也越大,符合直观的猜想。固定 $Re$ 时,$Da$ 越小,则 $St$ 越接近固体圆柱绕流的情形,随着达西数趋于零,得到的一系列曲线也将趋于某一条极限曲线,即为固体情形下的曲线。当 $Da=1 \times 10^{-5}$ 时,$St$ 曲线已经和固体圆柱绕流的变化曲线相吻合,见 \ref{sec: result validation} 节中图~\ref{fig: validation-St} 的比较。

\begin{figure}
	\centering
	\includegraphics[width=0.8\textwidth]{../analysis/St_Re}
	\caption{不同达西数下 Strouhal 数随 $Re$ 的变化}
	\label{fig: St}
\end{figure}

对随时间变化的物理量做 Fourier 分析,可以得到离散频率的频谱。已知某个样本点的速度、圆柱的阻力和升力系数随时间变化的数据,对这些物理量做快速 Fourier 变换,可以得到这些量在频域中的分布。频域的横坐标是频率,也是 Strouhal 数,某些频率上有峰值出现,表示此处能量密度很高,能量集中在这个频率附近。Fourier 变换之后可能有多个频率分量,类似谐波,形成若干离散的峰值,这些峰值按一定的间距分布,大小不一。%?

以 $Da=0.0001$、$Re=100$ 为例,水平速度、垂直速度、升力系数随时间的变化以及相应的 Fourier 变换如图~\ref{fig: spectrum-uvcl} 所示,图中画出了稳定时的 10 个周期。其他参数下的图没有列出。

当 $Da=1\times 10^{-5}$ 时,第三个样本点处的水平速度 $U$ 随时间周期性波动,从图中所示的十个周期可以看到,每个周期内速度都具有一大一小两个小的周期,代表了速度的两个主要频率成分,对应于右图中第二个峰值和第三个峰值($f=0$ 处的第一个峰值反映的是 $U$ 的时间平均值,即左图中的水平直线)。右图第三个峰值之后还有一系列更小的峰值,由于值太小而无法表现出来,说明速度不仅具有左图中可见的两个频率,还具有无数的频率分量,只是所占比例很小,可以忽略。右图中第二个频率即为该达西数和雷诺数下流体运动的 Strouhal 数,与前文结果一致。从 $Re=50$ 到 $Re=200$,随着雷诺数的增加,$f=0$ 处的峰值增大,说明这一点的水平速度增大;主频率向右移动,即 Strouhal 数逐渐增大,与前文所述一致。V. Babua 和 Arunn Narasimhan \cite{Babu2010} 对二维多孔方柱绕流的数值结果进行了能谱分析,可以看出当前研究与之具有相似的结果和变化趋势。%?

图~\ref{fig: spectrum-Cl}、\ref{fig: spectrum-U} 和~\ref{fig: spectrum-V} 分别显示了不同达西数和雷诺数下升力系数、水平速度、垂直速度随时间的变化经 Fourier 变换后所得的频谱图。主频率对应的单个尖峰表明尾迹中出现了剧烈的涡脱落。由图~\ref{fig: spectrum-Cl} 可知,当 $Da=1\times 10^{-5}$ 时,多孔介质接近固体,频谱图中会有一个占优势地位的明显的峰值,和固体的情形相匹配。随着达西数的增加,$Da=0.0001$ 时,频谱图中主频所对应的峰值振幅变大了,表明尾迹中涡脱落的强度增强,当 $Da=0.001$ 时,峰值减小了许多,表明尾迹中涡脱落的强度减弱了许多。达西数继续提高时,主频的峰值将越来越不明显,直到在整个频率范围内都没有明显的突出。固定达西数,观察不同雷诺数下的频谱图,随着雷诺数的增加,峰值振幅也随之增大,即更大的雷诺数对应着更强的涡脱落程度。图~\ref{fig: spectrum-U} 和图~\ref{fig: spectrum-V} 具有相同的趋势。图~\ref{fig: spectrum-U} 在主频率周围还出现了其他的峰值,因为水平速度的时变曲线不是简单的周期性,如图~\ref{fig: spectrum-uvcl} a) 所示。图~\ref{fig: spectrum-uvcl} 以 $Da=0.0001$、$Re=100$ 为例,展示了物理量随时间的变化曲线和相应的频谱图,右侧的频谱图是对左侧 10 个周期内的曲线进行变换得到的,峰值情形与前面的图相一致。%?

% \begin{figure}
% 	\centering
% 	\begin{minipage}{\textwidth}
% 		\centering
% 		\subfigure[$Re=50$]{\includegraphics[width=0.6\textwidth]{../figs/0.0001_50/psd_U}}
% 	\end{minipage}
% 	\centering
% 	\begin{minipage}{\textwidth}
% 		\centering
% 		\subfigure[$Re=90$]{\includegraphics[width=0.6\textwidth]{../figs/0.0001_90/psd_U}}
% 	\end{minipage}
% 	\centering
% 	\begin{minipage}{\textwidth}
% 		\centering
% 		\subfigure[$Re=140$]{\includegraphics[width=0.6\textwidth]{../figs/0.0001_140/psd_U}}
% 	\end{minipage}
% 	\centering
% 	\begin{minipage}{\textwidth}
% 		\centering
% 		\subfigure[$Re=200$]{\includegraphics[width=0.6\textwidth]{../figs/0.0001_200/psd_U}}
% 	\end{minipage}
% 	\caption{坐标 (3,0) 处 $Da=0.0001$ 时不同雷诺数下水平速度 $U$ 的 Fourier 变换}
% 	\label{fig: U Fourier-1e-4}
% \end{figure}

% \begin{figure}
% 	\centering
% 	\begin{minipage}{\textwidth}
% 		\centering
% 		\subfigure[$Re=50$]{\includegraphics[width=0.6\textwidth]{../figs/0.001_50/psd_U}}
% 	\end{minipage}
% 	\centering
% 	\begin{minipage}{\textwidth}
% 		\centering
% 		\subfigure[$Re=90$]{\includegraphics[width=0.6\textwidth]{../figs/0.001_90/psd_U}}
% 	\end{minipage}
% 	\centering
% 	\begin{minipage}{\textwidth}
% 		\centering
% 		\subfigure[$Re=140$]{\includegraphics[width=0.6\textwidth]{../figs/0.001_140/psd_U}}
% 	\end{minipage}
% 	\centering
% 	\begin{minipage}{\textwidth}
% 		\centering
% 		\subfigure[$Re=200$]{\includegraphics[width=0.6\textwidth]{../figs/0.001_200/psd_U}}
% 	\end{minipage}
% 	\caption{坐标 (3,0) 处 $Da=0.001$ 时不同雷诺数下水平速度 $U$ 的 Fourier 变换}
% 	\label{fig: U Fourier-1e-3}
% \end{figure}

% \begin{figure}
% 	\centering
% 	\begin{minipage}{\textwidth}
% 		\centering
% 		\subfigure[$Re=50$]{\includegraphics[width=0.6\textwidth]{../figs/0.0001_50/psd_V}}
% 	\end{minipage}
% 	\centering
% 	\begin{minipage}{\textwidth}
% 		\centering
% 		\subfigure[$Re=90$]{\includegraphics[width=0.6\textwidth]{../figs/0.0001_90/psd_V}}
% 	\end{minipage}
% 	\centering
% 	\begin{minipage}{\textwidth}
% 		\centering
% 		\subfigure[$Re=140$]{\includegraphics[width=0.6\textwidth]{../figs/0.0001_140/psd_V}}
% 	\end{minipage}
% 	\centering
% 	\begin{minipage}{\textwidth}
% 		\centering
% 		\subfigure[$Re=200$]{\includegraphics[width=0.6\textwidth]{../figs/0.0001_200/psd_V}}
% 	\end{minipage}
% 	\caption{坐标 (3,0) 处 $Da=0.0001$ 时不同雷诺数下竖直速度 $V$ 的 Fourier 变换}
% 	\label{fig: V Fourier-1e-4}
% \end{figure}

% \begin{figure}
% 	\centering
% 	\begin{minipage}{\textwidth}
% 		\centering
% 		\subfigure[$Re=50$]{\includegraphics[width=0.6\textwidth]{../figs/0.001_50/psd_V}}
% 	\end{minipage}
% 	\centering
% 	\begin{minipage}{\textwidth}
% 		\centering
% 		\subfigure[$Re=90$]{\includegraphics[width=0.6\textwidth]{../figs/0.001_90/psd_V}}
% 	\end{minipage}
% 	\centering
% 	\begin{minipage}{\textwidth}
% 		\centering
% 		\subfigure[$Re=140$]{\includegraphics[width=0.6\textwidth]{../figs/0.001_140/psd_V}}
% 	\end{minipage}
% 	\centering
% 	\begin{minipage}{\textwidth}
% 		\centering
% 		\subfigure[$Re=200$]{\includegraphics[width=0.6\textwidth]{../figs/0.001_200/psd_V}}
% 	\end{minipage}
% 	\caption{坐标 (3,0) 处 $Da=0.001$ 时不同雷诺数下竖直速度 $V$ 的 Fourier 变换}
% 	\label{fig: V Fourier-1e-3}
% \end{figure}

% \begin{figure}
% 	\centering
% 	\begin{minipage}{\textwidth}
% 		\centering
% 		\subfigure[$Re=50$]{\includegraphics[width=0.6\textwidth]{../figs/0.0001_50/psd_Cl}}
% 	\end{minipage}
% 	\centering
% 	\begin{minipage}{\textwidth}
% 		\centering
% 		\subfigure[$Re=90$]{\includegraphics[width=0.6\textwidth]{../figs/0.0001_90/psd_Cl}}
% 	\end{minipage}
% 	\centering
% 	\begin{minipage}{\textwidth}
% 		\centering
% 		\subfigure[$Re=140$]{\includegraphics[width=0.6\textwidth]{../figs/0.0001_140/psd_Cl}}
% 	\end{minipage}
% 	\centering
% 	\begin{minipage}{\textwidth}
% 		\centering
% 		\subfigure[$Re=200$]{\includegraphics[width=0.6\textwidth]{../figs/0.0001_200/psd_Cl}}
% 	\end{minipage}
% 	\caption{$Da=0.0001$ 时不同雷诺数下升力系数的 Fourier 变换}
% 	\label{fig: Cl Fourier-1e-4}
% \end{figure}

% \begin{figure}
% 	\centering
% 	\begin{minipage}{\textwidth}
% 		\centering
% 		\subfigure[$Re=50$]{\includegraphics[width=0.6\textwidth]{../figs/0.001_50/psd_Cl}}
% 	\end{minipage}
% 	\centering
% 	\begin{minipage}{\textwidth}
% 		\centering
% 		\subfigure[$Re=90$]{\includegraphics[width=0.6\textwidth]{../figs/0.001_90/psd_Cl}}
% 	\end{minipage}
% 	\centering
% 	\begin{minipage}{\textwidth}
% 		\centering
% 		\subfigure[$Re=140$]{\includegraphics[width=0.6\textwidth]{../figs/0.001_140/psd_Cl}}
% 	\end{minipage}
% 	\centering
% 	\begin{minipage}{\textwidth}
% 		\centering
% 		\subfigure[$Re=200$]{\includegraphics[width=0.6\textwidth]{../figs/0.001_200/psd_Cl}}
% 	\end{minipage}
% 	\caption{$Da=0.001$ 时不同雷诺数下升力系数的 Fourier 变换}
% 	\label{fig: Cl Fourier-1e-3}
% \end{figure}

\begin{figure}
	\centering
	\includegraphics[width=\textwidth]{../analysis/psd/Cl}
	\caption{不同达西数和雷诺数下升力系数 $C_L$ 的频谱图}
	\label{fig: spectrum-Cl}
\end{figure}

\begin{figure}
	\centering
	\includegraphics[width=\textwidth]{../analysis/psd/U}
	\caption{不同达西数和雷诺数下 (3,0) 点水平速度 $U$ 的频谱图}
	\label{fig: spectrum-U}
\end{figure}

\begin{figure}
	\centering
	\includegraphics[width=\textwidth]{../analysis/psd/V}
	\caption{不同达西数和雷诺数下 (3,0) 点垂直速度 $V$ 的频谱图}
	\label{fig: spectrum-V}
\end{figure}

\begin{figure}
	\centering
	\begin{minipage}{\textwidth}
		\centering
		\subfigure[坐标 (3,0) 处的水平速度 $U$]{\includegraphics[width=.9\textwidth]{../analysis/psd/{0.0001_100_U}.pdf}}
	\end{minipage}
	\centering
	\begin{minipage}{\textwidth}
		\centering
		\subfigure[坐标 (3,0) 处的垂直速度 $V$]{\includegraphics[width=.9\textwidth]{../analysis/psd/{0.0001_100_V}.pdf}}
	\end{minipage}
	\centering
	\begin{minipage}{\textwidth}
		\centering
		\subfigure[升力系数 $C_L$]{\includegraphics[width=.9\textwidth]{../analysis/psd/{0.0001_100_Cl}.pdf}}
	\end{minipage}
	\caption{同一雷诺数和达西数下不同物理量的时间序列图及能谱图($Da=0.0001$, $Re=100$)}
	\label{fig: spectrum-uvcl}
\end{figure}

\section{本章小结}

本章分析了 Strouhal 数、阻力系数、升力系数等几个重要的参数。Strouhal 数表征波动的频率,随着雷诺数的增加而增加。阻力和升力都随时间作周期性变化,因而分别取一个周期内的平均值,阻力直接取平均值,升力取方均根值。压差阻力在总阻力中的占比大于摩擦阻力,且二者随雷诺数变化趋势相反。雷诺数增大时压差占比也增大,因为尾迹区在不断扩大。方均根升力系数随雷诺数的增加而增加,表明涡脱落越来越剧烈。计算表明,在升力系数的组成中,由压力引起的波动同样占主要地位,摩擦则居于次位。另外,还可以获得压力和摩擦引起的升力波动之间的相位差距。\ref{sec: St} 节通过对速度、升力系数随时间变化的能谱分析,再次得到了不同参数流动下的 Strouhal 数。涡脱落的频率为频谱图中峰值对应的频率,它的振幅体现了涡脱落的剧烈程度。随着达西数和雷诺数的增加,这个峰值逐渐减小,即涡脱落的强度逐渐减弱。
