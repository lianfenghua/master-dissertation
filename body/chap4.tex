% !Mode:: "TeX:UTF-8"
% !TEX root = ../main.tex
\chapter{流场结果与分析}\label{chap: flow pattern}

流动由雷诺数、达西数、孔隙率等多个参数决定,当前的研究中主要通过改变雷诺数和达西数得到不同的流动,并探究参数对流动的影响。非稳态流动的雷诺数范围大约为从 40 到 200, $Da=1\times 10^{-5}$ 时多孔区域接近固体,$Da=1\times 10^{-2}$ 时多孔介质的效果已经非常微弱,所以当前计算时设置的雷诺数从区间 $[40,200]$ 内选择,达西数从区间 $[10^{-5},10^{-2}]$ 内选择,表~\ref{tab: DaRe} 列出了达西数和雷诺数可能的组合值,实际计算过的值用对勾($\surd$)标记。本文未改变孔隙率和应力阶跃参数的值,依承文献 \inlinecite{Yu2007} 的设定,将孔隙率设为 0.7,将应力阶跃的两个系数均设为零:$\varepsilon=0.7$,$\beta_1 = \beta_2=0$。

\begin{table}
	\caption{计算时用到的达西数和雷诺数的值的设置}\label{tab: DaRe}
	\vspace{.5em}\centering\wuhao
	\begin{tabular}{*{7}{c}}
		\toprule[1.5pt]
		\multirow{2}[3]{*}{$Re$} & \multicolumn{6}{c}{$Da$} \\
		\cmidrule[.67pt](lr){2-7}
		& 0.00001 & 0.0001 & 0.0005 & 0.001 & 0.005 & 0.01 \\
		\midrule[1pt]
		40  & $\surd$ & $\surd$ & $\surd$ & $\surd$ & $\surd$ & -- \\
		41  & --      & $\surd$ & $\surd$ & $\surd$ & $\surd$ & -- \\
		42  & $\surd$ & $\surd$ & $\surd$ & $\surd$ & $\surd$ & -- \\
		43  & $\surd$ & $\surd$ & $\surd$ & $\surd$ & $\surd$ & -- \\
		44  & $\surd$ & $\surd$ & $\surd$ & $\surd$ & $\surd$ & -- \\
		45  & $\surd$ & $\surd$ & $\surd$ & $\surd$ & $\surd$ & -- \\
		50  & $\surd$ & $\surd$ & $\surd$ & $\surd$ & $\surd$ & -- \\
		60  & $\surd$ & $\surd$ & $\surd$ & $\surd$ & $\surd$ & $\surd$ \\
		70  & $\surd$ & $\surd$ & $\surd$ & $\surd$ & $\surd$ & $\surd$ \\
		80  & $\surd$ & $\surd$ & $\surd$ & $\surd$ & $\surd$ & $\surd$ \\
		90  & $\surd$ & $\surd$ & $\surd$ & $\surd$ & $\surd$ & $\surd$ \\
		100 & $\surd$ & $\surd$ & $\surd$ & $\surd$ & $\surd$ & $\surd$ \\
		120 & $\surd$ & $\surd$ & $\surd$ & $\surd$ & $\surd$ & $\surd$ \\
		140 & $\surd$ & $\surd$ & $\surd$ & $\surd$ & $\surd$ & $\surd$ \\
		160 & $\surd$ & $\surd$ & $\surd$ & $\surd$ & $\surd$ & $\surd$ \\
		180 & $\surd$ & $\surd$ & $\surd$ & $\surd$ & $\surd$ & $\surd$ \\
		200 & $\surd$ & $\surd$ & $\surd$ & $\surd$ & $\surd$ & $\surd$ \\
		\bottomrule[1.5pt]
	\end{tabular}
\end{table}

%变了,要改
由结果可知,非稳态流动开始时所对应的雷诺数的值都在 40 之上,研究的第一步在于确定非稳态流动开始发生时所对应的临界雷诺数,从而确定此次研究的雷诺数范围。接着,本章还将展示非稳态流动的整体概貌,即不同参数下的流动特性,大体了解流动随雷诺数、达西数的变化情况。

\section{流场的瞬时分布}\label{sec: transient}

%涡量等值线-远程
经过计算,可以得到不同达西数和雷诺数下整个流动区域的涡量云图,如图~\ref{fig: vortlong-1e-4} $\sim$ 图~\ref{fig: vortlong-1e-2} 所示。图~\ref{fig: vortlong-1e-4} 选取了六幅 $Da=0.0001$ 时不同雷诺数下流动已达平稳时涡量的等值线图。此处“平稳”一词指流动特性随时间做周期性变化,在时间平均意义下是稳定的(periodic but statistically stationary state),而“稳定”指流动特性不随时间发生任何变化,每一时刻的状态都是相同的(steady state)。涡量值的变化范围是 $-0.5$ 到 0.5。圆柱背面具有正、负涡量值的区域成对出现,表示一对对脱落的涡。从图中可以看出,$Re=40$ 时流动处于稳态,并在圆柱的后端产生了一对很长的漩涡,随着雷诺数的增大,尾迹的长度逐渐缩短;$Re=45$ 时尾迹已经开始波动,流动开始处于非稳态,与前文所述一致;$Re=50$ 时尾迹更短,非稳态已比较明显,涡脱落之后向下游移动,形成一列整齐的涡街;之后更是完全进入了非稳态,尾迹的长度也越来越短,一对对的涡向后整齐地排列。当雷诺数达到 200 时,离圆柱较远处的涡已经变得不再规律,可能是因为此时的三维效应已经比较明显,流体存在沿着圆柱长度方向的流动现象。

图~\ref{fig: vortlong-1e-3} 展示了 $Da=0.001$ 时不同雷诺数下涡量的等值线图。总体的变化情况和 $Da=0.0001$ 时相同,除了在更大的雷诺数下达到非稳态。图~\ref{fig: vortlong-1e-2} 展示了 $Da=0.01$ 时不同雷诺数下涡量的等值线图。从图中可以看出,直到 $Re=200$ 时尾迹才具有并不大的波动,雷诺数更小的时候流动基本处于稳态。

Karasudani 和 Funakoshi\cite{Karasudani1994} 在实验中研究了圆柱绕流远尾迹处涡街结构的演化,发现沿着下游经过一段距离之后,初始涡街分解成了几乎平行的剪切流动,具有高斯分布,再经过一段距离,形成了具有更大尺度的第二涡街。图~\ref{fig: vortlong-1e-4} 展示了 $Re=50,70,90,120,160,200$ 时涡量的演化情况。从 $Re=70$ 开始,尾迹中的涡沿着下游逐渐演化成了近乎平行的剪切流动。$Re>50$ 时可以观察到两次卡门涡的失真和旋转。卡门涡大约形成于 $x=5$ 的位置并从圆柱表面脱落,同时涡的中心离开了尾迹的中心线。第二次形成于 $x=20$ 或更远的位置。在 $x=15$ 以后,涡中心距尾迹中心线($y=0$)之间的垂直距离保持不变。沿着尾迹中心线,涡的形状从圆变得扭曲,类似于椭圆形。从 $Re120$ 开始,$x=25$ 之后的远尾迹处相邻的涡逐渐融合起来成为一体。

\begin{figure}
	\setlength{\subfigcapskip}{-1bp}
	\centering
	\includegraphics[width=0.7\textwidth]{../analysis/flow/vort_legend_line}
	\begin{minipage}{\textwidth}
		\centering
		\subfigure[$Re=40$]{\includegraphics[width=0.47\textwidth]{../analysis/flow/{0.0001_40_long_line}.png}}
		\subfigure[$Re=45$]{\includegraphics[width=0.47\textwidth]{../analysis/flow/{0.0001_45_long_line}.png}}
	\end{minipage}
	\centering
	\begin{minipage}{\textwidth}
		\centering
		\subfigure[$Re=50$]{\includegraphics[width=0.47\textwidth]{../analysis/flow/{0.0001_50_long_line}.png}}
		\subfigure[$Re=70$]{\includegraphics[width=0.47\textwidth]{../analysis/flow/{0.0001_70_long_line}.png}}
	\end{minipage}
	\centering
	\begin{minipage}{\textwidth}
		\centering
		\subfigure[$Re=90$]{\includegraphics[width=0.47\textwidth]{../analysis/flow/{0.0001_90_long_line}.png}}
		\subfigure[$Re=120$]{\includegraphics[width=0.47\textwidth]{../analysis/flow/{0.0001_120_long_line}.png}}
	\end{minipage}
	\centering
	\begin{minipage}{\textwidth}
		\centering
		\subfigure[$Re=160$]{\includegraphics[width=0.47\textwidth]{../analysis/flow/{0.0001_160_long_line}.png}}
		\subfigure[$Re=200$]{\includegraphics[width=0.47\textwidth]{../analysis/flow/{0.0001_200_long_line}.png}}
	\end{minipage}
	\vskip 0.2em
	\wuhao 注:图中所示的流动区域为 $-1<x<30$,$-5<y<5$。
	\vspace{0.2em}
	\caption{$Da=0.0001$ 时不同雷诺数下的涡量等值线}
	\label{fig: vortlong-1e-4}
\end{figure}

\begin{figure}
	\setlength{\subfigcapskip}{-1bp}
	\centering
	\includegraphics[width=0.7\textwidth]{../analysis/flow/vort_legend_line}
	\begin{minipage}{\textwidth}
		\centering
		\subfigure[$Re=40$]{\includegraphics[width=0.47\textwidth]{../analysis/flow/{0.001_40_long_line}.png}}
		\subfigure[$Re=45$]{\includegraphics[width=0.47\textwidth]{../analysis/flow/{0.001_45_long_line}.png}}
	\end{minipage}
	\centering
	\begin{minipage}{\textwidth}
		\centering
		\subfigure[$Re=50$]{\includegraphics[width=0.47\textwidth]{../analysis/flow/{0.001_50_long_line}.png}}
		\subfigure[$Re=70$]{\includegraphics[width=0.47\textwidth]{../analysis/flow/{0.001_70_long_line}.png}}
	\end{minipage}
	\centering
	\begin{minipage}{\textwidth}
		\centering
		\subfigure[$Re=90$]{\includegraphics[width=0.47\textwidth]{../analysis/flow/{0.001_90_long_line}.png}}
		\subfigure[$Re=120$]{\includegraphics[width=0.47\textwidth]{../analysis/flow/{0.001_120_long_line}.png}}
	\end{minipage}
	\centering
	\begin{minipage}{\textwidth}
		\centering
		\subfigure[$Re=160$]{\includegraphics[width=0.47\textwidth]{../analysis/flow/{0.001_160_long_line}.png}}
		\subfigure[$Re=200$]{\includegraphics[width=0.47\textwidth]{../analysis/flow/{0.001_200_long_line}.png}}
	\end{minipage}
	\vskip 0.2em
	\wuhao 注:图中所示的流动区域为 $-1<x<30$,$-5<y<5$。
	\vspace{0.2em}
	\caption{$Da=0.001$ 时不同雷诺数下的涡量等值线}
	\label{fig: vortlong-1e-3}
\end{figure}

\begin{figure}
	\setlength{\subfigcapskip}{-1bp}
	\centering
	\includegraphics[width=0.7\textwidth]{../analysis/flow/vort_legend_line}
	\begin{minipage}{\textwidth}
		\centering
		\subfigure[$Re=60$]{\includegraphics[width=0.47\textwidth]{../analysis/flow/{0.01_60_long_line}.png}}
		\subfigure[$Re=70$]{\includegraphics[width=0.47\textwidth]{../analysis/flow/{0.01_70_long_line}.png}}
	\end{minipage}
	\centering
	\begin{minipage}{\textwidth}
		\centering
		\subfigure[$Re=90$]{\includegraphics[width=0.47\textwidth]{../analysis/flow/{0.01_90_long_line}.png}}
		\subfigure[$Re=120$]{\includegraphics[width=0.47\textwidth]{../analysis/flow/{0.01_120_long_line}.png}}
	\end{minipage}
	\centering
	\begin{minipage}{\textwidth}
		\centering
		\subfigure[$Re=160$]{\includegraphics[width=0.47\textwidth]{../analysis/flow/{0.01_160_long_line}.png}}
		\subfigure[$Re=200$]{\includegraphics[width=0.47\textwidth]{../analysis/flow/{0.01_200_long_line}.png}}
	\end{minipage}
	\vskip 0.2em
	\wuhao 注:图中所示的流动区域为 $-1<x<30$,$-5<y<5$。
	\vspace{0.2em}
	\caption{$Da=0.01$ 时不同雷诺数下的涡量等值线}
	\label{fig: vortlong-1e-2}
\end{figure}

%涡量等值线-近程
近程涡量的等值线图如图~\ref{fig: vort-1e-4} $\sim$ 图~\ref{fig: vort-1e-2} 所示。

\begin{figure}
	\setlength{\subfigcapskip}{-1bp}
	\centering
	\includegraphics[width=0.7\textwidth]{../analysis/flow/vort_legend}
	\begin{minipage}{\textwidth}
		\centering
		\subfigure[$Re=40$]{\includegraphics[width=0.47\textwidth]{../analysis/flow/{0.0001_40}.png}}
		\subfigure[$Re=45$]{\includegraphics[width=0.47\textwidth]{../analysis/flow/{0.0001_45}.png}}
	\end{minipage}
	\centering
	\begin{minipage}{\textwidth}
		\centering
		\subfigure[$Re=50$]{\includegraphics[width=0.47\textwidth]{../analysis/flow/{0.0001_50}.png}}
		\subfigure[$Re=70$]{\includegraphics[width=0.47\textwidth]{../analysis/flow/{0.0001_70}.png}}
	\end{minipage}
	\centering
	\begin{minipage}{\textwidth}
		\centering
		\subfigure[$Re=90$]{\includegraphics[width=0.47\textwidth]{../analysis/flow/{0.0001_90}.png}}
		\subfigure[$Re=120$]{\includegraphics[width=0.47\textwidth]{../analysis/flow/{0.0001_120}.png}}
	\end{minipage}
	\centering
	\begin{minipage}{\textwidth}
		\centering
		\subfigure[$Re=160$]{\includegraphics[width=0.47\textwidth]{../analysis/flow/{0.0001_160}.png}}
		\subfigure[$Re=200$]{\includegraphics[width=0.47\textwidth]{../analysis/flow/{0.0001_200}.png}}
	\end{minipage}
	\vskip 0.2em
	\wuhao 注:图中所示的流动区域为 $-1<x<15$,$-4<y<4$。
	\vspace{0.2em}
	\caption{$Da=0.0001$ 时近尾迹处不同雷诺数下涡量的等值线}
	\label{fig: vort-1e-4}
\end{figure}

\begin{figure}
	\setlength{\subfigcapskip}{-1bp}
	\centering
	\includegraphics[width=0.7\textwidth]{../analysis/flow/vort_legend}
	\begin{minipage}{\textwidth}
		\centering
		\subfigure[$Re=40$]{\includegraphics[width=0.47\textwidth]{../analysis/flow/{0.001_40}.png}}
		\subfigure[$Re=45$]{\includegraphics[width=0.47\textwidth]{../analysis/flow/{0.001_45}.png}}
	\end{minipage}
	\centering
	\begin{minipage}{\textwidth}
		\centering
		\subfigure[$Re=50$]{\includegraphics[width=0.47\textwidth]{../analysis/flow/{0.001_50}.png}}
		\subfigure[$Re=70$]{\includegraphics[width=0.47\textwidth]{../analysis/flow/{0.001_70}.png}}
	\end{minipage}
	\centering
	\begin{minipage}{\textwidth}
		\centering
		\subfigure[$Re=90$]{\includegraphics[width=0.47\textwidth]{../analysis/flow/{0.001_90}.png}}
		\subfigure[$Re=120$]{\includegraphics[width=0.47\textwidth]{../analysis/flow/{0.001_120}.png}}
	\end{minipage}
	\centering
	\begin{minipage}{\textwidth}
		\centering
		\subfigure[$Re=160$]{\includegraphics[width=0.47\textwidth]{../analysis/flow/{0.001_160}.png}}
		\subfigure[$Re=200$]{\includegraphics[width=0.47\textwidth]{../analysis/flow/{0.001_200}.png}}
	\end{minipage}
	\vskip 0.2em
	\wuhao 注:图中所示的流动区域为 $-1<x<15$,$-4<y<4$。
	\vspace{0.2em}
	\caption{$Da=0.001$ 时近尾迹处不同雷诺数下涡量的等值线图}
	\label{fig: vort-1e-3}
\end{figure}

\begin{figure}
	\setlength{\subfigcapskip}{-1bp}
	\centering
	\includegraphics[width=0.7\textwidth]{../analysis/flow/vort_legend}
	\begin{minipage}{\textwidth}
		\centering
		\subfigure[$Re=60$]{\includegraphics[width=0.47\textwidth]{../analysis/flow/{0.01_60}.png}}
		\subfigure[$Re=70$]{\includegraphics[width=0.47\textwidth]{../analysis/flow/{0.01_70}.png}}
	\end{minipage}
	\centering
	\begin{minipage}{\textwidth}
		\centering
		\subfigure[$Re=90$]{\includegraphics[width=0.47\textwidth]{../analysis/flow/{0.01_90}.png}}
		\subfigure[$Re=120$]{\includegraphics[width=0.47\textwidth]{../analysis/flow/{0.01_120}.png}}
	\end{minipage}
	\centering
	\begin{minipage}{\textwidth}
		\centering
		\subfigure[$Re=160$]{\includegraphics[width=0.47\textwidth]{../analysis/flow/{0.01_160}.png}}
		\subfigure[$Re=200$]{\includegraphics[width=0.47\textwidth]{../analysis/flow/{0.01_200}.png}}
	\end{minipage}
	\vskip 0.2em
	\wuhao 注:图中所示的流动区域为 $-1<x<15$,$-4<y<4$。
	\vspace{0.2em}
	\caption{$Da=0.01$ 时近尾迹处不同雷诺数下涡量的等值线图}
	\label{fig: vort-1e-2}
\end{figure}

%流线-近程
近程瞬时速度流线图如图~\ref{fig: streamfunc-1e-4} 所示。

\begin{figure}
	\setlength{\subfigcapskip}{-1bp}
	\centering
	\begin{minipage}{\textwidth}
		\centering
		\subfigure[$Re=40$]{\includegraphics[width=0.47\textwidth]{../analysis/flow/{0.0001_40_streamfn}.png}}
		\subfigure[$Re=45$]{\includegraphics[width=0.47\textwidth]{../analysis/flow/{0.0001_45_streamfn}.png}}
	\end{minipage}
	\centering
	\begin{minipage}{\textwidth}
		\centering
		\subfigure[$Re=50$]{\includegraphics[width=0.47\textwidth]{../analysis/flow/{0.0001_50_streamfn}.png}}
		\subfigure[$Re=90$]{\includegraphics[width=0.47\textwidth]{../analysis/flow/{0.0001_90_streamfn}.png}}
	\end{minipage}
	\centering
	\begin{minipage}{\textwidth}
		\centering
		\subfigure[$Re=140$]{\includegraphics[width=0.47\textwidth]{../analysis/flow/{0.0001_140_streamfn}.png}}
		\subfigure[$Re=200$]{\includegraphics[width=0.47\textwidth]{../analysis/flow/{0.0001_200_streamfn}.png}}
	\end{minipage}
	\vspace{0.2em}
	\caption{$Da=0.0001$ 时不同雷诺数下的流线图}
	\label{fig: streamfunc-1e-4}
\end{figure}

%一个周期内的流动变化,四个涡脱落图和四个流线图
对于非稳态流动,得到流动的周期后,可以继续观察流动在一个周期内的变化情况。以 $Da=1\times 10^{-4}$、$Re=100$ 为例,图~\ref{fig: 4*vortex} 显示了一个周期内的涡量等值线图。如图所示,四个时刻的流动是一致的且连续变化而来,$t=0$ 时刻圆柱背面下方有一个涡准备脱落,它的右侧有一个刚刚脱落的涡;四分之一周期过后,$t=1/4\,T$,下方的涡已经快要脱落;$t=2/4\,T$ 时刻的状态和 $t=0$ 时刻相反,上下对称,即圆柱背面的上方有一个准备脱落的涡;再过四分之一周期,$t=3/4\,T$,此时流动的状态又与 $t=1/4\,T$ 时刻对称。之后,随着时间的演进,这一周期内产生的涡从左向右移动,最终耗散在流体中,同时有新的涡不断产生,重复着这一周期内的现象。

\begin{figure}
	\setlength{\subfigcapskip}{-1bp}
	\centering
	\includegraphics[width=0.7\textwidth]{../analysis/flow/vort_legend_line}
	\begin{minipage}{\textwidth}
		\centering
		\subfigure[$t=0$]{\includegraphics[width=0.47\textwidth]{../analysis/period/{0.0001_100_T0}.png}}
		\subfigure[$t=1/4\,T$]{\includegraphics[width=0.47\textwidth]{../analysis/period/{0.0001_100_T1}.png}}
	\end{minipage}
	\centering
	\begin{minipage}{\textwidth}
		\centering
		\subfigure[$t=2/4\,T$]{\includegraphics[width=0.47\textwidth]{../analysis/period/{0.0001_100_T2}.png}}
		\subfigure[$t=3/4\,T$]{\includegraphics[width=0.47\textwidth]{../analysis/period/{0.0001_100_T3}.png}}
	\end{minipage}
	\vskip 0.2em
	\wuhao 注:图中所示的流动区域为 $-1<x<30$,$-5<y<5$。
	\vspace{0.2em}
	\caption{在 0,1/4,2/4,和 3/4 周期时涡的脱落图($Da=0.0001$,$Re=100$)}
	\label{fig: 4*vortex}
\end{figure}

\begin{figure}
	\setlength{\subfigcapskip}{-1bp}
	\centering
	\begin{minipage}{\textwidth}
		\centering
		\subfigure[$t=0$]{\includegraphics[width=0.47\textwidth]{../analysis/period/{0.0001_100_streamfnT0}.png}}
		\subfigure[$t=1/4\,T$]{\includegraphics[width=0.47\textwidth]{../analysis/period/{0.0001_100_streamfnT1}.png}}
	\end{minipage}
	\centering
	\begin{minipage}{\textwidth}
		\centering
		\subfigure[$t=2/4\,T$]{\includegraphics[width=0.47\textwidth]{../analysis/period/{0.0001_100_streamfnT2}.png}}
		\subfigure[$t=3/4\,T$]{\includegraphics[width=0.47\textwidth]{../analysis/period/{0.0001_100_streamfnT3}.png}}
	\end{minipage}
	\vskip 0.2em
	\vspace{0.2em}
	\caption{在 0,1/4,2/4,和 3/4 周期时涡的流线图($Da=0.0001$,$Re=100$)}
	\label{fig: 4*stream}
\end{figure}

\section{平均量的空间分布}\label{sec: average}

图~\ref{fig: average} 展示了 $Da=0.0001$、$Re=100$ 时平均流场的涡量等值线和流线图。

\begin{figure}
	\setlength{\subfigcapskip}{-1bp}
	\centering
	\begin{minipage}{\textwidth}
		\centering
		\subfigure[涡量等值线]{\includegraphics[width=0.47\textwidth]{../analysis/average/flow/{0.0001_100}.png}}
		\subfigure[流线]{\includegraphics[width=0.47\textwidth]{../analysis/average/flow/{0.0001_100_streamfn}.png}}
	\end{minipage}
	\vspace{0.2em}
	\caption{平均流场的涡量等值线和流线图($Da=0.0001$,$Re=100$)}
	\label{fig: average}
\end{figure}

%沿尾迹中心线的分布(平均速度,方均根速度,平均压力系数,方均根压力系数)
图~\ref{fig: average-center} 展示了平均速度沿尾迹中心线的分布,图中曲线是 $Da=0.0001$、$Re=100$ 下的情形。图~\ref{fig: average-center} a) 为平均流向速度 $\overline{u}$ 沿尾迹中心线的分布。$x=8.0$ 时平均速度达到局部极大值,在之后的 $8.0<x<30$ 范围内,平均速度逐渐减小。

图~\ref{fig: average-center} c) 为平均压力系数沿尾迹中心线的分布。在尾迹中,压力系数逐渐减小,直到达到最小值,之后开始逐渐增大,在大约 $x=3$ 之前增长十分迅速,随后增长放缓,慢慢恢复至来流值的大小,$x=30$ 时已经十分接近 $C_p=0$。

\begin{figure}
	\setlength{\subfigcapskip}{-1bp}
	\centering
	\begin{minipage}{\textwidth}
		\centering
		\subfigure[流向平均速度]{\includegraphics[width=0.47\textwidth]{../analysis/average/u_x}}
		\subfigure[流向平均速度在尾迹附近的放大图]{\includegraphics[width=0.47\textwidth]{../analysis/average/u_x_ma}}
	\end{minipage}
	\centering
	\begin{minipage}{\textwidth}
		\centering
		\subfigure[平均压力系数]{\includegraphics[width=0.47\textwidth]{../analysis/average/Cp_x}}
		\subfigure[平均压力系数在尾迹附近的放大图]{\includegraphics[width=0.47\textwidth]{../analysis/average/Cp_x_ma}}
	\end{minipage}
	\vspace{0.2em}
	\caption{时间平均量沿尾迹中心线的分布($Da=0.0001$,$Re=100$)}
	\label{fig: average-center}
\end{figure}

%沿圆柱表面的分布(平均压力系数,方均根压力系数)
图~\ref{fig: average-surface} 展示了平均速度沿尾迹中心线的分布。

\begin{figure}
	\setlength{\subfigcapskip}{-1bp}
	\centering
	\includegraphics[width=0.47\textwidth]{../analysis/average/Cp_theta}
	% \begin{minipage}{\textwidth}
	% 	\centering
	% 	\subfigure[压力系数]{\includegraphics[width=0.47\textwidth]{../analysis/average/Cp_theta}}
	% 	\subfigure[平均涡量]{\includegraphics[width=0.47\textwidth]{../analysis/average/Vort_theta}}
	% \end{minipage}
	\vspace{0.2em}
	\caption{时间平均量沿圆柱表面的分布($Da=0.0001$,$Re=100$)}
	\label{fig: average-surface}
\end{figure}

\section{本章小结}

本章在非稳态层流范围内选取了合适的例子进行计算,设置了合适的参数值,获得输出的数据文件,对数据进行初步处理,得到远程和进程的涡量云图和速度流线,经过平均化处理,得到平均涡量的等值线和平均速度的流线,继而得到回流区的几何结构和均值压力分布。
