% !Mode:: "TeX:UTF-8"
% !TEX root = ../main.tex
\chapter{结果与讨论}

\section{流动实例和流场图}

流动由雷诺数、达西数、孔隙率等多个参数决定,当前的研究中主要通过改变雷诺数和达西数得到不同的流动,并探究参数对流动的影响。非稳态流动的雷诺数范围大约为从 40 到 200, $Da=1\times 10^{-5}$ 时多孔区域接近固体,$Da=1\times 10^{-2}$ 时多孔内的流动已经非常微弱,所以当前计算时设置的雷诺数从区间 $[40,200]$ 内选择,达西数从区间 $[10^{-5},10^{-2}]$ 内选择,表~\ref{tab: DaRe} 列出了达西数和雷诺数可能的组合值,实际计算过的值用对勾($\surd$)标记。本文未改变孔隙率和应力阶跃参数的值,依承文献 \cite{} 的设定,将孔隙率设为 0.7,将应力阶跃的两个系数均设为零:$\varepsilon=0.7$,$\beta_1 = \beta_2=0$。

\begin{table}
	\caption{计算时用到的达西数和雷诺数的值的设置}\label{tab: DaRe}
	\vspace{.5em}\centering\wuhao
	\begin{tabular}{*{7}{c}}
		\toprule[1.5pt]
		\multirow{2}[3]{*}{$Re$} & \multicolumn{6}{c}{$Da$} \\
		\cmidrule[.67pt](lr){2-7}
		& 0.00001 & 0.0001 & 0.0005 & 0.001 & 0.005 & 0.01 \\
		\midrule[1pt]
		40  & ${\color{red}\surd}$ & ${\color{red}\surd}$ &         & ${\color{red}\surd}$ & $\surd$ \\
		41  & $\surd$              & ${\color{red}\surd}$ &         & ${\color{red}\surd}$ &   \\
		42  & ${\color{red}\surd}$ & ${\color{red}\surd}$ & $\surd$ & ${\color{red}\surd}$ &   \\
		43  & ${\color{red}\surd}$ & ${\color{red}\surd}$ & $\surd$ & ${\color{red}\surd}$ &   \\
		44  & ${\color{red}\surd}$ & ${\color{red}\surd}$ & $\surd$ & ${\color{red}\surd}$ &   \\
		45  & ${\color{red}\surd}$ & ${\color{red}\surd}$ & $\surd$ & ${\color{red}\surd}$ &   \\
		50  & ${\color{red}\surd}$ & ${\color{red}\surd}$ &         & ${\color{red}\surd}$ & $\surd$ \\
		60  & ${\color{red}\surd}$ & ${\color{red}\surd}$ & $\surd$ & ${\color{red}\surd}$ & $\surd$ \\
		70  & ${\color{red}\surd}$ & ${\color{red}\surd}$ & $\surd$ & ${\color{red}\surd}$ & $\surd$ \\
		80  & ${\color{red}\surd}$ & ${\color{red}\surd}$ & $\surd$ & ${\color{red}\surd}$ & $\surd$ \\
		90  & ${\color{red}\surd}$ & ${\color{red}\surd}$ & $\surd$ & ${\color{red}\surd}$ & $\surd$ \\
		100 & ${\color{red}\surd}$ & ${\color{red}\surd}$ & $\surd$ & ${\color{red}\surd}$ & $\surd$ \\
		120 & ${\color{red}\surd}$ & ${\color{red}\surd}$ & $\surd$ & ${\color{red}\surd}$ & $\surd$ \\
		140 & ${\color{red}\surd}$ & ${\color{red}\surd}$ & $\surd$ & ${\color{red}\surd}$ & $\surd$ \\
		160 & ${\color{red}\surd}$ & ${\color{red}\surd}$ & $\surd$ & ${\color{red}\surd}$ & $\surd$ \\
		180 & ${\color{red}\surd}$ & ${\color{red}\surd}$ & $\surd$ & ${\color{red}\surd}$ & $\surd$ \\
		200 & ${\color{red}\surd}$ & ${\color{red}\surd}$ & $\surd$ & ${\color{red}\surd}$ & $\surd$ \\
		\bottomrule[1.5pt]
	\end{tabular}
\end{table}

由结果可知,非稳态流动开始时所对应的雷诺数的值都在 40 之上,研究的第一步在于确定非稳态流动开始发生时所对应的临界雷诺数,从而确定此次研究的雷诺数范围。接着,本节还将展示非稳态流动的整体概貌,即不同参数下的流动特性,大体了解流动随雷诺数、达西数的变化情况。

\subsection{从稳态流动到非稳态流动}

在稳定流动状态(L2),流动在圆柱背面分离,形成了上下对称的一对漩涡,随着雷诺数的增加,圆柱背面尾迹的长度随之增加,分离点的角度则逐渐减小(从更靠前的位置开始分离)。这些量的变化情况可以参考文献 \cite{}。

固定达西数,当雷诺数较小时,流动处于稳态;随着雷诺数的增加,流动逐渐转变为非稳态,尾迹出现振荡,然后上下两个漩涡从圆柱表面交替脱落,各变量随时间做周期性变化。对于每个达西数,都存在一个临界雷诺数,该雷诺数标志着流动状态的转变。为了知晓某一达西数所对应的临界雷诺数,从已有数据计算出不同雷诺数下某一物理量随时间的变化图,通过观察物理量是否随时间波动,从而判断流动是否处于非稳态。由于计算具有一定误差,所以稳态时物理量也会随时间有微小的波动,实际当波动小于某一特定精度时就任务流动处于稳态,波动大于此精度则认为流动处于非稳态。下面将通过升力系数随时间的变化图($C_L$-$t$)来判断流动状态。

以 $Da=10^{-4}$ 为例,图~\ref{fig: Cl_t-1e-4} 显示了不同雷诺数下升力系数随时间的变化。当 $Re=40$ 时,升力系数随着时间逐渐减小,并最终减小到一个极小的值($<10^{-6}$),因此 $Re=40$ 时流动尚处于稳态。当 $Re=45$ 时,升力系数不会随时间减小,而是几乎稳定地波动,因此 $Re=45$ 时流动已经处于非稳态。由此可知,$Da=0.0001$ 所对应的临界雷诺数处于 40 和 45 之间。$Da=0.1$ 时整个雷诺数范围都处于稳态。

\begin{figure}
	\setlength{\subfigcapskip}{-1bp}
	\centering
	\begin{minipage}{\textwidth}
		\centering
		\subfigure[$Re=40$]{\includegraphics[width=0.48\textwidth]{../figs/0.0001_40/Cl_t}}
		\subfigure[$Re=45$]{\includegraphics[width=0.48\textwidth]{../figs/0.0001_45/Cl_t}}
	\end{minipage}
	\vspace{0.2em}
	\caption{不同雷诺数下升力系数随时间的变化($Da=0.0001$)}
	\label{fig: Cl_t-1e-4}
\end{figure}

\begin{figure}
	\setlength{\subfigcapskip}{-1bp}
	\centering
	\begin{minipage}{\textwidth}
		\centering
		\subfigure[$Re=40$]{\includegraphics[width=0.48\textwidth]{../figs/0.001_40/Cl_t}}
		\subfigure[$Re=45$]{\includegraphics[width=0.48\textwidth]{../figs/0.001_45/Cl_t}}
	\end{minipage}
	\vspace{0.2em}
	\caption{不同雷诺数下升力系数随时间的变化($Da=0.001$)}
	\label{fig: Cl_t-1e-3}
\end{figure}

同时,通过直接分析误差随时间的变化也可以判断流动状态。图~\ref{fig: resd}显示了 $Da=0.0001$ 时不同雷诺数下在计算过程中误差随时间步数的变化(包括 Resor0、Resor1、Resor2)。从图中可以看出,当 $Re=40$ 时,随着计算的进行,误差越来越小,并最终达到一个接近零的极小值,于是可以认为流动已经处于稳定状态;当 $Re=45$ 时,随着计算的进行,误差早早达到一个稳定波动的状态,但不会接近零,所以此时流动处于非稳态。由此可知临界雷诺数大于 40 而小于 45,与通过升力得到的结果一致。

\begin{figure}
	\centering
	\begin{minipage}{\textwidth}
		\centering
		\subfigure[$Re=40$]{\includegraphics[width=0.8\textwidth]{../figs/0.0001_40/resd}}
	\end{minipage}
	\centering
	\begin{minipage}{\textwidth}
		\centering
		\subfigure[$Re=45$]{\includegraphics[width=0.8\textwidth]{../figs/0.0001_45/resd}}
	\end{minipage}
	\caption{不同雷诺数下误差随时间的变化($Da=0.0001$)}
	\label{fig: resd}
\end{figure}

当流动处于稳态时,误差基本不再变化,当流动处于非稳态时,误差随时间呈周期性变化,计算出误差在一个周期内的平均值,作为该达西数和雷诺数下的平均误差,画出不同达西数下平均误差随雷诺数的变化,如图~\ref{fig: error}所示。从图中可以看出,不同达西数下误差的变化趋势基本一致。Resor2 都接近零,Resor0、Resor1都随着雷诺数的增大而增大,伴随着流动由稳态转变为非稳态,可以设定一个精度,当误差大于改值时流动状态发生改变,由此可以得到临界雷诺数。

\begin{figure}
	\centering
	\includegraphics[width=0.8\textwidth]{../analysis/meanError_Re}
	\caption{平均误差随 $Re$ 和 $Da$ 的变化}
	\label{fig: error}
\end{figure}

同理,对 $Da=10^{-5},5\times 10^{-5},5\times 10^{-4},10^{-3},5\times 10^{-3},10^{-2}$,也可以得到相应的临界雷诺数。最终得到的临界雷诺数见表~\ref{tab: critical Re}。从表中可以看到,达西数越大,临界雷诺数也越大,当达西数小于 $10^{-3}$ 时,临界雷诺数小于 45,当达西数达到 $10^{-2}$ 时,临界雷诺数已在 150 附近,当达西数等于 0.1 时,雷诺数 200 以下已经完全处于稳态,由于 雷诺数 200 以上具有一定的三维效应,所以不再计算。达西数为无量纲的渗透率,不同达西数下临界雷诺数具有不同的数值,这反映了达西数对流动的影响。当达西数很小时,多孔区几乎为固态,只有极少量的流体可以穿过圆柱,此时,临界雷诺数等于固体圆柱绕流的值。随着达西数的增大,多孔介质的影响体现出来,流体可以从圆柱内部穿过,一定程度上减弱了原来的不稳定。达西数很大时,多孔介质接近消失,相当于整个区域都是纯流体的流动,没有了圆柱内部的渗流以及多孔介质的阻碍,流动更接近稳态,需要更大的雷诺数才能由稳态转变为非稳态。

\begin{table}[h]
	\caption{不同达西数下的临界雷诺数}\label{tab: critical Re}
	\vspace{.5em}\centering\wuhao
	\begin{tabular}{cc}
		\toprule[1.5pt]
		$Da$ & 临界 $Re$ \\
		\midrule[1pt]
		$1\times 10^{-5}$ & 40--45 \\
		$1\times 10^{-4}$ & 40--45 \\
		$1\times 10^{-3}$ & 40--45 \\
		$5\times 10^{-3}$ & ? \\
		$1\times 10^{-2}$ & 140--160 \\
		$1\times 10^{-1}$ & >200 \\
	\bottomrule[1.5pt]
	\end{tabular}
\end{table}

\subsection{不同参数下的流场图}

根据测量数据,可以得到不同达西数和雷诺数下涡量的等值线图。图~\ref{fig: vorticity-contour-1e-4} 选取了六幅 $Da=0.0001$ 时不同雷诺数下流动已达平稳 \footnote{此处“平稳”一词指流动特性随时间做周期性变化,在时间平均意义下是稳定的(periodic but statistically stationary state),而“稳定”指流动特性不随时间发生任何变化,每一时刻的状态都是相同的(steady state)。} 时涡量的等值线图。从图中可以看出,$Re=40$ 时流动处于稳态,并在圆柱的后端产生了一对很长的漩涡,随着雷诺数的增大,尾迹的长度逐渐缩短;$Re=45$ 时尾迹已经开始波动,流动开始处于非稳态,与前文所述一致;$Re=50$ 时尾迹更短,非稳态已比较明显,涡脱落之后向下游移动,形成一列整齐的涡街;之后更是完全进入了非稳态,尾迹的长度也越来越短。

%还可以得到非稳态情形下一个周期内涡的脱落情形。

\begin{figure}
	\centering
	\begin{minipage}{\textwidth}
		\centering
		\subfigure[$Re=40$]{\includegraphics[width=0.47\textwidth]{../figs/0.0001_40/flow}}
		\subfigure[$Re=45$]{\includegraphics[width=0.47\textwidth]{../figs/0.0001_45/flow}}
	\end{minipage}
	\centering
	\begin{minipage}{\textwidth}
		\centering
		\subfigure[$Re=50$]{\includegraphics[width=0.47\textwidth]{../figs/0.0001_50/flow}}
		\subfigure[$Re=90$]{\includegraphics[width=0.47\textwidth]{../figs/0.0001_90/flow}}
	\end{minipage}
	\centering
	\begin{minipage}{\textwidth}
		\centering
		\subfigure[$Re=140$]{\includegraphics[width=0.47\textwidth]{../figs/0.0001_140/flow}}
		\subfigure[$Re=200$]{\includegraphics[width=0.47\textwidth]{../figs/0.0001_200/flow}}
	\end{minipage}
	\caption{$Da=1\times 10^{-4}$ 时不同雷诺数下涡量的等值线图}
	\label{fig: vorticity-contour-1e-4}
\end{figure}

图~\ref{fig: vorticity-contour-1e-3}展示了 $Da=0.001$ 时不同雷诺数下涡量的等值线图。总体的变化情况和 $Da=0.0001$ 时相同,除了在更大的雷诺数下达到非稳态。

\begin{figure}
	\centering
	\begin{minipage}{\textwidth}
		\centering
		\subfigure[$Re=40$]{\includegraphics[width=0.47\textwidth]{../figs/0.001_40/flow}}
		\subfigure[$Re=45$]{\includegraphics[width=0.47\textwidth]{../figs/0.001_45/flow}}
	\end{minipage}
	\centering
	\begin{minipage}{\textwidth}
		\centering
		\subfigure[$Re=50$]{\includegraphics[width=0.47\textwidth]{../figs/0.001_50/flow}}
		\subfigure[$Re=90$]{\includegraphics[width=0.47\textwidth]{../figs/0.001_90/flow}}
	\end{minipage}
	\centering
	\begin{minipage}{\textwidth}
		\centering
		\subfigure[$Re=140$]{\includegraphics[width=0.47\textwidth]{../figs/0.001_140/flow}}
		\subfigure[$Re=200$]{\includegraphics[width=0.47\textwidth]{../figs/0.001_200/flow}}
	\end{minipage}
	\caption{$Da=1\times 10^{-3}$ 时不同雷诺数下涡量的等值线图}
	\label{fig: vorticity-contour-1e-3}
\end{figure}

图~\ref{fig: vorticity-contour-5e-3}展示了 $Da=0.005$ 时不同雷诺数下涡量的等值线图。总体的变化情况和 $Da=0.0001$ 时相同,除了在更大的雷诺数下达到非稳态。(尚无)

\begin{figure}
	\centering
	\begin{minipage}{\textwidth}
		\centering
		\subfigure[$Re=40$]{\includegraphics[width=0.47\textwidth]{../figs/0.0001_40/flow}}
		\subfigure[$Re=45$]{\includegraphics[width=0.47\textwidth]{../figs/0.0001_45/flow}}
	\end{minipage}
	\centering
	\begin{minipage}{\textwidth}
		\centering
		\subfigure[$Re=50$]{\includegraphics[width=0.47\textwidth]{../figs/0.0001_50/flow}}
		\subfigure[$Re=90$]{\includegraphics[width=0.47\textwidth]{../figs/0.0001_90/flow}}
	\end{minipage}
	\centering
	\begin{minipage}{\textwidth}
		\centering
		\subfigure[$Re=140$]{\includegraphics[width=0.47\textwidth]{../figs/0.0001_140/flow}}
		\subfigure[$Re=200$]{\includegraphics[width=0.47\textwidth]{../figs/0.0001_200/flow}}
	\end{minipage}
	\caption{$Da=5\times 10^{-3}$ 时不同雷诺数下涡量的等值线图(尚无)}
	\label{fig: vorticity-contour-5e-3}
\end{figure}

图~\ref{fig: vorticity-contour-1e-2}展示了 $Da=0.01$ 时不同雷诺数下涡量的等值线图。从图中可以看出,直到 $Re=200$ 时尾迹才具有并不大的波动,雷诺数更小的时候流动基本处于稳态。

\begin{figure}
	\centering
	\begin{minipage}{\textwidth}
		\centering
		\subfigure[$Re=50$]{\includegraphics[width=0.47\textwidth]{../figs/0.01_50/flow}}
		\subfigure[$Re=90$]{\includegraphics[width=0.47\textwidth]{../figs/0.01_90/flow}}
	\end{minipage}
	\centering
	\begin{minipage}{\textwidth}
		\centering
		\subfigure[$Re=140$]{\includegraphics[width=0.47\textwidth]{../figs/0.01_140/flow}}
		\subfigure[$Re=200$]{\includegraphics[width=0.47\textwidth]{../figs/0.01_200/flow}}
	\end{minipage}
	\caption{$Da=0.01$ 时不同雷诺数下涡量的等值线图}
	\label{fig: vorticity-contour-1e-2}
\end{figure}

\section{Strouhal 数和阻力、升力系数}

非稳态层流流动中的几个重要变量是 Strouhal 数、阻力和升力系数,Strouhal 数体现了流体的非稳态流动所具有的频率,通过阻力和升力系数则可以知晓圆柱受到的阻力和升力,在工程中的一些多孔绕流问题中具有一定的意义。

表~\ref{tab: results-1e-5}展示了达西数 $10^{-4}$--$10^{-2}$、雷诺数 40--200 时二维流动的计算结果,其他参数的设置同 \ref{sec: grid-independent} 中的表~\ref{tab: parameters} 保持一致。

\begin{table}[h]
	\caption{不同雷诺数下的计算结果($Da=1\times 10^{-5}$)}\label{tab: results-1e-5}
	\vspace{.5em}\centering\wuhao
	\begin{tabular}{*{9}{c}}
		\toprule[1.5pt]
		$Re$ & $St$ & $C_D$ & $C_{Dp}$ & $C_{Df}$ & $C_{L'}$ & $C_L^A$ & $-C_{pb}$ & $\theta_s[^\circ]$ \\
		\midrule[1pt]
		 40 &  \\
		 41 &  \\
		 42 &  \\
		 43 &  \\
		 44 &  \\
		 45 & 0.1160 & 1.3671 & 0.9695 & 0.3977 & 0.0005  	 \\
		 50 & 0.1239 & 1.3495 & 0.9702 & 0.3793 & 0.0369 	 \\
		 60 & 0.1351 & 1.3150 & 0.9675 & 0.3474 & 0.0779 	 \\
		 70 & 0.1460 & 1.3041 & 0.9789 & 0.3252 & 0.1313 	 \\
		 80 & 0.1531 & 1.2894 &	0.9840 & 0.3054 & 0.1313  \\
		 90 & 0.1592 & 1.2772 &	0.9887 & 0.2885 & 0.1907 \\
		100 & 0.1650 & 1.2724 &	0.9976 & 0.2749 & 0.2232 \\
		120 & 0.1733 & 1.2646 &	1.0127 & 0.2520 & 0.2766 \\
		140 & 0.1805 & 1.2676 &	1.0328 & 0.2348 & 0.3318 \\
		160 & 0.1866 & 1.2739 &	1.0532 & 0.2207 & 0.3816 \\
		180 & 0.1916 & 1.2836 &	1.0745 & 0.2091 & 0.3318 \\
		200 & 0.1957 & 1.2938 &	1.0947 & 0.1991 & 0.4728 \\
		\bottomrule[1.5pt]
	\end{tabular}
\end{table}

\begin{table}[h]
	\caption{不同雷诺数下的计算结果($Da=0.0001$)}\label{tab: results-1e-4}
	\vspace{.5em}\centering\wuhao
	\begin{tabular}{*{9}{c}}
		\toprule[1.5pt]
		$Re$ & $St$ & $C_D$ & $C_{Dp}$ & $C_{Df}$ & $C_{L'}$ & $C_L^A$ & $-C_{pb}$ & $\theta_s[^\circ]$ \\
		\midrule[1pt]
		 40 &  \\
		\bottomrule[1.5pt]
	\end{tabular}
\end{table}

\begin{table}[h]
	\caption{不同雷诺数下的计算结果($Da=0.001$)}\label{tab: results-1e-3}
	\vspace{.5em}\centering\wuhao
	\begin{tabular}{*{9}{c}}
		\toprule[1.5pt]
		$Re$ & $St$ & $C_D$ & $C_{Dp}$ & $C_{Df}$ & $C_{L'}$ & $C_L^A$ & $-C_{pb}$ & $\theta_s[^\circ]$ \\
		\midrule[1pt]
		 40 &  \\
		\bottomrule[1.5pt]
	\end{tabular}
\end{table}

\subsection{Strouhal 数}

对于非稳态流动,流动波动的频率是一个重要参数,在工程中具有重要价值。而 Strouhal 数是表述流体波动的无量纲频率:
\begin{equation}\label{eq: St}
	St = \frac{Df}{U} = f = \frac{1}{T}
\end{equation}
\begin{tabularx}{\textwidth}{@{}l@{\quad}r@{——}X@{}}
	式中 & $St$ & Strouhal 数;\\
		& $D$ & 圆柱的直径,$=1$ m;\\
		& $f$ & 波动的频率;\\
		& $U$ & 远方的来流速度,$=1$ m/s;\\
		& $T$ & 波动的周期。 
\end{tabularx}\vspace{3.15bp}
频率与周期互为倒数,周期可以根据某一物理量(例如 Point 3 的水平速度)随时间的变化曲线而得到。继而根据方程 \eqref{eq: St} 可以得到 Strouhal 数。%得到的周期和 Strouhal 数记录在 \texttt{record.xlsx} 中。

图~\ref{fig: St} 显示了 Strouhal 数随着雷诺数和达西数的变化。从图中可以看出,$Da$ 固定时,$St$ 随 $Re$ 的增大而增大,且增长得越来越缓慢。在指定圆柱大小和流体种类的前提下,雷诺数由来流速度决定,来流速度越大,流体在圆柱背面形成的漩涡产生及脱落的频率也越大,符合直观的猜想。固定 $Re$ 时,$Da$ 越小,则 $St$ 越接近固体圆柱绕流的情形,随着达西数趋于零,得到的一系列曲线也将趋于某一条极限曲线,即为固体情形下的曲线。当 $Da=1 \times 10^{-5}$ 时,$St$ 曲线已经和固体圆柱绕流的变化曲线相吻合,\ref{sec: validation}中的图~\ref{fig: validation-St}已作出了验证。

\begin{figure}
	\centering
	\includegraphics[width=0.8\textwidth]{../analysis/St_Re}
	\caption{Strouhal 数随 $Re$ 的变化}
	\label{fig: St}
\end{figure}

对于非稳态流动,得到流动的周期后,可以继续观察流动在一个周期内的变化情况。以 $Da=1\times 10^{-4}$、$Re=100$ 为例,图~\ref{fig: 4*vortex}显示了一个周期内的涡量等值线图。如图所示,四个时刻的流动是一致的且连续变化而来,$t=0$ 时刻圆柱背面下方有一个涡准备脱落,它的右侧有一个刚刚脱落的涡;四分之一周期过后,$t=1/4\,T$,下方的涡已经快要脱落;$t=2/4\,T$ 时刻的状态和 $t=0$ 时刻相反,上下对称,即圆柱背面的上方有一个准备脱落的涡;再过四分之一周期,$t=3/4\,T$,此时流动的状态又与 $t=1/4\,T$ 时刻对称。之后,随着时间的演进,这一周期内产生的涡从左向右移动,最终耗散在流体中,同时有新的涡不断产生,重复着这一周期内的现象。

\begin{figure}
	\centering
	\begin{minipage}{\textwidth}
		\centering
		\subfigure[$t=0$]{\includegraphics[width=0.47\textwidth]{../figs/0.0001_100/T0}}
		\subfigure[$t=1/4\,T$]{\includegraphics[width=0.47\textwidth]{../figs/0.0001_100/T1}}
	\end{minipage}
	\centering
	\begin{minipage}{\textwidth}
		\centering
		\subfigure[$t=2/4\,T$]{\includegraphics[width=0.47\textwidth]{../figs/0.0001_100/T2}}
		\subfigure[$t=3/4\,T$]{\includegraphics[width=0.47\textwidth]{../figs/0.0001_100/T3}}
	\end{minipage}
	\caption{在 0,1/4,2/4,和 3/4 周期时涡的脱落图($Da=1\times 10^{-4}$,$Re=100$)}
	\label{fig: 4*vortex}
\end{figure}

\subsection{阻力和升力系数}

图~\ref{fig: ClCd_t-1e-4}显示了非稳态层流中在一定的达西数和雷诺数下阻力和升力系数随时间的变化。对于每一个雷诺数和达西数,经过开始的一段时间之后,流动逐渐进入随时间作周期性变化的平稳状态。设定的计算时间普遍为 300 秒。当达西数较小时,进入平稳状态所需的时间也更短有时无法在 300 秒内达到平稳状态。例如,$Da=1\times 10^{-5}$ 时,$40<Re<60$ 在 300 秒内波动的振幅还在变化,未达到平稳状态;$Re=60$ 时可以在 150 秒时稳定下来,此后所需的时间逐渐减少,$Re=100$ 时所需时间为 70 秒。$Da=1\times 10^{-4}$ 时,从 $Re=50$ 开始可以在 300 秒内达到稳定。总体趋势是达到平稳时所需的时间按达西数的增大而减小。

\begin{figure}
	\setlength{\subfigcapskip}{-1bp}
	\centering
	\begin{minipage}{\textwidth}
		\centering
		\subfigure[$Re=40$]{\includegraphics[width=0.47\textwidth]{../figs/0.0001_40/Cl_t}\includegraphics[width=0.47\textwidth]{../figs/0.0001_40/Cl_t}}
	\end{minipage}
	\centering
	\begin{minipage}{\textwidth}
		\centering
		\subfigure[$Re=45$]{\includegraphics[width=0.47\textwidth]{../figs/0.0001_45/Cl_t}\includegraphics[width=0.47\textwidth]{../figs/0.0001_45/Cl_t}}
	\end{minipage}
	\centering
	\begin{minipage}{\textwidth}
		\centering
		\subfigure[$Re=45$]{\includegraphics[width=0.48\textwidth]{../figs/0.0001_45/Cl_t}\includegraphics[width=0.48\textwidth]{../figs/0.0001_45/Cl_t}}
	\end{minipage}
	\centering
	\begin{minipage}{\textwidth}
		\centering
		\subfigure[$Re=45$]{\includegraphics[width=0.48\textwidth]{../figs/0.0001_45/Cl_t}\includegraphics[width=0.48\textwidth]{../figs/0.0001_45/Cl_t}}
	\end{minipage}
	\vspace{0.2em}
	\caption{升力系数和阻力系数随时间的变化($Da=0.0001$)}
	\label{fig: ClCd_t-1e-4}
\end{figure}

非稳态情形下阻力和升力都随时间做周期性变化,所以此处取一个周期内的平均阻力和升力。平均阻力系数随雷诺数的变化如图~\ref{fig: meanCd} 所示。当 $Da=1\times 10^{-5}$ 时,阻力系数先随雷诺数的增大而减小,在大约 $Re=120$ 时达到最小值,之后 $Re=120$--$200$ 的范围内,阻力系数随着雷诺数的增加而缓慢地增大,比之前减小时的速率小了很多。当 $Da=0.0001$ 时,从 $Re=45$ 开始,阻力系数下降得较快,与 $Da=1\times 10^{-4}$ 的下降速率相当,当 $Re=90$ 时,阻力系数达到了极小值,此时阻力系数的值比 $Re=45$ 降低了 4.74\%,随后在 $Re=90$--$200$ 的范围内,阻力系数不断增大,增速比 $Da=1\times 10^{-5}$ 的情形块许多,并且最终超过了 $Re=45$ 时的值。当 $Da=0.001$ 时,阻力系数的变化有所不同。在 $Re=60$ 之前,阻力系数随着雷诺数的增大而减小,在 $Re=60$ 之后,阻力系数突然开始增大,并于 $Re=140$ 时达到了最大值,相比 $Re=60$ 时增大了 3.23\%,随后才逐渐下降。所以,在该达西数下,阻力再 $Re=60$--$200$ 区间内的变化趋势和其他情形完全相反,可能是多孔介质渗流增大所导致的(?)。

观察同一雷诺数下不同达西数对应的阻力系数,可以得知多孔介质的存在对流体流动的影响。在 $Re=45$--$200$ 范围内,固定雷诺数,达西数越大则阻力系数越大。$Re=45$ 时,$Da=1\times 10^{-5},\,0.0001,\,0.001$ 的阻力系数之比为 $1:1.013:1.034$,此时雷诺数很小,多孔介质的影响并不明显。随着雷诺数的增大,这一差距也逐渐拉大,$Re=120$ 时,三者之比达到了 $1:1.05:1.15$,$Re=200$ 时的比例为 $1:1.10:1.11$,由于 $Da=0.0001$ 和 0.001 变化趋势相反,所以二者逐渐接近。

\begin{figure}
	\centering
	\includegraphics[width=0.8\textwidth]{../analysis/meanCd_Re}
	\caption{平均阻力系数 $C_D$ 随 $Re$ 的变化}
	\label{fig: meanCd}
\end{figure}

当流体绕过钝体流动时,流体的粘性作用形成了摩擦阻力,物体前后的压强差形成了压差阻力。摩擦阻力为作用在物体表面的切向力在来流方向的分量之和,而压差阻力是作用在物体表面的法向力在来流方向的分量之和,二者共同形成了物体所受的阻力。用符号表示如下(不再写平均符号):
\begin{equation}
	C_D = C_{Dp} + C_{Df}
\end{equation}
图~\ref{fig: meanCdpf} 反映了不同达西数下压差阻力和摩擦阻力随雷诺数的变化。当流体在物体背面形成尾迹时,流体的能量不断耗散在尾迹的漩涡中,使得物体背面的压强较低,物体前后产生压强差,形成了压差阻力。尾迹区域越大则耗散越强,压差阻力也就越大。对于流线型物体,流体几乎顺着物体的表面流动,在物体末端才出现分离现象,所以它的阻力主要来源于摩擦阻力。钝体受到的压差阻力则相对较大,流体分离点的位置越靠前,尾迹区越大,压差阻力也越大。对于圆柱绕流,从图中可以看出,在 $Re=45$--$200$ 区间内,压差阻力显著大于摩擦阻力,说明阻力主要由流动的尾迹造成。随着雷诺数的增大,$Da=1\times 10^{-5}$ 下压差阻力占总阻力的比例从 $Re=45$ 时的 71\% 增加到了 $Re=200$ 时的 85\%;$Da=0.0001$ 下从 $Re=45$ 时的 71\% 增加到了 $Re=200$ 时的 87\%;$Da=0.001$ 下更是从 $Re=45$ 时的 76\% 增加到了 $Re=200$ 时的 92\%。随着雷诺数的增大,分离点向上游移动(见图~),尾迹区扩大,压差阻力增大,与图~\ref{fig: meanCdpf} 中 $C_{Dp}$ 的变化一致。摩擦阻力则随着雷诺数的增大而减小(?)。

对于不同的达西数,多孔圆柱受到的阻力不尽相同。同一雷诺数下,达西数越大,则压差阻力也越大(?)。同一雷诺数下,达西数越大,摩擦阻力越小(?)。

\begin{figure}
	\centering
	\includegraphics[width=0.8\textwidth]{../analysis/meanCdpf_Re}
	\caption{压差阻力系数 $C_{Dp}$ 和摩擦阻力系数 $C_{Df}$ 随 $Re$ 的变化}
	\label{fig: meanCdpf}
\end{figure}

方均根升力系数 $C_{L'}$ 随达西数和雷诺数的变化如图~\ref{fig: rmsCl} 所示。$C_{L'}$ 反映了涡脱落的剧烈程度,由图可知,$C_{L'}$ 随雷诺数的增加而增加,雷诺数增大时涡脱落也变得更加剧烈。\cite{} 得出,当 $Re=50,\,60$ 时 $C_{L'}$ 的大小正比于 $\sqrt{Re}$。在整个雷诺数范围内,升力系数的波动主要由壁面上的压力引起,壁面摩擦力导致的波动只占一小部分。根据计算结果,$Da=1\times 10^{-5}$ 时,比值 $C_{L'p}/C_{L'}$ 从 $Re=45$ 时的 80\% 增加到了 $Re=200$ 时的 93\%;$Da=0.0001$ 时从 $Re=45$ 的 87\% 增加到了 $Re=200$ 时的 94\%;$Da=0.001$ 时从 $Re=45$ 的 87\% 增加到了 $Re=200$ 时的 96\%。与之相反,比值 $C_{L'f}/C_{L'}$ 随着雷诺数下降,并和雷诺数的平方根成反比。Park 等人 \cite{} 判断,在由压强和摩擦引起的升力波动之间存在一个相位的转变。随时间变化的三个升力系数具有如下关系:
\begin{equation}\label{eq: C_L}
	C_L = C_{Lp} + C_{Lf}
\end{equation}
$C_{L'}$ 为 $C_L$ 的方均根:
\begin{equation}\label{eq: C_L'}
	C_{L'} = \sqrt{\frac{1}{T}\int_{t_0}^{t_0+T}C_L^2\diff t} \approx
	\sqrt{\frac{1}{N}\sum_{n=1}^N C_{L,n}^2}
\end{equation}
由式 \eqref{eq: C_L} 和 \eqref{eq: C_L'} 可得
\begin{equation}\label{eq: C_L'-C_L}
	C_{L'}^2 = C_{L'p}^2 + C_{L'f}^2 + 2\cdot\frac{1}{T}\int_{t_0}^{t_0+T}C_{Lp}C_{Lf}\diff t
\end{equation}
如果升力的波动可以看作正弦曲线,并且压强和摩擦之间的相位差是一个常数 $\phi$ 的话,式 \eqref{eq: C_L'-C_L} 可进一步写为三个方均根升力系数之间的关系
\begin{equation}
	C_{L'}^2 = C_{L'p}^2 + C_{L'f}^2 + 2C_{L'p}C_{L'f}\cos\phi
\end{equation}
从数据可以知晓,$Re=45$--$200$ 范围内计算得出的 $\phi$ 的确几乎是常数,大约为 $30^\circ$。

\begin{figure}
	\centering
	\includegraphics[width=0.8\textwidth]{../analysis/rmsCl_Re}
	\caption{方均根升力系数 $C_{L'}$、$C_{L'p}$ 和 $C_{L'f}$ 随 $Re$ 的变化}
	\label{fig: rmsCl}
\end{figure}

升力系数波动的振幅随雷诺数的变化如图~\ref{fig: ClAmplitude} 所示,与文献结果一致 \cite{Park1998}。

\begin{figure}
	\centering
	\includegraphics[width=0.8\textwidth]{../analysis/ClAmplitude_Re}
	\caption{升力系数波动的振幅 $C_L^A$ 随 $Re$ 的变化}
	\label{fig: ClAmplitude}
\end{figure}

\section{能谱分析} % FFT

对随时间变化的物理量做傅里叶分析,可以得到离散频率的频谱。已知某个样本点的速度、圆柱的阻力和升力系数随时间变化的数据,对这些物理量做快速傅里叶变换,可以得到这些量在频域中的分布。频域的横坐标时频率,也是 Strouhal 数,某些频率上有峰值出现,表示此处能量密度很高,能量集中在这个频率附近。傅里叶变换之后可能由多个频率分量,类似谐波(?),形成若干离散的峰值,这些峰值按一定的间距分布,大小不一。

水平速度随时间的变化以及相应的 Fourier 变换如图~\ref{fig: U Fourier-1e-4} 和~\ref{fig: U Fourier-1e-3} 所示,竖直速度随时间的变化以及相应的 Fourier 变换如图~\ref{fig: V Fourier-1e-4} 和~\ref{fig: V Fourier-1e-3} 所示,升力系数随时间的变化以及相应的 Fourier 变换如图~\ref{fig: Cl Fourier-1e-4} 和~\ref{fig: Cl Fourier-1e-3} 所示。图中画出了稳定时的 10 个周期。

当 $Da=1\times 10^{-5}$ 时,第三个样本点处的水平速度 $U$ 随时间周期性波动,从图中所示的十个周期可以看到,每个周期内速度都具有一大一小两个小的周期,代表了速度的两个主要频率成分,对应于右图中第二个峰值和第三个峰值($f=0$ 处的第一个峰值反映的是 $U$ 的时间平均值,即左图中的水平线)。右图第三个峰值之后还有一系列的更小的峰值,由于值太小而无法表现出来,说明速度不仅具有左图中可见的两个频率,还具有无数的频率分量,只是所占比例很小,可以忽略。右图中第二个频率即为该达西数和雷诺数下流体运动的 Strouhal 数,与前文结果一致。从 $Re=50$ 到 $Re=200$,随着雷诺数的增加,$f=0$ 处的峰值增大,说明这一点的水平速度增大;主频率向右移动,即 Strouhal 数逐渐增大,与前文所述一致。

\begin{figure}
	\centering
	\begin{minipage}{\textwidth}
		\centering
		\subfigure[$Re=50$]{\includegraphics[width=0.6\textwidth]{../figs/0.0001_50/psd_U}}
	\end{minipage}
	\centering
	\begin{minipage}{\textwidth}
		\centering
		\subfigure[$Re=90$]{\includegraphics[width=0.6\textwidth]{../figs/0.0001_90/psd_U}}
	\end{minipage}
	\centering
	\begin{minipage}{\textwidth}
		\centering
		\subfigure[$Re=140$]{\includegraphics[width=0.6\textwidth]{../figs/0.0001_140/psd_U}}
	\end{minipage}
	\centering
	\begin{minipage}{\textwidth}
		\centering
		\subfigure[$Re=200$]{\includegraphics[width=0.6\textwidth]{../figs/0.0001_200/psd_U}}
	\end{minipage}
	\caption{$Da=0.0001$ 时不同雷诺数下水平速度 $U$ 的 Fourier 变换}
	\label{fig: U Fourier-1e-4}
\end{figure}

\begin{figure}
	\centering
	\begin{minipage}{\textwidth}
		\centering
		\subfigure[$Re=50$]{\includegraphics[width=0.6\textwidth]{../figs/0.001_50/psd_U}}
	\end{minipage}
	\centering
	\begin{minipage}{\textwidth}
		\centering
		\subfigure[$Re=90$]{\includegraphics[width=0.6\textwidth]{../figs/0.001_90/psd_U}}
	\end{minipage}
	\centering
	\begin{minipage}{\textwidth}
		\centering
		\subfigure[$Re=140$]{\includegraphics[width=0.6\textwidth]{../figs/0.001_140/psd_U}}
	\end{minipage}
	\centering
	\begin{minipage}{\textwidth}
		\centering
		\subfigure[$Re=200$]{\includegraphics[width=0.6\textwidth]{../figs/0.001_200/psd_U}}
	\end{minipage}
	\caption{$Da=0.001$ 时不同雷诺数下水平速度 $U$ 的 Fourier 变换}
	\label{fig: U Fourier-1e-3}
\end{figure}

\begin{figure}
	\centering
	\begin{minipage}{\textwidth}
		\centering
		\subfigure[$Re=50$]{\includegraphics[width=0.6\textwidth]{../figs/0.0001_50/psd_V}}
	\end{minipage}
	\centering
	\begin{minipage}{\textwidth}
		\centering
		\subfigure[$Re=90$]{\includegraphics[width=0.6\textwidth]{../figs/0.0001_90/psd_V}}
	\end{minipage}
	\centering
	\begin{minipage}{\textwidth}
		\centering
		\subfigure[$Re=140$]{\includegraphics[width=0.6\textwidth]{../figs/0.0001_140/psd_V}}
	\end{minipage}
	\centering
	\begin{minipage}{\textwidth}
		\centering
		\subfigure[$Re=200$]{\includegraphics[width=0.6\textwidth]{../figs/0.0001_200/psd_V}}
	\end{minipage}
	\caption{$Da=0.0001$ 时不同雷诺数下竖直速度 $V$ 的 Fourier 变换}
	\label{fig: V Fourier-1e-4}
\end{figure}

\begin{figure}
	\centering
	\begin{minipage}{\textwidth}
		\centering
		\subfigure[$Re=50$]{\includegraphics[width=0.6\textwidth]{../figs/0.001_50/psd_V}}
	\end{minipage}
	\centering
	\begin{minipage}{\textwidth}
		\centering
		\subfigure[$Re=90$]{\includegraphics[width=0.6\textwidth]{../figs/0.001_90/psd_V}}
	\end{minipage}
	\centering
	\begin{minipage}{\textwidth}
		\centering
		\subfigure[$Re=140$]{\includegraphics[width=0.6\textwidth]{../figs/0.001_140/psd_V}}
	\end{minipage}
	\centering
	\begin{minipage}{\textwidth}
		\centering
		\subfigure[$Re=200$]{\includegraphics[width=0.6\textwidth]{../figs/0.001_200/psd_V}}
	\end{minipage}
	\caption{$Da=0.001$ 时不同雷诺数下竖直速度 $V$ 的 Fourier 变换}
	\label{fig: V Fourier-1e-3}
\end{figure}

\begin{figure}
	\centering
	\begin{minipage}{\textwidth}
		\centering
		\subfigure[$Re=50$]{\includegraphics[width=0.6\textwidth]{../figs/0.0001_50/psd_Cl}}
	\end{minipage}
	\centering
	\begin{minipage}{\textwidth}
		\centering
		\subfigure[$Re=90$]{\includegraphics[width=0.6\textwidth]{../figs/0.0001_90/psd_Cl}}
	\end{minipage}
	\centering
	\begin{minipage}{\textwidth}
		\centering
		\subfigure[$Re=140$]{\includegraphics[width=0.6\textwidth]{../figs/0.0001_140/psd_Cl}}
	\end{minipage}
	\centering
	\begin{minipage}{\textwidth}
		\centering
		\subfigure[$Re=200$]{\includegraphics[width=0.6\textwidth]{../figs/0.0001_200/psd_Cl}}
	\end{minipage}
	\caption{$Da=0.0001$ 时不同雷诺数下升力系数的 Fourier 变换}
	\label{fig: Cl Fourier-1e-4}
\end{figure}

\begin{figure}
	\centering
	\begin{minipage}{\textwidth}
		\centering
		\subfigure[$Re=50$]{\includegraphics[width=0.6\textwidth]{../figs/0.001_50/psd_Cl}}
	\end{minipage}
	\centering
	\begin{minipage}{\textwidth}
		\centering
		\subfigure[$Re=90$]{\includegraphics[width=0.6\textwidth]{../figs/0.001_90/psd_Cl}}
	\end{minipage}
	\centering
	\begin{minipage}{\textwidth}
		\centering
		\subfigure[$Re=140$]{\includegraphics[width=0.6\textwidth]{../figs/0.001_140/psd_Cl}}
	\end{minipage}
	\centering
	\begin{minipage}{\textwidth}
		\centering
		\subfigure[$Re=200$]{\includegraphics[width=0.6\textwidth]{../figs/0.001_200/psd_Cl}}
	\end{minipage}
	\caption{$Da=0.001$ 时不同雷诺数下升力系数的 Fourier 变换}
	\label{fig: Cl Fourier-1e-3}
\end{figure}

\section{多孔介质对流动状态的影响——与固体圆柱绕流相比}

\section{流动状态从稳态向非稳态的转变——与稳态圆柱绕流相比}

\section{本章小结}
