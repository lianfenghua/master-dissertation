% !Mode:: "TeX:UTF-8"
% !TEX root = ../main.tex
\chapter{结果与讨论}

\section{流动实例和流场图}

流动由雷诺数、达西数、孔隙率等多个参数决定,当前的研究中主要通过改变雷诺数和达西数得到不同的流动,并探究参数对流动的影响。非稳态流动的雷诺数范围大约为从 40 到 200, $Da=1\times 10^{-5}$ 时多孔区域接近固体,$Da=1\times 10^{-2}$ 时多孔内的流动已经非常微弱,所以当前计算时设置的雷诺数从区间 $[40,200]$ 内选择,达西数从区间 $[10^{-5},10^{-2}]$ 内选择,表~\ref{tab: DaRe} 列出了达西数和雷诺数可能的组合值,实际计算过的值用对勾($\surd$)标记。本文未改变孔隙率和应力阶跃参数的值,依承文献 \cite{} 的设定,将孔隙率设为 0.7,将应力阶跃的两个系数均设为零:$\varepsilon=0.7$,$\beta_1 = \beta_2=0$。

\begin{table}
	\caption{计算时用到的达西数和雷诺数的值的设置}\label{tab: DaRe}
	\vspace{.5em}\centering\wuhao
	\begin{tabular}{*{6}{c}}
		\toprule[1.5pt]
		\multirow{2}[3]{*}{$Re$} & \multicolumn{5}{c}{$Da$} \\
		\cmidrule[.67pt](lr){2-6}
		& 0.00001 & 0.0001 & 0.0005 & 0.001 & 0.005 \\
		\midrule[1pt]
		40 & ${\color{red}\surd}$ & $\surd$ &   & $\surd$ & $\surd$ \\
		41 & $\surd$ & $\surd$ &   & $\surd$ &   \\
		42 & ${\color{red}\surd}$ &   & $\surd$ & $\surd$ &   \\
		43 & ${\color{red}\surd}$ &   & $\surd$ & $\surd$ &   \\
		44 & ${\color{red}\surd}$ &   & $\surd$ & $\surd$ &   \\
		45 & ${\color{red}\surd}$ &   & $\surd$ & $\surd$ &   \\
		46 & $\surd$ &         &   &         &   \\
		47 & $\surd$ &         &   &         &   \\
		48 & $\surd$ &         &   &         &   \\
		49 & $\surd$ &         &   &         &   \\
		50 & ${\color{red}\surd}$ & $\surd$ &   & $\surd$ & $\surd$ \\
		60 & ${\color{red}\surd}$ & ${\color{red}\surd}$ & ${\color{red}\surd}$ & $\surd$ & $\surd$ \\
		70 & ${\color{red}\surd}$ & ${\color{red}\surd}$ & ${\color{red}\surd}$ & $\surd$ & $\surd$ \\
		80 & ${\color{red}\surd}$ & ${\color{red}\surd}$ & ${\color{red}\surd}$ & $\surd$ & $\surd$ \\
		90 & ${\color{red}\surd}$ & ${\color{red}\surd}$ & ${\color{red}\surd}$ & $\surd$ & $\surd$ \\
		100 & ${\color{red}\surd}$ & ${\color{red}\surd}$ & ${\color{red}\surd}$ & $\surd$ & $\surd$ \\
		120 & ${\color{red}\surd}$ & ${\color{red}\surd}$ & ${\color{red}\surd}$ & $\surd$ & $\surd$ \\
		140 & ${\color{red}\surd}$ & ${\color{red}\surd}$ & ${\color{red}\surd}$ & $\surd$ & $\surd$ \\
		160 & ${\color{red}\surd}$ & ${\color{red}\surd}$ & ${\color{red}\surd}$ & $\surd$ & $\surd$ \\
		180 & ${\color{red}\surd}$ & ${\color{red}\surd}$ & ${\color{red}\surd}$ & $\surd$ & $\surd$ \\
		200 & ${\color{red}\surd}$ & ${\color{red}\surd}$ & ${\color{red}\surd}$ & $\surd$ & $\surd$ \\
		\bottomrule[1.5pt]
	\end{tabular}
\end{table}

由结果可知,非稳态流动开始时所对应的雷诺数的值都在 40 之上,研究的第一步在于确定非稳态流动开始发生时所对应的临界雷诺数,从而确定此次研究的雷诺数范围。接着,本节还将展示非稳态流动的整体概貌,即不同参数下的流动特性,大体了解流动随雷诺数、达西数的变化情况。

\subsection{从稳态流动到非稳态流动}

在稳定流动状态(L2),流动在圆柱背面分离,形成了上下对称的一对漩涡,随着雷诺数的增加,圆柱背面尾迹的长度随之增加,分离点的角度则逐渐减小(从更靠前的位置开始分离)。这些量的变化情况可以参考文献 \cite{}。

固定达西数,当雷诺数较小时,流动处于稳态;随着雷诺数的增加,流动逐渐转变为非稳态,尾迹出现振荡,然后上下两个漩涡从圆柱表面交替脱落,各变量随时间做周期性变化。对于每个达西数,都存在一个临界雷诺数,该雷诺数标志着流动状态的转变。为了知晓某一达西数所对应的临界雷诺数,从已有数据计算出不同雷诺数下某一物理量随时间的变化图,通过观察物理量是否随时间波动,从而判断流动是否处于非稳态。由于计算具有一定误差,所以稳态时物理量也会随时间有微小的波动,实际当波动小于某一特定精度时就任务流动处于稳态,波动大于此精度则认为流动处于非稳态。下面将通过升力系数随时间的变化图($C_{\mathrm{l}}$-$t$)来判断流动状态。

以 $Da=10^{-4}$ 为例,图~\ref{fig: Cl_t} 显示了不同雷诺数下升力系数随时间的变化。当 $Re=40$ 时,升力系数随着时间逐渐减小,并最终减小到一个极小的值($<10^{-6}$),因此 $Re=40$ 时流动尚处于稳态。当 $Re=45$ 时,升力系数不会随时间减小,而是几乎稳定地波动,因此 $Re=45$ 时流动已经处于非稳态。由此可知,$Da=0.0001$ 所对应的临界雷诺数处于 40 和 45 之间。

\begin{figure}
	\setlength{\subfigcapskip}{-1bp}
	\centering
	\begin{minipage}{\textwidth}
		\centering
		\subfigure[$Re=40$]{\includegraphics[width=0.8\textwidth]{../figs/0.0001_40/Cl_t}}
	\end{minipage}
	\centering
	\begin{minipage}{\textwidth}
		\centering
		\subfigure[$Re=45$]{\includegraphics[width=0.8\textwidth]{../figs/0.0001_45/Cl_t}}
	\end{minipage}
	\vspace{0.2em}
	\caption{不同雷诺数下升力系数随时间的变化($Da=0.0001$)}
	\label{fig: Cl_t}
\end{figure}

同时,通过直接分析误差随时间的变化也可以判断流动状态。图~\ref{fig: resd}显示了 $Da=0.0001$ 时不同雷诺数下在计算过程中误差随时间步数的变化(包括 Resor0、Resor1、Resor2)。从图中可以看出,当 $Re=40$ 时,随着计算的进行,误差越来越小,并最终达到一个接近零的极小值,于是可以认为流动已经处于稳定状态;当 $Re=45$ 时,随着计算的进行,误差早早达到一个稳定波动的状态,但不会接近零,所以此时流动处于非稳态。由此可知临界雷诺数大于 40 而小于 45,与通过升力得到的结果一致。

\begin{figure}
	\centering
	\begin{minipage}{\textwidth}
		\centering
		\subfigure[$Re=40$]{\includegraphics[width=0.8\textwidth]{../figs/0.0001_40/resd}}
	\end{minipage}
	\centering
	\begin{minipage}{\textwidth}
		\centering
		\subfigure[$Re=45$]{\includegraphics[width=0.8\textwidth]{../figs/0.0001_45/resd}}
	\end{minipage}
	\caption{不同雷诺数下误差随时间的变化($Da=0.0001$)}
	\label{fig: resd}
\end{figure}

当流动处于稳态时,误差基本不再变化,当流动处于非稳态时,误差随时间呈周期性变化,计算出误差在一个周期内的平均值,作为该达西数和雷诺数下的平均误差,画出不同达西数下平均误差随雷诺数的变化,如图~\ref{fig: error}所示。从图中可以看出,不同达西数下误差的变化趋势基本一致。Resor2 都接近零,Resor0、Resor1都随着雷诺数的增大而增大,伴随着流动由稳态转变为非稳态,可以设定一个精度,当误差大于改值时流动状态发生改变,由此可以得到临界雷诺数。

\begin{figure}
	\centering
	\includegraphics[width=0.8\textwidth]{../analysis/meanError_Re}
	\caption{平均误差随 $Re$ 和 $Da$ 的变化}
	\label{fig: error}
\end{figure}

同理,对 $Da=10^{-5},5\times 10^{-5},5\times 10^{-4},10^{-3},5\times 10^{-3},10^{-2}$,也可以得到相应的临界雷诺数。最终得到的临界雷诺数见表~\ref{tab: critical Re}。从表中可以看到,达西数越大,临界雷诺数也越大,当达西数小于 $10^{-3}$ 时,临界雷诺数小于 45,当达西数达到 $10^{-2}$ 时,临界雷诺数已在 150 附近,当达西数等于 0.1 时,雷诺数 200 以下已经完全处于稳态,由于 雷诺数 200 以上具有一定的三维效应,所以不再计算。达西数为无量纲的渗透率,不同达西数下临界雷诺数具有不同的数值,这反映了达西数对流动的影响。当达西数很小时,多孔区几乎为固态,只有极少量的流体可以穿过圆柱,此时,临界雷诺数等于固体圆柱绕流的值。随着达西数的增大,多孔介质的影响体现出来,流体可以从圆柱内部穿过,一定程度上减弱了原来的不稳定。达西数很大时,多孔介质接近消失,相当于整个区域都是纯流体的流动,没有了圆柱内部的渗流以及多孔介质的阻碍,流动更接近稳态,需要更大的雷诺数才能由稳态转变为非稳态。

\begin{table}[h]
	\caption{不同达西数下的临界雷诺数}\label{tab: critical Re}
	\vspace{.5em}\centering\wuhao
	\begin{tabular}{cc}
		\toprule[1.5pt]
		$Da$ & 临界 $Re$ \\
		\midrule[1pt]
		$1\times 10^{-5}$ & 40--45 \\
		$1\times 10^{-4}$ & 40--45 \\
		$1\times 10^{-3}$ & 40--45 \\
		$5\times 10^{-3}$ & ? \\
		$1\times 10^{-2}$ & 140--160 \\
		$1\times 10^{-1}$ & >200 \\
	\bottomrule[1.5pt]
	\end{tabular}
\end{table}

\subsection{不同参数下的流场图}

根据测量数据,可以得到不同达西数和雷诺数下涡量的等值线图。图~\ref{fig: vorticity-contour-1e-4} 选取了六幅 $Da=0.0001$ 时不同雷诺数下流动已达稳定 \footnote{此处“稳定”一词对应于“stable”,指流动特性不随时间变化,或随时间做周期性变化,因此不同于“稳态”,指特性不随时间变化。} 时涡量的等值线图。从图中可以看出,$Re=40$ 时流动处于稳态,并在圆柱的后端产生了一对很长的旋涡,随着雷诺数的增大,尾迹的长度逐渐缩短;$Re=45$ 时尾迹已经开始波动,流动开始处于非稳态,与前文所述一致;$Re=50$ 时尾迹更短,非稳态已比较明显,涡脱落之后向下游移动,形成一列整齐的涡街;之后更是完全进入了非稳态,尾迹的长度也越来越短。

%还可以得到非稳态情形下一个周期内涡的脱落情形。

\begin{figure}
	\centering
	\begin{minipage}{\textwidth}
		\centering
		\subfigure[$Re=40$]{\includegraphics[width=0.4\textwidth]{../figs/0.0001_40/flow}}
		\subfigure[$Re=45$]{\includegraphics[width=0.4\textwidth]{../figs/0.0001_45/flow}}
	\end{minipage}
	\centering
	\begin{minipage}{\textwidth}
		\centering
		\subfigure[$Re=50$]{\includegraphics[width=0.4\textwidth]{../figs/0.0001_50/flow}}
		\subfigure[$Re=90$]{\includegraphics[width=0.4\textwidth]{../figs/0.0001_90/flow}}
	\end{minipage}
	\centering
	\begin{minipage}{\textwidth}
		\centering
		\subfigure[$Re=140$]{\includegraphics[width=0.4\textwidth]{../figs/0.0001_140/flow}}
		\subfigure[$Re=200$]{\includegraphics[width=0.4\textwidth]{../figs/0.0001_200/flow}}
	\end{minipage}
	\caption{$Da=1\times 10^{-4}$ 时不同雷诺数下涡量的等值线图}
	\label{fig: vorticity-contour-1e-4}
\end{figure}

图~\ref{fig: vorticity-contour-1e-3}展示了 $Da=0.001$ 时不同雷诺数下涡量的等值线图。总体的变化情况和 $Da=0.0001$ 时相同,除了在更大的雷诺数下达到非稳态。

\begin{figure}
	\centering
	\begin{minipage}{\textwidth}
		\centering
		\subfigure[$Re=40$]{\includegraphics[width=0.4\textwidth]{../figs/0.001_40/flow}}
		\subfigure[$Re=45$]{\includegraphics[width=0.4\textwidth]{../figs/0.001_45/flow}}
	\end{minipage}
	\centering
	\begin{minipage}{\textwidth}
		\centering
		\subfigure[$Re=50$]{\includegraphics[width=0.4\textwidth]{../figs/0.001_50/flow}}
		\subfigure[$Re=90$]{\includegraphics[width=0.4\textwidth]{../figs/0.001_90/flow}}
	\end{minipage}
	\centering
	\begin{minipage}{\textwidth}
		\centering
		\subfigure[$Re=140$]{\includegraphics[width=0.4\textwidth]{../figs/0.001_140/flow}}
		\subfigure[$Re=200$]{\includegraphics[width=0.4\textwidth]{../figs/0.001_200/flow}}
	\end{minipage}
	\caption{$Da=1\times 10^{-3}$ 时不同雷诺数下涡量的等值线图}
	\label{fig: vorticity-contour-1e-3}
\end{figure}

图~\ref{fig: vorticity-contour-5e-3}展示了 $Da=0.005$ 时不同雷诺数下涡量的等值线图。总体的变化情况和 $Da=0.0001$ 时相同,除了在更大的雷诺数下达到非稳态。(尚无)

\begin{figure}
	\centering
	\begin{minipage}{\textwidth}
		\centering
		\subfigure[$Re=40$]{\includegraphics[width=0.4\textwidth]{../figs/0.0001_40/flow}}
		\subfigure[$Re=45$]{\includegraphics[width=0.4\textwidth]{../figs/0.0001_45/flow}}
	\end{minipage}
	\centering
	\begin{minipage}{\textwidth}
		\centering
		\subfigure[$Re=50$]{\includegraphics[width=0.4\textwidth]{../figs/0.0001_50/flow}}
		\subfigure[$Re=90$]{\includegraphics[width=0.4\textwidth]{../figs/0.0001_90/flow}}
	\end{minipage}
	\centering
	\begin{minipage}{\textwidth}
		\centering
		\subfigure[$Re=140$]{\includegraphics[width=0.4\textwidth]{../figs/0.0001_140/flow}}
		\subfigure[$Re=200$]{\includegraphics[width=0.4\textwidth]{../figs/0.0001_200/flow}}
	\end{minipage}
	\caption{$Da=5\times 10^{-3}$ 时不同雷诺数下涡量的等值线图(尚无)}
	\label{fig: vorticity-contour-5e-3}
\end{figure}

图~\ref{fig: vorticity-contour-1e-2}展示了 $Da=0.01$ 时不同雷诺数下涡量的等值线图。从图中可以看出,直到 $Re=200$ 时尾迹才具有并不大的波动,雷诺数更小的时候流动基本处于稳态。

\begin{figure}
	\centering
	\begin{minipage}{\textwidth}
		\centering
		\subfigure[$Re=50$]{\includegraphics[width=0.4\textwidth]{../figs/0.01_50/flow}}
		\subfigure[$Re=90$]{\includegraphics[width=0.4\textwidth]{../figs/0.01_90/flow}}
	\end{minipage}
	\centering
	\begin{minipage}{\textwidth}
		\centering
		\subfigure[$Re=140$]{\includegraphics[width=0.4\textwidth]{../figs/0.01_140/flow}}
		\subfigure[$Re=200$]{\includegraphics[width=0.4\textwidth]{../figs/0.01_200/flow}}
	\end{minipage}
	\caption{$Da=0.01$ 时不同雷诺数下涡量的等值线图}
	\label{fig: vorticity-contour-1e-2}
\end{figure}

\section{Strouhal 数和阻力、升力系数}

\subsection{Strouhal 数}

对于非稳态流动,流动波动的频率是一个重要参数。而 Strouhal 数是表述流体波动的无量纲频率。频率与周期互为倒数,周期可以根据测量数据(例如 Point 3 的水平速度)从图中得出。继而根据方程 \eqref{eq: St} 可以得到 Strouhal 数。式 \eqref{eq: St} 描述了这些物理量之间的关系:
\begin{equation}\label{eq: St}
	St = \frac{Df}{U} = f = \frac{1}{T}
\end{equation}
\begin{tabularx}{\textwidth}{@{}l@{\quad}r@{——}X@{}}
	式中 & $St$ & Strouhal 数;\\
		& $D$ & 圆柱的直径,$=1$ m;\\
		& $f$ & 波动的频率;\\
		& $U$ & 远方的来流速度,$=1$ m/s;\\
		& $T$ & 波动的周期。 
\end{tabularx}\vspace{3.15bp}
得到的周期和 Strouhal 数记录在 \texttt{record.xlsx} 中。

Strouhal 数随雷诺数的变化如图~\ref{fig: St} 所示。从图中可以看出,$Da$ 固定时,$St$ 随 $Re$ 的增大而增大。固定 $Re$ 时,$Da$ 越小,则 $St$ 越接近固体圆柱绕流的情形。当 $Da=1 \times 10^{-5}$ 时,$St$ 曲线已经和固体圆柱绕流的变化曲线 \cite{} 相吻合。

\begin{figure}
	\centering
	\includegraphics[width=0.8\textwidth]{../analysis/St_Re2}
	\caption{Strouhal 数随 $Re$ 的变化}
	\label{fig: St}
\end{figure}

对于非稳态流动,一个周期内的涡量图如~\ref{fig: 4*vortex} 所示。

\begin{figure}
	\centering
	\begin{minipage}{\textwidth}
		\centering
		\subfigure[0]{\includegraphics[width=0.4\textwidth]{../figs/0.0001_100/T0}}
		\subfigure[1/4]{\includegraphics[width=0.4\textwidth]{../figs/0.0001_100/T1}}
	\end{minipage}
	\centering
	\begin{minipage}{\textwidth}
		\centering
		\subfigure[2/4]{\includegraphics[width=0.4\textwidth]{../figs/0.0001_100/T2}}
		\subfigure[3/4]{\includegraphics[width=0.4\textwidth]{../figs/0.0001_100/T3}}
	\end{minipage}
	\caption{在 0,1/4,2/4,和 3/4 周期时涡的脱落图 $Da=0.0001$,$Re=100$}
	\label{fig: 4*vortex}
\end{figure}

\subsection{阻力系数}

非稳态情形下阻力和升力都随时间做周期性变化,所以此处取一个周期内的平均阻力和升力。平均阻力系数随雷诺数的变化如图~\ref{fig: meanCd} 所示。对于 $Da=0.0001$、$Da=0.001$,平均阻力系数随着雷诺数的增加而减小,然后随雷诺数的增加而增加。

\begin{figure}
	\centering
	\includegraphics[width=0.8\textwidth]{../analysis/meanCd_Re2}
	\caption{平均阻力系数随 $Re$ 的变化}
	\label{fig: meanCd}
\end{figure}

\subsection{升力系数}

升力系数波动的振幅随雷诺数的变化如图~\ref{fig: ClAmplitude} 所示,与文献结果一致 \cite{Park1998}。

\begin{figure}
	\centering
	\includegraphics[width=0.8\textwidth]{../analysis/ClAmplitude_Re}
	\caption{升力系数波动的振幅随 $Re$ 的变化}
	\label{fig: ClAmplitude}
\end{figure}

\section{能谱分析} % FFT

速度随时间的变化以及相应的 Fourier 变换如图~\ref{fig: velocity Fourier} 所示。升力系数随时间的变化以及相应的 Fourier 变换如图~\ref{fig: Cl Fourier} 所示。图中画出了稳定时的 10 个周期。

\begin{figure}
	\centering
	\begin{minipage}{\textwidth}
		\centering
		\includegraphics[width=0.8\textwidth]{../figs/0.0001_50/psd_U}
	\end{minipage}
	\centering
	\begin{minipage}{\textwidth}
		\centering
		\includegraphics[width=0.8\textwidth]{../figs/0.0001_50/psd_V}
	\end{minipage}
	\caption{Fourier 变换 $Da=0.0001$, $Re=50$.}
	\label{fig: velocity Fourier}
\end{figure}

\begin{figure}
	\centering
	\includegraphics[width=0.8\textwidth]{../figs/0.0001_50/psd_Cl}
	\caption{升力系数和 Fourier 变换 $Da=0.0001$, $Re=50$.}
	\label{fig: Cl Fourier}
\end{figure}

\section{多孔介质对流动状态的影响——与固体圆柱绕流相比}

\section{流动状态从稳态向非稳态的转变——与稳态圆柱绕流相比}

\section{本章小结}
