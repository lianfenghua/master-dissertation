% !Mode:: "TeX:UTF-8"
% !TEX root = ../main.tex
\chapter{流场结果与分析}\label{chap: flow pattern}

\section{瞬态行为}\label{sec: transient}

流动由雷诺数、达西数、孔隙率等多个参数决定,当前的研究中主要通过改变雷诺数和达西数得到不同的流动,并探究参数对流动的影响。非稳态流动的雷诺数范围大约为从 40 到 200, $Da=1\times 10^{-5}$ 时多孔区域接近固体,$Da=1\times 10^{-2}$ 时多孔介质的效果已经非常微弱,所以当前计算时设置的雷诺数从区间 $[40,200]$ 内选择,达西数从区间 $[10^{-5},10^{-2}]$ 内选择,表~\ref{tab: DaRe} 列出了达西数和雷诺数可能的组合值,实际计算过的值用对勾($\surd$)标记。本文未改变孔隙率和应力阶跃参数的值,依承文献 \inlinecite{Yu2007} 的设定,将孔隙率设为 0.7,将应力阶跃的两个系数均设为零:$\varepsilon=0.7$,$\beta_1 = \beta_2=0$。

\begin{table}
	\caption{计算时用到的达西数和雷诺数的值的设置}\label{tab: DaRe}
	\vspace{.5em}\centering\wuhao
	\begin{tabular}{*{7}{c}}
		\toprule[1.5pt]
		\multirow{2}[3]{*}{$Re$} & \multicolumn{6}{c}{$Da$} \\
		\cmidrule[.67pt](lr){2-7}
		& 0.00001 & 0.0001 & 0.0005 & 0.001 & 0.005 & 0.01 \\
		\midrule[1pt]
		40  & $\surd$ & $\surd$ & $\surd$ & $\surd$ & $\surd$ & -- \\
		41  & --      & $\surd$ & $\surd$ & $\surd$ & $\surd$ & -- \\
		42  & $\surd$ & $\surd$ & $\surd$ & $\surd$ & $\surd$ & -- \\
		43  & $\surd$ & $\surd$ & $\surd$ & $\surd$ & $\surd$ & -- \\
		44  & $\surd$ & $\surd$ & $\surd$ & $\surd$ & $\surd$ & -- \\
		45  & $\surd$ & $\surd$ & $\surd$ & $\surd$ & $\surd$ & -- \\
		50  & $\surd$ & $\surd$ & $\surd$ & $\surd$ & $\surd$ & -- \\
		60  & $\surd$ & $\surd$ & $\surd$ & $\surd$ & $\surd$ & $\surd$ \\
		70  & $\surd$ & $\surd$ & $\surd$ & $\surd$ & $\surd$ & $\surd$ \\
		80  & $\surd$ & $\surd$ & $\surd$ & $\surd$ & $\surd$ & $\surd$ \\
		90  & $\surd$ & $\surd$ & $\surd$ & $\surd$ & $\surd$ & $\surd$ \\
		100 & $\surd$ & $\surd$ & $\surd$ & $\surd$ & $\surd$ & $\surd$ \\
		120 & $\surd$ & $\surd$ & $\surd$ & $\surd$ & $\surd$ & $\surd$ \\
		140 & $\surd$ & $\surd$ & $\surd$ & $\surd$ & $\surd$ & $\surd$ \\
		160 & $\surd$ & $\surd$ & $\surd$ & $\surd$ & $\surd$ & $\surd$ \\
		180 & $\surd$ & $\surd$ & $\surd$ & $\surd$ & $\surd$ & $\surd$ \\
		200 & $\surd$ & $\surd$ & $\surd$ & $\surd$ & $\surd$ & $\surd$ \\
		\bottomrule[1.5pt]
	\end{tabular}
\end{table}

由结果可知,非稳态流动开始时所对应的雷诺数的值都在 40 之上,研究的第一步在于确定非稳态流动开始发生时所对应的临界雷诺数,从而确定此次研究的雷诺数范围。接着,本节还将展示非稳态流动的整体概貌,即不同参数下的流动特性,大体了解流动随雷诺数、达西数的变化情况。

根据测量数据,可以得到不同达西数和雷诺数下涡量的等值线图。图~\ref{fig: vorticity-contour-1e-4} 选取了六幅 $Da=0.0001$ 时不同雷诺数下流动已达平稳时涡量的等值线图。此处“平稳”一词指流动特性随时间做周期性变化,在时间平均意义下是稳定的(periodic but statistically stationary state),而“稳定”指流动特性不随时间发生任何变化,每一时刻的状态都是相同的(steady state)。从图中可以看出,$Re=40$ 时流动处于稳态,并在圆柱的后端产生了一对很长的漩涡,随着雷诺数的增大,尾迹的长度逐渐缩短;$Re=45$ 时尾迹已经开始波动,流动开始处于非稳态,与前文所述一致;$Re=50$ 时尾迹更短,非稳态已比较明显,涡脱落之后向下游移动,形成一列整齐的涡街;之后更是完全进入了非稳态,尾迹的长度也越来越短。

\begin{figure}
	\centering
	\begin{minipage}{\textwidth}
		\centering
		\subfigure[$Re=40$]{\includegraphics[width=0.47\textwidth]{../figs/0.0001_40/flow}}
		\subfigure[$Re=45$]{\includegraphics[width=0.47\textwidth]{../figs/0.0001_45/flow}}
	\end{minipage}
	\centering
	\begin{minipage}{\textwidth}
		\centering
		\subfigure[$Re=50$]{\includegraphics[width=0.47\textwidth]{../figs/0.0001_50/flow}}
		\subfigure[$Re=90$]{\includegraphics[width=0.47\textwidth]{../figs/0.0001_90/flow}}
	\end{minipage}
	\centering
	\begin{minipage}{\textwidth}
		\centering
		\subfigure[$Re=140$]{\includegraphics[width=0.47\textwidth]{../figs/0.0001_140/flow}}
		\subfigure[$Re=200$]{\includegraphics[width=0.47\textwidth]{../figs/0.0001_200/flow}}
	\end{minipage}
	\caption{$Da=0.0001$ 时不同雷诺数下涡量的等值线图}
	\label{fig: vorticity-contour-1e-4}
\end{figure}

图~\ref{fig: vorticity-contour-1e-3}展示了 $Da=0.001$ 时不同雷诺数下涡量的等值线图。总体的变化情况和 $Da=0.0001$ 时相同,除了在更大的雷诺数下达到非稳态。

\begin{figure}
	\centering
	\begin{minipage}{\textwidth}
		\centering
		\subfigure[$Re=40$]{\includegraphics[width=0.47\textwidth]{../figs/0.001_40/flow}}
		\subfigure[$Re=45$]{\includegraphics[width=0.47\textwidth]{../figs/0.001_45/flow}}
	\end{minipage}
	\centering
	\begin{minipage}{\textwidth}
		\centering
		\subfigure[$Re=50$]{\includegraphics[width=0.47\textwidth]{../figs/0.001_50/flow}}
		\subfigure[$Re=90$]{\includegraphics[width=0.47\textwidth]{../figs/0.001_90/flow}}
	\end{minipage}
	\centering
	\begin{minipage}{\textwidth}
		\centering
		\subfigure[$Re=140$]{\includegraphics[width=0.47\textwidth]{../figs/0.001_140/flow}}
		\subfigure[$Re=200$]{\includegraphics[width=0.47\textwidth]{../figs/0.001_200/flow}}
	\end{minipage}
	\caption{$Da=0.001$ 时不同雷诺数下涡量的等值线图}
	\label{fig: vorticity-contour-1e-3}
\end{figure}

图~\ref{fig: vorticity-contour-5e-3}展示了 $Da=0.005$ 时不同雷诺数下涡量的等值线图。总体的变化情况和 $Da=0.0001$ 时相同,除了在更大的雷诺数下达到非稳态。%(尚无)

\begin{figure}
	\centering
	\begin{minipage}{\textwidth}
		\centering
		\subfigure[$Re=40$]{\includegraphics[width=0.47\textwidth]{../figs/0.0001_40/flow}}
		\subfigure[$Re=45$]{\includegraphics[width=0.47\textwidth]{../figs/0.0001_45/flow}}
	\end{minipage}
	\centering
	\begin{minipage}{\textwidth}
		\centering
		\subfigure[$Re=50$]{\includegraphics[width=0.47\textwidth]{../figs/0.0001_50/flow}}
		\subfigure[$Re=90$]{\includegraphics[width=0.47\textwidth]{../figs/0.0001_90/flow}}
	\end{minipage}
	\centering
	\begin{minipage}{\textwidth}
		\centering
		\subfigure[$Re=140$]{\includegraphics[width=0.47\textwidth]{../figs/0.0001_140/flow}}
		\subfigure[$Re=200$]{\includegraphics[width=0.47\textwidth]{../figs/0.0001_200/flow}}
	\end{minipage}
	\caption{$Da=0.005$ 时不同雷诺数下涡量的等值线图}%(尚无)
	\label{fig: vorticity-contour-5e-3}
\end{figure}

图~\ref{fig: vorticity-contour-1e-2}展示了 $Da=0.01$ 时不同雷诺数下涡量的等值线图。从图中可以看出,直到 $Re=200$ 时尾迹才具有并不大的波动,雷诺数更小的时候流动基本处于稳态。

\begin{figure}
	\centering
	\begin{minipage}{\textwidth}
		\centering
		\subfigure[$Re=50$]{\includegraphics[width=0.47\textwidth]{../figs/0.01_50/flow}}
		\subfigure[$Re=90$]{\includegraphics[width=0.47\textwidth]{../figs/0.01_90/flow}}
	\end{minipage}
	\centering
	\begin{minipage}{\textwidth}
		\centering
		\subfigure[$Re=140$]{\includegraphics[width=0.47\textwidth]{../figs/0.01_140/flow}}
		\subfigure[$Re=200$]{\includegraphics[width=0.47\textwidth]{../figs/0.01_200/flow}}
	\end{minipage}
	\caption{$Da=0.01$ 时不同雷诺数下涡量的等值线图}
	\label{fig: vorticity-contour-1e-2}
\end{figure}

\section{平均值行为}\label{sec: average}

\section{本章小结}

本章在非稳态层流范围内选取了合适的例子进行计算,设置了合适的参数值,获得输出的数据文件,对数据进行初步处理,得到了一些重要物理量的值,然后对所得结果进行分析。
