% !Mode:: "TeX:UTF-8"
% !TEX root = ../main.tex
\chapter{流场结果与分析}\label{chap: flow pattern}

流动由雷诺数、达西数、孔隙率等多个参数决定,当前的研究中主要通过改变雷诺数和达西数得到不同的流动,并探究参数对流动的影响。非稳态流动的雷诺数范围大约为从 40 到 200, $Da=1\times 10^{-5}$ 时多孔区域接近固体,$Da=1\times 10^{-2}$ 时多孔介质的效果已经非常微弱,所以当前计算时设置的雷诺数从区间 $[40,200]$ 内选择,达西数从区间 $[10^{-5},10^{-2}]$ 内选择,表~\ref{tab: DaRe} 列出了达西数和雷诺数可能的组合值,实际计算过的值用对勾($\surd$)标记。本文未改变孔隙率和应力阶跃参数的值,依承文献 \inlinecite{Yu2007} 的设定,将孔隙率设定为 $\varepsilon=0.7$,将应力阶跃的两个系数设定为 $\beta_1=\beta_2=0$。

\begin{table}
	\caption{计算时用到的达西数和雷诺数的值的设置}\label{tab: DaRe}
	\vspace{.5em}\centering\wuhao
	\begin{tabular}{*{7}{c}}
		\toprule[1.5pt]
		\multirow{2}[3]{*}{$Re$} & \multicolumn{6}{c}{$Da$} \\
		\cmidrule[.67pt](lr){2-7}
		& 0.00001 & 0.0001 & 0.0005 & 0.001 & 0.005 & 0.01 \\
		\midrule[1pt]
		40  & $\surd$ & $\surd$ & $\surd$ & $\surd$ & $\surd$ & -- \\
		41  & --      & $\surd$ & $\surd$ & $\surd$ & $\surd$ & -- \\
		42  & $\surd$ & $\surd$ & $\surd$ & $\surd$ & $\surd$ & -- \\
		43  & $\surd$ & $\surd$ & $\surd$ & $\surd$ & $\surd$ & -- \\
		44  & $\surd$ & $\surd$ & $\surd$ & $\surd$ & $\surd$ & -- \\
		45  & $\surd$ & $\surd$ & $\surd$ & $\surd$ & $\surd$ & -- \\
		50  & $\surd$ & $\surd$ & $\surd$ & $\surd$ & $\surd$ & -- \\
		60  & $\surd$ & $\surd$ & $\surd$ & $\surd$ & $\surd$ & $\surd$ \\
		70  & $\surd$ & $\surd$ & $\surd$ & $\surd$ & $\surd$ & $\surd$ \\
		80  & $\surd$ & $\surd$ & $\surd$ & $\surd$ & $\surd$ & $\surd$ \\
		90  & $\surd$ & $\surd$ & $\surd$ & $\surd$ & $\surd$ & $\surd$ \\
		100 & $\surd$ & $\surd$ & $\surd$ & $\surd$ & $\surd$ & $\surd$ \\
		120 & $\surd$ & $\surd$ & $\surd$ & $\surd$ & $\surd$ & $\surd$ \\
		140 & $\surd$ & $\surd$ & $\surd$ & $\surd$ & $\surd$ & $\surd$ \\
		160 & $\surd$ & $\surd$ & $\surd$ & $\surd$ & $\surd$ & $\surd$ \\
		180 & $\surd$ & $\surd$ & $\surd$ & $\surd$ & $\surd$ & $\surd$ \\
		200 & $\surd$ & $\surd$ & $\surd$ & $\surd$ & $\surd$ & $\surd$ \\
		\bottomrule[1.5pt]
	\end{tabular}
\end{table}

根据表~\ref{tab: DaRe} 的设定进行计算,可以得到不同参数下计算出的流场数据,由于大多参数都处于非稳态范围内,所以得到的流场为各个时刻的瞬时流场,不同时刻的涡量、流线体现了流动特性随时间的变化,展示出非稳态流动的整体概貌。通过对一个周期内的流场进行平均化处理,可以得到平均流场,平均流场的分布情形与稳态流动类似,通过对各个平均量的空间分布进行分析,可以了解流动随雷诺数、达西数的变化情况。

\section{流场的瞬时分布}\label{sec: transient}

%涡量等值线-远程
经过计算,可以得到不同达西数和雷诺数下整个流动区域的涡量等值线,图~\ref{fig: vortlong-1e-4} $\sim$ \ref{fig: vortlong-1e-2} 分别展示了 $Da=0.0001,0.001,0.01$ 时不同雷诺数下流动已达平稳时的涡量等值线。此处“平稳”一词指流动特性随时间做周期性变化,在时间平均意义下是稳定的(periodic but statistically stationary state),而“稳定”指流动特性不随时间发生任何变化,每一时刻的状态都是相同的(steady state)。图中涡量值的变化范围是 $-1$ 到 1,圆柱背面具有正、负涡量值的区域成对出现,表示一对对脱落的涡。以达西数为 0.0001 时的图~\ref{fig: vortlong-1e-4} 为例,从图中可以看出,$Re=40$ 时流动处于稳态,并在圆柱的后端产生了一对对称的漩涡,随着雷诺数的增大,圆柱两侧形成的涡的长度逐渐缩短;$Re=45$ 时尾迹已经开始波动,流动开始处于非稳态;$Re=50$ 时涡的长度变短,非稳态已比较明显,涡脱落之后向下游移动,形成一列整齐的涡街;之后更是完全进入了非稳态,涡的长度也越来越短,一对对的涡向后整齐排列。当雷诺数达到 200 时,离圆柱较远处的涡已经变得不再规律,可能是因为此时的二维模拟结果难以展现三维效应,流体存在沿圆柱长度方向的流动现象。

图~\ref{fig: vortlong-1e-3} 展示了 $Da=0.001$ 时不同雷诺数下涡量的等值线图。总体的变化情况和 $Da=0.0001$ 时相同,除了在更大的雷诺数下达到非稳态。图~\ref{fig: vortlong-1e-2} 展示了 $Da=0.01$ 时不同雷诺数下涡量的等值线图。从图中可以看出,直到 $Re=200$ 时尾迹才具有不太大的波动,雷诺数更小的时候流动基本处于稳态。

Karasudani 和 Funakoshi\cite{Karasudani1994} 在实验中研究了圆柱绕流远尾迹处涡街结构的演化,发现沿着下游经过一段距离之后,初始涡街分解成了几乎平行的剪切流动,具有高斯分布,再经过一段距离,形成了具有更大尺度的第二涡街。图~\ref{fig: vortlong-1e-4} 展示了 $Re=50,70,90,120,160,200$ 时涡量的演化情况。从 $Re=70$ 开始,尾迹中的涡沿着下游逐渐演化成了近乎平行的剪切流动。$Re>50$ 时可以观察到两次卡门涡的失真和旋转。卡门涡大约形成于 $x=5$ 的位置并从圆柱表面脱落,同时涡的中心离开了尾迹的中心线。第二次形成于 $x=20$ 或更远的位置。在 $x=15$ 以后,涡中心距尾迹中心线($y=0$)之间的垂直距离保持不变。沿着尾迹中心线,涡的形状从圆变得扭曲,类似于椭圆形。从 $Re=120$ 开始,$x=25$ 之后的远尾迹处相邻的涡逐渐出现融合趋势,到 $Re=200$ 时已经融合起来成为一体。%?

\begin{figure}
	\setlength{\subfigcapskip}{-1bp}
	\centering
	\includegraphics[width=0.7\textwidth]{../analysis/flow/vort_legend_line}
	\begin{minipage}{\textwidth}
		\centering
		\subfigure[$Re=40$]{\includegraphics[width=0.47\textwidth]{../analysis/flow/{0.0001_40_long_line}.png}}
		\subfigure[$Re=45$]{\includegraphics[width=0.47\textwidth]{../analysis/flow/{0.0001_45_long_line}.png}}
	\end{minipage}
	\centering
	\begin{minipage}{\textwidth}
		\centering
		\subfigure[$Re=50$]{\includegraphics[width=0.47\textwidth]{../analysis/flow/{0.0001_50_long_line}.png}}
		\subfigure[$Re=70$]{\includegraphics[width=0.47\textwidth]{../analysis/flow/{0.0001_70_long_line}.png}}
	\end{minipage}
	\centering
	\begin{minipage}{\textwidth}
		\centering
		\subfigure[$Re=90$]{\includegraphics[width=0.47\textwidth]{../analysis/flow/{0.0001_90_long_line}.png}}
		\subfigure[$Re=120$]{\includegraphics[width=0.47\textwidth]{../analysis/flow/{0.0001_120_long_line}.png}}
	\end{minipage}
	\centering
	\begin{minipage}{\textwidth}
		\centering
		\subfigure[$Re=160$]{\includegraphics[width=0.47\textwidth]{../analysis/flow/{0.0001_160_long_line}.png}}
		\subfigure[$Re=200$]{\includegraphics[width=0.47\textwidth]{../analysis/flow/{0.0001_200_long_line}.png}}
	\end{minipage}
	\vskip 0.2em
	\wuhao 注:图中所示的流动区域为 $-1<x<30$,$-5<y<5$。
	\vspace{0.2em}
	\caption{不同雷诺数下的涡量等值线($Da=0.0001$)}
	\label{fig: vortlong-1e-4}
\end{figure}

\begin{figure}
	\setlength{\subfigcapskip}{-1bp}
	\centering
	\includegraphics[width=0.7\textwidth]{../analysis/flow/vort_legend_line}
	\begin{minipage}{\textwidth}
		\centering
		\subfigure[$Re=40$]{\includegraphics[width=0.47\textwidth]{../analysis/flow/{0.001_40_long_line}.png}}
		\subfigure[$Re=45$]{\includegraphics[width=0.47\textwidth]{../analysis/flow/{0.001_45_long_line}.png}}
	\end{minipage}
	\centering
	\begin{minipage}{\textwidth}
		\centering
		\subfigure[$Re=50$]{\includegraphics[width=0.47\textwidth]{../analysis/flow/{0.001_50_long_line}.png}}
		\subfigure[$Re=70$]{\includegraphics[width=0.47\textwidth]{../analysis/flow/{0.001_70_long_line}.png}}
	\end{minipage}
	\centering
	\begin{minipage}{\textwidth}
		\centering
		\subfigure[$Re=90$]{\includegraphics[width=0.47\textwidth]{../analysis/flow/{0.001_90_long_line}.png}}
		\subfigure[$Re=120$]{\includegraphics[width=0.47\textwidth]{../analysis/flow/{0.001_120_long_line}.png}}
	\end{minipage}
	\centering
	\begin{minipage}{\textwidth}
		\centering
		\subfigure[$Re=160$]{\includegraphics[width=0.47\textwidth]{../analysis/flow/{0.001_160_long_line}.png}}
		\subfigure[$Re=200$]{\includegraphics[width=0.47\textwidth]{../analysis/flow/{0.001_200_long_line}.png}}
	\end{minipage}
	\vskip 0.2em
	\wuhao 注:图中所示的流动区域为 $-1<x<30$,$-5<y<5$。
	\vspace{0.2em}
	\caption{不同雷诺数下的涡量等值线($Da=0.001$)}
	\label{fig: vortlong-1e-3}
\end{figure}

\begin{figure}
	\setlength{\subfigcapskip}{-1bp}
	\centering
	\includegraphics[width=0.7\textwidth]{../analysis/flow/vort_legend_line}
	\begin{minipage}{\textwidth}
		\centering
		\subfigure[$Re=60$]{\includegraphics[width=0.47\textwidth]{../analysis/flow/{0.01_60_long_line}.png}}
		\subfigure[$Re=70$]{\includegraphics[width=0.47\textwidth]{../analysis/flow/{0.01_70_long_line}.png}}
	\end{minipage}
	\centering
	\begin{minipage}{\textwidth}
		\centering
		\subfigure[$Re=90$]{\includegraphics[width=0.47\textwidth]{../analysis/flow/{0.01_90_long_line}.png}}
		\subfigure[$Re=120$]{\includegraphics[width=0.47\textwidth]{../analysis/flow/{0.01_120_long_line}.png}}
	\end{minipage}
	\centering
	\begin{minipage}{\textwidth}
		\centering
		\subfigure[$Re=160$]{\includegraphics[width=0.47\textwidth]{../analysis/flow/{0.01_160_long_line}.png}}
		\subfigure[$Re=200$]{\includegraphics[width=0.47\textwidth]{../analysis/flow/{0.01_200_long_line}.png}}
	\end{minipage}
	\vskip 0.2em
	\wuhao 注:图中所示的流动区域为 $-1<x<30$,$-5<y<5$。
	\vspace{0.2em}
	\caption{不同雷诺数下的涡量等值线($Da=0.01$)}
	\label{fig: vortlong-1e-2}
\end{figure}

%涡量等值线-近程
%近程涡量的等值线图如图~\ref{fig: vort-1e-4} $\sim$ 图~\ref{fig: vort-1e-2} 所示。

% \begin{figure}
% 	\setlength{\subfigcapskip}{-1bp}
% 	\centering
% 	\includegraphics[width=0.7\textwidth]{../analysis/flow/vort_legend}
% 	\begin{minipage}{\textwidth}
% 		\centering
% 		\subfigure[$Re=40$]{\includegraphics[width=0.47\textwidth]{../analysis/flow/{0.0001_40}.png}}
% 		\subfigure[$Re=45$]{\includegraphics[width=0.47\textwidth]{../analysis/flow/{0.0001_45}.png}}
% 	\end{minipage}
% 	\centering
% 	\begin{minipage}{\textwidth}
% 		\centering
% 		\subfigure[$Re=50$]{\includegraphics[width=0.47\textwidth]{../analysis/flow/{0.0001_50}.png}}
% 		\subfigure[$Re=70$]{\includegraphics[width=0.47\textwidth]{../analysis/flow/{0.0001_70}.png}}
% 	\end{minipage}
% 	\centering
% 	\begin{minipage}{\textwidth}
% 		\centering
% 		\subfigure[$Re=90$]{\includegraphics[width=0.47\textwidth]{../analysis/flow/{0.0001_90}.png}}
% 		\subfigure[$Re=120$]{\includegraphics[width=0.47\textwidth]{../analysis/flow/{0.0001_120}.png}}
% 	\end{minipage}
% 	\centering
% 	\begin{minipage}{\textwidth}
% 		\centering
% 		\subfigure[$Re=160$]{\includegraphics[width=0.47\textwidth]{../analysis/flow/{0.0001_160}.png}}
% 		\subfigure[$Re=200$]{\includegraphics[width=0.47\textwidth]{../analysis/flow/{0.0001_200}.png}}
% 	\end{minipage}
% 	\vskip 0.2em
% 	\wuhao 注:图中所示的流动区域为 $-1<x<15$,$-4<y<4$。
% 	\vspace{0.2em}
% 	\caption{近尾迹处不同雷诺数下涡量的等值线($Da=0.0001$)}
% 	\label{fig: vort-1e-4}
% \end{figure}

% \begin{figure}
% 	\setlength{\subfigcapskip}{-1bp}
% 	\centering
% 	\includegraphics[width=0.7\textwidth]{../analysis/flow/vort_legend}
% 	\begin{minipage}{\textwidth}
% 		\centering
% 		\subfigure[$Re=40$]{\includegraphics[width=0.47\textwidth]{../analysis/flow/{0.001_40}.png}}
% 		\subfigure[$Re=45$]{\includegraphics[width=0.47\textwidth]{../analysis/flow/{0.001_45}.png}}
% 	\end{minipage}
% 	\centering
% 	\begin{minipage}{\textwidth}
% 		\centering
% 		\subfigure[$Re=50$]{\includegraphics[width=0.47\textwidth]{../analysis/flow/{0.001_50}.png}}
% 		\subfigure[$Re=70$]{\includegraphics[width=0.47\textwidth]{../analysis/flow/{0.001_70}.png}}
% 	\end{minipage}
% 	\centering
% 	\begin{minipage}{\textwidth}
% 		\centering
% 		\subfigure[$Re=90$]{\includegraphics[width=0.47\textwidth]{../analysis/flow/{0.001_90}.png}}
% 		\subfigure[$Re=120$]{\includegraphics[width=0.47\textwidth]{../analysis/flow/{0.001_120}.png}}
% 	\end{minipage}
% 	\centering
% 	\begin{minipage}{\textwidth}
% 		\centering
% 		\subfigure[$Re=160$]{\includegraphics[width=0.47\textwidth]{../analysis/flow/{0.001_160}.png}}
% 		\subfigure[$Re=200$]{\includegraphics[width=0.47\textwidth]{../analysis/flow/{0.001_200}.png}}
% 	\end{minipage}
% 	\vskip 0.2em
% 	\wuhao 注:图中所示的流动区域为 $-1<x<15$,$-4<y<4$。
% 	\vspace{0.2em}
% 	\caption{近尾迹处不同雷诺数下涡量的等值线($Da=0.001$)}
% 	\label{fig: vort-1e-3}
% \end{figure}

% \begin{figure}
% 	\setlength{\subfigcapskip}{-1bp}
% 	\centering
% 	\includegraphics[width=0.7\textwidth]{../analysis/flow/vort_legend}
% 	\begin{minipage}{\textwidth}
% 		\centering
% 		\subfigure[$Re=60$]{\includegraphics[width=0.47\textwidth]{../analysis/flow/{0.01_60}.png}}
% 		\subfigure[$Re=70$]{\includegraphics[width=0.47\textwidth]{../analysis/flow/{0.01_70}.png}}
% 	\end{minipage}
% 	\centering
% 	\begin{minipage}{\textwidth}
% 		\centering
% 		\subfigure[$Re=90$]{\includegraphics[width=0.47\textwidth]{../analysis/flow/{0.01_90}.png}}
% 		\subfigure[$Re=120$]{\includegraphics[width=0.47\textwidth]{../analysis/flow/{0.01_120}.png}}
% 	\end{minipage}
% 	\centering
% 	\begin{minipage}{\textwidth}
% 		\centering
% 		\subfigure[$Re=160$]{\includegraphics[width=0.47\textwidth]{../analysis/flow/{0.01_160}.png}}
% 		\subfigure[$Re=200$]{\includegraphics[width=0.47\textwidth]{../analysis/flow/{0.01_200}.png}}
% 	\end{minipage}
% 	\vskip 0.2em
% 	\wuhao 注:图中所示的流动区域为 $-1<x<15$,$-4<y<4$。
% 	\vspace{0.2em}
% 	\caption{近尾迹处不同雷诺数下涡量的等值线($Da=0.01$)}
% 	\label{fig: vort-1e-2}
% \end{figure}

%流线-近程
%近程瞬时速度流线图如图~\ref{fig: streamfunc-1e-4} 所示。

% \begin{figure}
% 	\setlength{\subfigcapskip}{-1bp}
% 	\centering
% 	\begin{minipage}{\textwidth}
% 		\centering
% 		\subfigure[$Re=40$]{\includegraphics[width=0.47\textwidth]{../analysis/flow/{0.0001_40_streamfn}.png}}
% 		\subfigure[$Re=45$]{\includegraphics[width=0.47\textwidth]{../analysis/flow/{0.0001_45_streamfn}.png}}
% 	\end{minipage}
% 	\centering
% 	\begin{minipage}{\textwidth}
% 		\centering
% 		\subfigure[$Re=50$]{\includegraphics[width=0.47\textwidth]{../analysis/flow/{0.0001_50_streamfn}.png}}
% 		\subfigure[$Re=90$]{\includegraphics[width=0.47\textwidth]{../analysis/flow/{0.0001_90_streamfn}.png}}
% 	\end{minipage}
% 	\centering
% 	\begin{minipage}{\textwidth}
% 		\centering
% 		\subfigure[$Re=140$]{\includegraphics[width=0.47\textwidth]{../analysis/flow/{0.0001_140_streamfn}.png}}
% 		\subfigure[$Re=200$]{\includegraphics[width=0.47\textwidth]{../analysis/flow/{0.0001_200_streamfn}.png}}
% 	\end{minipage}
% 	\vspace{0.2em}
% 	\caption{不同雷诺数下的流线图($Da=0.0001$)}
% 	\label{fig: streamfunc-1e-4}
% \end{figure}

%一个周期内的流动变化,四个涡脱落图和四个流线图
对于非稳态流动,得到流动的周期后,可以继续观察流动在一个周期内的变化情况。以 $Da=0.0001$、$Re=100$ 为例,图~\ref{fig: 4*vortex} 和图~\ref{fig: 4*stream} 分别显示了一个周期内的涡量等值线图和流线图。如图所示,四个时刻的流动是一致的且连续变化而来,$t=0$ 时刻圆柱背面下方有一个涡准备脱落,它的右侧有一个刚刚脱落的涡;四分之一周期过后,$t=1/4\,T$,下方的涡已经快要脱落;$t=2/4\,T$ 时刻的状态和 $t=0$ 时刻相反,上下对称,即圆柱背面的上方有一个准备脱落的涡;再过四分之一周期,$t=3/4\,T$,此时流动的状态又与 $t=1/4\,T$ 时刻对称。之后,随着时间的演进,这一周期内产生的涡从左向右移动,最终耗散在流体中,同时有新的涡不断产生,重复着这一周期内的现象。

\begin{figure}
	\setlength{\subfigcapskip}{-1bp}
	\centering
	\includegraphics[width=0.7\textwidth]{../analysis/flow/vort_legend_line}
	\begin{minipage}{\textwidth}
		\centering
		\subfigure[$t=0$]{\includegraphics[width=0.47\textwidth]{../analysis/period/{0.0001_100_T0}.png}}
		\subfigure[$t=1/4\,T$]{\includegraphics[width=0.47\textwidth]{../analysis/period/{0.0001_100_T1}.png}}
	\end{minipage}
	\centering
	\begin{minipage}{\textwidth}
		\centering
		\subfigure[$t=2/4\,T$]{\includegraphics[width=0.47\textwidth]{../analysis/period/{0.0001_100_T2}.png}}
		\subfigure[$t=3/4\,T$]{\includegraphics[width=0.47\textwidth]{../analysis/period/{0.0001_100_T3}.png}}
	\end{minipage}
	\vskip 0.2em
	\wuhao 注:图中所示的流动区域为 $-1<x<30$,$-5<y<5$。
	\vspace{0.2em}
	\caption{在 0,1/4,2/4,和 3/4 周期时涡的脱落图($Da=0.0001$,$Re=100$)}
	\label{fig: 4*vortex}
\end{figure}

\begin{figure}
	\setlength{\subfigcapskip}{-1bp}
	\centering
	\begin{minipage}{\textwidth}
		\centering
		\subfigure[$t=0$]{\includegraphics[width=0.47\textwidth]{../analysis/period/{0.0001_100_streamfnT0}.png}}
		\subfigure[$t=1/4\,T$]{\includegraphics[width=0.47\textwidth]{../analysis/period/{0.0001_100_streamfnT1}.png}}
	\end{minipage}
	\centering
	\begin{minipage}{\textwidth}
		\centering
		\subfigure[$t=2/4\,T$]{\includegraphics[width=0.47\textwidth]{../analysis/period/{0.0001_100_streamfnT2}.png}}
		\subfigure[$t=3/4\,T$]{\includegraphics[width=0.47\textwidth]{../analysis/period/{0.0001_100_streamfnT3}.png}}
	\end{minipage}
	\vskip 0.2em
	\vspace{0.2em}
	\caption{在 0,1/4,2/4,和 3/4 周期时的流线图($Da=0.0001$,$Re=100$)}
	\label{fig: 4*stream}
\end{figure}

\clearpage
\section{平均流场的空间分布}\label{sec: average}

%\subsection{平均流场}

通过对一个周期内若干个时刻的流场数据进行平均化处理,可以得到一个周期内的平均流场数据,进而可以得到不同参数下的平均流场特性。图~\ref{fig: average-vort} 和图~\ref{fig: average-strfn} 分别展示了 $Da=0.0001$ 时不同雷诺数下平均流场的涡量等值线和流线图,达西数取其他值时的流场图与之类似。从图中可以看出流场几何特性的变化,随着雷诺数的增大,尾迹的长度 $L_f$ 逐渐缩短,这一数值可从下文对图~\ref{fig: average-center-1e-05} 的说明中得到,不同达西数和雷诺数下的回流区长度见表~\ref{tab: geometry},其中 $x_0$ 表示尾迹末端平均速度为零时的位置,$x_m$ 表示尾迹之后平均速度取得最大值时的位置,$L_f$ 即为回流区的长度,均在下文平均量的分布中有所说明。

\begin{table}[ht]
	\caption{不同达西数和雷诺数下平均流场的几何特性}\label{tab: geometry}
	\vspace{.5em}\centering\wuhao
	\begin{tabular}{*{10}{c}}
		\toprule[1.5pt]
		\multirow{2}[3]{*}{$Re$} & \multicolumn{3}{c}{$Da=1\times 10^{-5}$} & \multicolumn{3}{c}{$Da=0.0001$} & \multicolumn{3}{c}{$Da=0.001$} \\
		\cmidrule[.67pt](lr){2-4} \cmidrule[.67pt](lr){5-7} \cmidrule[.67pt](lr){8-10}
		& $x_0$ & $x_m$ & $L_f$ & $x_0$ & $x_m$ & $L_f$ & $x_0$ & $x_m$ & $L_f$ \\
		\midrule[1pt]
		50	& 2.92 & 无    & 2.42 & 2.89 & 无    & 2.39 & 2.92 & 无    & 2.42 \\
		70	& 2.33 & 22.07 & 1.83 & 2.30 & 21.61 & 1.80 & 2.30 & 20.28 & 1.80 \\
		100 & 1.91 & 8.01  & 1.41 & 1.87 & 7.68  & 1.37 & 1.94 & 7.52  & 1.44 \\
		120 & 1.74 & 6.36  & 1.24 & 1.69 & 6.23  & 1.19 & 1.82 & 6.23  & 1.32 \\
		160 & 1.49 & 4.95  & 0.99 & 1.44 & 4.65  & 0.94 & 1.69 & 5.27  & 1.19 \\
		200 & 1.34 & 4.19  & 0.84 & 1.28 & 4.01  & 0.78 & 1.64 & 4.84  & 1.14 \\
		\bottomrule[1.5pt]
	\end{tabular}
\end{table}

\begin{figure}
	\setlength{\subfigcapskip}{-1bp}
	\centering
	\includegraphics[width=0.7\textwidth]{../analysis/flow/vort_legend_line}	
	\begin{minipage}{\textwidth}
		\centering
		\subfigure[$Re=50$]{\includegraphics[width=0.47\textwidth]{../analysis/average/flow/{0.0001_50}.png}}
		\subfigure[$Re=70$]{\includegraphics[width=0.47\textwidth]{../analysis/average/flow/{0.0001_70}.png}}
	\end{minipage}
	\centering
	\begin{minipage}{\textwidth}
		\centering
		\subfigure[$Re=100$]{\includegraphics[width=0.47\textwidth]{../analysis/average/flow/{0.0001_100}.png}}
		\subfigure[$Re=120$]{\includegraphics[width=0.47\textwidth]{../analysis/average/flow/{0.0001_120}.png}}
	\end{minipage}
	\centering
	\begin{minipage}{\textwidth}
		\centering
		\subfigure[$Re=160$]{\includegraphics[width=0.47\textwidth]{../analysis/average/flow/{0.0001_160}.png}}
		\subfigure[$Re=200$]{\includegraphics[width=0.47\textwidth]{../analysis/average/flow/{0.0001_200}.png}}
	\end{minipage}
	\vskip 0.2em
	\wuhao 注:图中所示的流动区域为 $-1<x<8$,$-2<y<2$。
	\vspace{0.2em}
	\caption{不同雷诺数下平均流场的涡量等值线($Da=0.0001$)}
	\label{fig: average-vort}
\end{figure}

\begin{figure}
	\setlength{\subfigcapskip}{-1bp}
	\centering
	\begin{minipage}{\textwidth}
		\centering
		\subfigure[$Re=50$]{\includegraphics[width=0.47\textwidth]{../analysis/average/flow/{0.0001_50_strfn}.png}}
		\subfigure[$Re=70$]{\includegraphics[width=0.47\textwidth]{../analysis/average/flow/{0.0001_70_strfn}.png}}
	\end{minipage}
	\centering
	\begin{minipage}{\textwidth}
		\centering
		\subfigure[$Re=100$]{\includegraphics[width=0.47\textwidth]{../analysis/average/flow/{0.0001_100_strfn}.png}}
		\subfigure[$Re=120$]{\includegraphics[width=0.47\textwidth]{../analysis/average/flow/{0.0001_120_strfn}.png}}
	\end{minipage}
	\centering
	\begin{minipage}{\textwidth}
		\centering
		\subfigure[$Re=160$]{\includegraphics[width=0.47\textwidth]{../analysis/average/flow/{0.0001_160_strfn}.png}}
		\subfigure[$Re=200$]{\includegraphics[width=0.47\textwidth]{../analysis/average/flow/{0.0001_200_strfn}.png}}
	\end{minipage}
	\vskip 0.2em
	\wuhao 注:图中所示的流动区域为 $-1<x<3.6$,$-1<y<1$。
	\vspace{0.2em}
	\caption{不同雷诺数下平均流场的流线图($Da=0.0001$)}
	\label{fig: average-strfn}
\end{figure}

%\subsection{平均量沿尾迹中心线和圆柱表面的分布}

%沿尾迹中心线的分布(平均速度,方均根速度,平均压力系数,方均根压力系数)
图~\ref{fig: average-center-1e-05} 展示了 $Da=1\times 10^{-5}$ 雷诺数取不同值时时间平均量沿尾迹中心线的分布,每幅图中的六条曲线分别表示 $Re=50,70,100,120,160,200$ 时的变化趋势。

图~\ref{fig: average-center-1e-05} a) 为 $Da=1\times 10^{-5}$ 时流向平均速度 $\overline{u}$ 沿尾迹中心线($y=0, -30\leq x\leq 30$)的分布。当 $x<-10$ 时,平均速度基本等于来流速度,流动仍保持为自由来流处的流动。在 $-10<x<-0.5$ 区间内,由于受到了圆柱的阻挡,速度迅速减小,并且减小的速度越来越快,在 $x=-0.5$ 附近,速度曲线几乎垂直地下降,到圆柱的前缘点($x=-0.5$)时降为零。$-0.5<x<0.5$ 区间位于圆柱内部,速度基本保持为零,这说明,对 $Da=1\times 10^{-5}$ 而言,多孔介质内部的流动十分微弱,多孔的效果可以忽略,此时多孔圆柱和固体圆柱几乎没有区别。在圆柱背面,速度变成了负值,说明此时处于回流区内,速度为负值时所对应的区间长度即为回流区的长度 $L_f$。在回流区内,速度先增大,在回流区的中部达到最大值,然后逐渐减小,到回流区末端时又恢复为零,这一点的位置记为 $x_0$(在图~\ref{fig: average-center-1e-05} b) 中用符号“$\times$”来标记),从而 $L_f=x_0-0.5$。需要注意的是,当达西数较大时涡可能从圆柱后方脱离,此时如果记涡脱离的位置为 $x_0'$,那么回流区长度应为 $L_f=x_0-x_0'$。当 $x>x_0$ 时,已穿过了回流区,速度从零开始快速增大,这时,随着雷诺数取值的不同,速度的变化趋势也具有明显的差别。对于 $Re=50$,经过一段距离的增大之后,增长速度逐渐放缓,并最终保持为常数。对于大于 50 的雷诺数,速度经过一段距离的增大之后会达到局部极大值,然后开始减小,将达到极大值的位置记为 $x_m$。随着雷诺数的增大,平均速度在越过极值之后的减小行为将变得复杂,渐渐出现了几段不同的变化趋势。例如,对于 $Re=200$,速度在 $1.34<x<4.19$ 区间内增大并于 $x=4.19$ 处达到极大值,之后整体呈下降趋势,又可细分为四段:$9<x<16$ 和 $21<x<29$ 两段区间内速度保持不变,其他两段区间内则逐渐减小。在 50--200 范围内,雷诺数越小则分段的趋势越不明显,雷诺数越接近 200 则分段趋势越明显。在远尾迹处(例如 $x=30$),速度已无法恢复均匀来流的值,因为经过圆柱后速度发生了损失,且雷诺数越大,速度的损失也越大,由此可以定义尾迹损失速度为 $u_d=1-\overline{u}$。

流向平均速度在尾迹附近的放大图见图~\ref{fig: average-center-1e-05} b),选取的范围是 $-0.5<x<8$。从图中可以更明显地看出,当 $-0.5<x<0.5$ 时,速度保持为零。$Re=50$ 时,尾迹长度 $L_f=2.42$,随着雷诺数的增大,尾迹缩短,尾迹中回流的最大速度增加。离开回流区之后,增大雷诺数,平均速度的增速随之增大,而且可以在更短的距离内达到极大值。

图~\ref{fig: average-center-1e-05} c) 为 $Da=1\times 10^{-5}$ 时平均压力系数 $C_p$ 沿尾迹中心线的分布。从 $x=-30$ 处的来流开始,与速度一样,压力系数也几乎一直保持着来流处的数值 $C_p=0$。在接近圆柱前缘的地方,压力系数迅速增大到 $C_p=1$,在圆柱前缘的驻点位置,压力系数达到了最大值,最大值略大于 1。在圆柱的前方($-30<x<-0.5$),压力系数的变化趋势和速度完全相反,反映了这一过程中动能和压力能的相互转化。流体进入圆柱之后,压力系数急剧下降,下降的速度逐渐减慢,到圆柱后缘点($x=0.5$)时压力系数已变为负值,即小于来流处的压力。在回流区 $0.5<x<x_0$ 内,压力系数继续下降,先减小到一个极小值,然后开始增大,在尾迹区之外继续增长,并在下游无穷远处恢复为来流处的压力。实际上,当 $x=30$ 时,压力系数已接近 $C_p=0$。从图~\ref{fig: average-center-1e-05} d) 可以看出,雷诺数越大,压力系数越低。在圆柱内部,随着雷诺数的增大,压力系数会降低到更小的值,在尾迹区的下降幅度也更大,从而取得了更小的极小值;另一方面,随着雷诺数的增大,达到极小值的位置也向上游移动。图~\ref{fig: average-center-1e-05} d) 中符号“$\times$”的含义与图~\ref{fig: average-center-1e-05} b) 相同,可以发现,随着雷诺数的增大,压力系数取极小值的位置也就越接近回流区的尾端点。

$Da=0.0001$ 和 $Da=0.001$ 时平均速度和压力系数沿 $y=0$ 的分布与图~\ref{fig: average-center-1e-05} 相似,方便起见,不再一一列出。与图~\ref{fig: average-center-1e-05} 相比,当达西数增大时,这些平均量具有相似的变化趋势,但也存在一些不同。主要区别体现在平均量在圆柱内部的变化情况。当 $Da=0.0001$ 时,圆柱前缘的速度将不再减小至零,从而速度在圆柱内部仍会缓慢下降。$Da=0.001$ 时仍然如此,而且变化情况更加明显。

\begin{figure}
	\setlength{\subfigcapskip}{-1bp}
	\centering
	\begin{minipage}{\textwidth}
		\centering
		\subfigure[流向平均速度]{\includegraphics[width=0.47\textwidth]{../analysis/average/{u_x_1e-05}.pdf}}
		\subfigure[流向平均速度在尾迹附近的放大图]{\includegraphics[width=0.47\textwidth]{../analysis/average/{u_x_wake_1e-05}.pdf}}
	\end{minipage}
	\centering
	\begin{minipage}{\textwidth}
		\centering
		\subfigure[平均压力系数]{\includegraphics[width=0.47\textwidth]{../analysis/average/{Cp_x_1e-05}.pdf}}
		\subfigure[平均压力系数在尾迹附近的放大图]{\includegraphics[width=0.47\textwidth]{../analysis/average/{Cp_x_wake_1e-05}.pdf}}
	\end{minipage}
	\vspace{0.2em}
	\caption{时间平均量沿尾迹中心线的分布($Da=1\times 10^{-5}$)}
	\label{fig: average-center-1e-05}
\end{figure}

为了解达西数的变化对平均量分布造成的影响,图~\ref{fig: average-center-Re100} 以 $Re=100$ 为例,画出了不同达西数下各平均量沿 $y=0$ 的变化情况。由图可知,不同达西数下速度分布具有相同的变化趋势,如上文所述,最明显的不同在于圆柱内部的速度分布。当 $Da=1\times 10^{-5}$ 时,圆柱内部速度为零,此时的速度分布应与固体圆柱绕流时的情形相同。$Da=0.0001$ 时圆柱内部的速度仍旧很小,流动十分微弱,说明此时圆柱内部的阻力仍然很大。当 $Da=0.001$ 时,流体到达圆柱前缘的速度已经增大到了 0.2,为来流速度的 20\%,接着流体的速度在圆柱内下降,到圆柱后缘时稍大于零。此时圆柱内部的阻力已经有所减少。如果达西数继续增大,圆柱内的速度仍会下降,但圆柱前后缘的速度都将变大,速度的变化曲线将不断向上移动。从图~\ref{fig: average-center-Re100} a) 可以看出,随着达西数的增大,圆柱下游的尾迹损失速度也越大。由图~\ref{fig: average-center-Re100} c,d) 可知,达西数越大时,压力系数在圆柱内下降得越少,随后在尾迹中会降到一个更低的值,但经过下游长距离的恢复,压力系数都到达了相同的终点。%之后画更大的达西数,这可能说明了圆柱后缘速度对尾迹的影响。

\begin{figure}
	\setlength{\subfigcapskip}{-1bp}
	\centering
	\begin{minipage}{\textwidth}
		\centering
		\subfigure[流向平均速度]{\includegraphics[width=0.47\textwidth]{../analysis/average/{u_x_Re100}.pdf}}
		\subfigure[流向平均速度在尾迹附近的放大图]{\includegraphics[width=0.47\textwidth]{../analysis/average/{u_x_wake_Re100}.pdf}}
	\end{minipage}
	\centering
	\begin{minipage}{\textwidth}
		\centering
		\subfigure[平均压力系数]{\includegraphics[width=0.47\textwidth]{../analysis/average/{Cp_x_Re100}.pdf}}
		\subfigure[平均压力系数在尾迹附近的放大图]{\includegraphics[width=0.47\textwidth]{../analysis/average/{Cp_x_wake_Re100}.pdf}}
	\end{minipage}
	\vspace{0.2em}
	\caption{时间平均量沿尾迹中心线的分布($Re=100$)}
	\label{fig: average-center-Re100}
\end{figure}

%沿圆柱表面的分布(平均压力系数,方均根压力系数)
图~\ref{fig: average-surface-1e-05} 展示了 $Da=1\times 10^{-5}$ 时圆柱表面上平均压力系数的分布。$\theta=0$ 对应于圆柱的前缘点,该点的压力系数 $C_{ps}$ 与图~\ref{fig: average-center-1e-05} 中 $x=-0.5$ 处是同一点,压力系数相同。沿着圆柱表面从上游到下游,随着 $\theta$ 的增大,压力系数逐渐降低,在 $\theta=\theta_0$ 处压力系数降为零,此时的压力等于来流处的压力,$\theta_0$ 的范围是 $37^\circ$--$44^\circ$。$\theta>\theta_0$ 时,压力系数继续降低,直到达到最小值,此时的角度记为 $\theta_m$,范围是 $82^\circ$--$89^\circ$。之后又随着 $\theta$ 的增大而增大,最后随雷诺数的不同而具有不同的变化,$\theta=180^\circ$ 对应圆柱的后缘点,该点的压力系数 $C_{pb}$ 与图~\ref{fig: average-center-1e-05} 中 $x=0.5$ 处是同一点,压力系数相同。比较不同雷诺数下的分布曲线,可以看到,$\theta=0$ 处各个曲线排列紧密,而 $\theta=180^\circ$ 则稀疏许多,即后缘点压力系数对雷诺数的变化更加敏感。对于圆柱表面的任意位置,雷诺数越大,压力系数越小。随着雷诺数的增大,$\theta_m$ 沿着圆柱表面向上游移动;同时,$\theta_0$ 也向上游移动,二者移动的幅度相同,即 $\Delta\theta=\theta_m-\theta_0$ 大约为常数,$\Delta\theta=44.87^\circ$--$45.62^\circ$。

$Da=0.0001$ 和 $Da=0.001$ 时圆柱表面上的平均压力系数分布与图~\ref{fig: average-surface-1e-05} 具有相同的趋势。其他不同之处由达西数的不同而引起。为了表明达西数的影响,图~\ref{fig: average-surface-Re100} 将雷诺数固定为 100,画出了不同达西数下平均压力系数沿圆柱表面的分布曲线。通过观察可知,随着达西数的增大,压力系数降为零的位置 $\theta_0$ 逐渐增大,压力系数的最小值点 $\theta_m$ 也随之增大。在 $\theta=120^\circ$ 之前,不同达西数下压力系数的差别较大,之后则差别较小。在圆柱前后两端,不同达西数所对应的压力系数 $C_{ps}$ 和 $C_{pb}$ 都差别不大,即前后缘点压力系数对达西数的敏感程度相近,这与图~\ref{fig: average-surface-1e-05} 中雷诺数的情形不同。%还可以画其他达西数。

\begin{figure}
	\setlength{\subfigcapskip}{-1bp}
	\centering
	\includegraphics[width=0.7\textwidth]{../analysis/average/{Cp_theta_1e-05}.pdf}
	% \begin{minipage}{\textwidth}
	% 	\centering
	% 	\subfigure[压力系数]{\includegraphics[width=0.47\textwidth]{../analysis/average/Cp_theta}}
	% 	\subfigure[平均涡量]{\includegraphics[width=0.47\textwidth]{../analysis/average/Vort_theta}}
	% \end{minipage}
	\vspace{0.2em}
	\caption{平均压力系数沿圆柱表面的分布($Da=1\times 10^{-5}$)}
	\label{fig: average-surface-1e-05}
\end{figure}

\begin{figure}
	\setlength{\subfigcapskip}{-1bp}
	\centering
	\includegraphics[width=0.7\textwidth]{../analysis/average/{Cp_theta_Re100}.pdf}
	% \begin{minipage}{\textwidth}
	% 	\centering
	% 	\subfigure[压力系数]{\includegraphics[width=0.47\textwidth]{../analysis/average/Cp_theta}}
	% 	\subfigure[平均涡量]{\includegraphics[width=0.47\textwidth]{../analysis/average/Vort_theta}}
	% \end{minipage}
	\vspace{0.2em}
	\caption{平均压力系数沿圆柱表面的分布($Re=100$)}
	\label{fig: average-surface-Re100}
\end{figure}

\section{本章小结}

本章在非稳态层流范围内选取了合适的例子进行计算,设置了适当的参数值,获得输出的数据文件。\ref{sec: transient} 节对数据进行初步处理,得到流场的瞬时分布,瞬态行为主要由涡量等直线和流线体现。\ref{sec: average} 节经过对流场平均化处理,得到平均涡量的等值线和平均速度流线,分析了各个平均量沿尾迹中心线和圆柱表面变化情况,得到达西数和雷诺数对平均流场的影响,还可以借此得到尾迹中心线和圆柱表面的若干重要位置。
