% !Mode:: "TeX:UTF-8"
% !TEX root = ../main.tex
\chapter{结果与讨论}

\section{流动实例和流场图}

\subsection{从稳态流动到非稳态流动}

固定达西数,当雷诺数较小时,流动处于稳态;随着雷诺数的增加,流动逐渐转变为非稳态。所以,对于每个达西数,都存在一个临界雷诺数,该雷诺数标志着流动状态的转变。临界雷诺数可以从升力系数随时间的变化图($C_l$-$t$ 图)中得到。例如,图~\ref{fig: Cl_t} 显示了不同雷诺数下升力系数随时间的变化。当 $Da=0.0001$ 时,当 $Re=40$ 时,升力系数随着时间逐渐减小,并最终减小到一个极小的值,因此 $Re=40$ 时流动尚处于稳态。当 $Re=45$ 时,升力系数不会随时间减小,而是几乎稳定地波动,因此 $Re=45$ 时流动已经处于非稳态。由此可知,$Da=0.0001$ 所对应的临界雷诺数处于 40 和 45 之间。其他达西数所对应的临界雷诺数可以由同样的方法得出,最终得到的临界雷诺数列于表~\ref{tab: critical Re} 中。另外,临界雷诺数还可以从图~\ref{fig: Cl_t}、图~\ref{fig: resd} 和图~\ref{fig: error} 上得到。图~\ref{fig: resd} 展示了不同雷诺数和达西数下的误差。

表~\ref{tab: DaRe} 列出了达西数和雷诺数可能的组合值。

\begin{table}
	\caption{计算时用到的达西数和雷诺数的值的设置}\label{tab: DaRe}
	\vspace{.5em}\centering\wuhao
	\begin{tabular}{*{6}{c}}
		\toprule[1.5pt]
		\multirow{2}[3]{*}{$Re$} & \multicolumn{5}{c}{$Da$} \\
		\cmidrule[.67pt](lr){2-6}
		& 0.00001 & 0.0001 & 0.0005 & 0.001 & 0.005 \\
		\midrule[1pt]
		40 & ${\color{red}\surd}$ & $\surd$ &   & $\surd$ & $\surd$ \\
		41 & $\surd$ & $\surd$ &   & $\surd$ &   \\
		42 & ${\color{red}\surd}$ &   & $\surd$ & $\surd$ &   \\
		43 & ${\color{red}\surd}$ &   & $\surd$ & $\surd$ &   \\
		44 & ${\color{red}\surd}$ &   & $\surd$ & $\surd$ &   \\
		45 & ${\color{red}\surd}$ &   & $\surd$ & $\surd$ &   \\
		46 & $\surd$ &         &   &         &   \\
		47 & $\surd$ &         &   &         &   \\
		48 & $\surd$ &         &   &         &   \\
		49 & $\surd$ &         &   &         &   \\
		50 & ${\color{red}\surd}$ & $\surd$ &   & $\surd$ & $\surd$ \\
		60 & ${\color{red}\surd}$ & ${\color{red}\surd}$ & ${\color{red}\surd}$ & $\surd$ & $\surd$ \\
		70 & ${\color{red}\surd}$ & ${\color{red}\surd}$ & ${\color{red}\surd}$ & $\surd$ & $\surd$ \\
		80 & ${\color{red}\surd}$ & ${\color{red}\surd}$ & ${\color{red}\surd}$ & $\surd$ & $\surd$ \\
		90 & ${\color{red}\surd}$ & ${\color{red}\surd}$ & ${\color{red}\surd}$ & $\surd$ & $\surd$ \\
		100 & ${\color{red}\surd}$ & ${\color{red}\surd}$ & ${\color{red}\surd}$ & $\surd$ & $\surd$ \\
		120 & ${\color{red}\surd}$ & ${\color{red}\surd}$ & ${\color{red}\surd}$ & $\surd$ & $\surd$ \\
		140 & ${\color{red}\surd}$ & ${\color{red}\surd}$ & ${\color{red}\surd}$ & $\surd$ & $\surd$ \\
		160 & ${\color{red}\surd}$ & ${\color{red}\surd}$ & ${\color{red}\surd}$ & $\surd$ & $\surd$ \\
		180 & ${\color{red}\surd}$ & ${\color{red}\surd}$ & ${\color{red}\surd}$ & $\surd$ & $\surd$ \\
		200 & ${\color{red}\surd}$ & ${\color{red}\surd}$ & ${\color{red}\surd}$ & $\surd$ & $\surd$ \\
		\bottomrule[1.5pt]
	\end{tabular}
\end{table}

\begin{figure}
	\setlength{\subfigcapskip}{-1bp}
	\centering
	\begin{minipage}{\textwidth}
		\centering
		\subfigure[$Re=40$]{\includegraphics[width=0.8\textwidth]{../figs/0.0001_40/Cl_t}}
	\end{minipage}
	\centering
	\begin{minipage}{\textwidth}
		\centering
		\subfigure[$Re=45$]{\includegraphics[width=0.8\textwidth]{../figs/0.0001_45/Cl_t}}
	\end{minipage}
	\vspace{0.2em}
	\caption{不同雷诺数下升力系数随时间的变化($Da=0.0001$)}
	\label{fig: Cl_t}
\end{figure}

\begin{table}
	\caption{临界雷诺数}\label{tab: critical Re}
	\vspace{.5em}\centering\wuhao
	\begin{tabular}{cc}
		\toprule[1.5pt]
		$Da$ & 临界 $Re$ \\
		\midrule[1pt]
		0.00001 & 40--45 \\
		0.0001  & 40--45 \\
		0.001   & 40--45 \\
		0.005   & ? \\
		0.01    & 140--160 \\
		0.1     & > 200 \\
	\bottomrule[1.5pt]
	\end{tabular}
\end{table}

\begin{figure}
	\centering
	\begin{minipage}{\textwidth}
		\centering
		\subfigure[$Re=40$]{\includegraphics[width=0.8\textwidth]{../figs/0.0001_40/resd}}
	\end{minipage}
	\centering
	\begin{minipage}{\textwidth}
		\centering
		\subfigure[$Re=45$]{\includegraphics[width=0.8\textwidth]{../figs/0.0001_45/resd}}
	\end{minipage}
	\caption{误差 $Da=0.0001$}
	\label{fig: resd}
\end{figure}

\begin{figure}
	\centering
	\includegraphics[width=0.8\textwidth]{../analysis/meanError_Re}
	\caption{平均误差随 $Re$ 的变化}
	\label{fig: error}
\end{figure}

\subsection{不同参数下的流场图}

根据测量数据,可以得到不同达西数和雷诺数下涡量的等值线图。图~\ref{fig: vorticity contour} 展示了 $Da=0.0001$ 时不同雷诺数下流动已达稳定 \footnote{此处“稳定”一词对应于“stable”,指流动特性不随时间变化,或随时间做周期性变化,因此不同于“稳态”,指特性不随时间变化。} 时涡量的等值线图。从图中可以看出,$Re=40$ 时流动处于稳态,随着雷诺数的增大,流动逐渐变为非稳态,并在圆柱的后端产生了一对旋涡,涡脱落之后向下游移动,形成一列整齐的涡街。

%还可以得到非稳态情形下一个周期内涡的脱落情形。

\begin{figure}
	\centering
	\begin{minipage}{\textwidth}
		\centering
		\subfigure[$Re=40$]{\includegraphics[width=0.4\textwidth]{../figs/0.0001_40/flow}}
		\subfigure[$Re=45$]{\includegraphics[width=0.4\textwidth]{../figs/0.0001_45/flow}}
	\end{minipage}
	\centering
	\begin{minipage}{\textwidth}
		\centering
		\subfigure[$Re=50$]{\includegraphics[width=0.4\textwidth]{../figs/0.0001_50/flow}}
		\subfigure[$Re=90$]{\includegraphics[width=0.4\textwidth]{../figs/0.0001_90/flow}}
	\end{minipage}
	\centering
	\begin{minipage}{\textwidth}
		\centering
		\subfigure[$Re=140$]{\includegraphics[width=0.4\textwidth]{../figs/0.0001_140/flow}}
		\subfigure[$Re=200$]{\includegraphics[width=0.4\textwidth]{../figs/0.0001_200/flow}}
	\end{minipage}
	\caption{$Da=0.0001$ 时不同雷诺数下涡量的等值线图}
	\label{fig: vorticity contour}
\end{figure}

\begin{figure}
	\centering
	\begin{minipage}{\textwidth}
		\centering
		\subfigure[$Re=40$]{\includegraphics[width=0.4\textwidth]{../figs/0.001_40/flow}}
		\subfigure[$Re=45$]{\includegraphics[width=0.4\textwidth]{../figs/0.001_45/flow}}
	\end{minipage}
	\centering
	\begin{minipage}{\textwidth}
		\centering
		\subfigure[$Re=50$]{\includegraphics[width=0.4\textwidth]{../figs/0.001_50/flow}}
		\subfigure[$Re=90$]{\includegraphics[width=0.4\textwidth]{../figs/0.001_90/flow}}
	\end{minipage}
	\centering
	\begin{minipage}{\textwidth}
		\centering
		\subfigure[$Re=140$]{\includegraphics[width=0.4\textwidth]{../figs/0.001_140/flow}}
		\subfigure[$Re=200$]{\includegraphics[width=0.4\textwidth]{../figs/0.001_200/flow}}
	\end{minipage}
	\caption{$Da=0.001$ 时不同雷诺数下涡量的等值线图}
	\label{fig: vorticity contour}
\end{figure}

\begin{figure}
	\centering
	\begin{minipage}{\textwidth}
		\centering
		\subfigure[$Re=40$]{\includegraphics[width=0.4\textwidth]{../figs/0.0001_40/flow}}
		\subfigure[$Re=45$]{\includegraphics[width=0.4\textwidth]{../figs/0.0001_45/flow}}
	\end{minipage}
	\centering
	\begin{minipage}{\textwidth}
		\centering
		\subfigure[$Re=50$]{\includegraphics[width=0.4\textwidth]{../figs/0.0001_50/flow}}
		\subfigure[$Re=90$]{\includegraphics[width=0.4\textwidth]{../figs/0.0001_90/flow}}
	\end{minipage}
	\centering
	\begin{minipage}{\textwidth}
		\centering
		\subfigure[$Re=140$]{\includegraphics[width=0.4\textwidth]{../figs/0.0001_140/flow}}
		\subfigure[$Re=200$]{\includegraphics[width=0.4\textwidth]{../figs/0.0001_200/flow}}
	\end{minipage}
	\caption{$Da=0.0001$ 时不同雷诺数下涡量的等值线图}
	\label{fig: vorticity contour}
\end{figure}

\begin{figure}
	\centering
	\begin{minipage}{\textwidth}
		\centering
		\subfigure[$Re=40$]{\includegraphics[width=0.4\textwidth]{../figs/0.0001_40/flow}}
		\subfigure[$Re=45$]{\includegraphics[width=0.4\textwidth]{../figs/0.0001_45/flow}}
	\end{minipage}
	\centering
	\begin{minipage}{\textwidth}
		\centering
		\subfigure[$Re=50$]{\includegraphics[width=0.4\textwidth]{../figs/0.0001_50/flow}}
		\subfigure[$Re=90$]{\includegraphics[width=0.4\textwidth]{../figs/0.0001_90/flow}}
	\end{minipage}
	\centering
	\begin{minipage}{\textwidth}
		\centering
		\subfigure[$Re=140$]{\includegraphics[width=0.4\textwidth]{../figs/0.0001_140/flow}}
		\subfigure[$Re=200$]{\includegraphics[width=0.4\textwidth]{../figs/0.0001_200/flow}}
	\end{minipage}
	\caption{$Da=0.0001$ 时不同雷诺数下涡量的等值线图}
	\label{fig: vorticity contour}
\end{figure}

\section{Strouhal 数和阻力、升力系数}

\subsection{Strouhal 数}

对于非稳态流动,流动波动的频率是一个重要参数。而 Strouhal 数是表述流体波动的无量纲频率。频率与周期互为倒数,周期可以根据测量数据(例如 Point 3 的水平速度)从图中得出。继而根据方程 \eqref{eq: St} 可以得到 Strouhal 数。式 \eqref{eq: St} 描述了这些物理量之间的关系:
\begin{equation}\label{eq: St}
	St = \frac{Df}{U} = f = \frac{1}{T}
\end{equation}
\begin{tabularx}{\textwidth}{@{}l@{\quad}r@{——}X@{}}
	式中 & $St$ & Strouhal 数;\\
		& $D$ & 圆柱的直径,$=1$ m;\\
		& $f$ & 波动的频率;\\
		& $U$ & 远方的来流速度,$=1$ m/s;\\
		& $T$ & 波动的周期。 
\end{tabularx}\vspace{3.15bp}
得到的周期和 Strouhal 数记录在 \texttt{record.xlsx} 中。

Strouhal 数随雷诺数的变化如图~\ref{fig: St} 所示。从图中可以看出,$Da$ 固定时,$St$ 随 $Re$ 的增大而增大。固定 $Re$ 时,$Da$ 越小,则 $St$ 越接近固体圆柱绕流的情形。当 $Da=1 \times 10^{-5}$ 时,$St$ 曲线已经和固体圆柱绕流的变化曲线 \cite{} 相吻合。

\begin{figure}
	\centering
	\includegraphics[width=0.8\textwidth]{../analysis/St_Re2}
	\caption{Strouhal 数随 $Re$ 的变化}
	\label{fig: St}
\end{figure}

对于非稳态流动,一个周期内的涡量图如~\ref{fig: 4*vortex} 所示。

\begin{figure}
	\centering
	\begin{minipage}{\textwidth}
		\centering
		\subfigure[0]{\includegraphics[width=0.4\textwidth]{../figs/0.0001_100/T0}}
		\subfigure[1/4]{\includegraphics[width=0.4\textwidth]{../figs/0.0001_100/T1}}
	\end{minipage}
	\centering
	\begin{minipage}{\textwidth}
		\centering
		\subfigure[2/4]{\includegraphics[width=0.4\textwidth]{../figs/0.0001_100/T2}}
		\subfigure[3/4]{\includegraphics[width=0.4\textwidth]{../figs/0.0001_100/T3}}
	\end{minipage}
	\caption{在 0,1/4,2/4,和 3/4 周期时涡的脱落图 $Da=0.0001$,$Re=100$}
	\label{fig: 4*vortex}
\end{figure}

\subsection{阻力系数}

非稳态情形下阻力和升力都随时间做周期性变化,所以此处取一个周期内的平均阻力和升力。平均阻力系数随雷诺数的变化如图~\ref{fig: meanCd} 所示。对于 $Da=0.0001$、$Da=0.001$,平均阻力系数随着雷诺数的增加而减小,然后随雷诺数的增加而增加。

\begin{figure}
	\centering
	\includegraphics[width=0.8\textwidth]{../analysis/meanCd_Re2}
	\caption{平均阻力系数随 $Re$ 的变化}
	\label{fig: meanCd}
\end{figure}

\subsection{升力系数}

升力系数波动的振幅随雷诺数的变化如图~\ref{fig: ClAmplitude} 所示,与文献结果一致 \cite{Park1998}。

\begin{figure}
	\centering
	\includegraphics[width=0.8\textwidth]{../analysis/ClAmplitude_Re}
	\caption{升力系数波动的振幅随 $Re$ 的变化}
	\label{fig: ClAmplitude}
\end{figure}

\section{能谱分析} % FFT

速度随时间的变化以及相应的 Fourier 变换如图~\ref{fig: velocity Fourier} 所示。升力系数随时间的变化以及相应的 Fourier 变换如图~\ref{fig: Cl Fourier} 所示。图中画出了稳定时的 10 个周期。

\begin{figure}
	\centering
	\begin{minipage}{\textwidth}
		\centering
		\includegraphics[width=0.8\textwidth]{../figs/0.0001_50/psd_U}
	\end{minipage}
	\centering
	\begin{minipage}{\textwidth}
		\centering
		\includegraphics[width=0.8\textwidth]{../figs/0.0001_50/psd_V}
	\end{minipage}
	\caption{Fourier 变换 $Da=0.0001$, $Re=50$.}
	\label{fig: velocity Fourier}
\end{figure}

\begin{figure}
	\centering
	\includegraphics[width=0.8\textwidth]{../figs/0.0001_50/psd_Cl}
	\caption{升力系数和 Fourier 变换 $Da=0.0001$, $Re=50$.}
	\label{fig: Cl Fourier}
\end{figure}

\section{多孔介质对流动状态的影响——与固体圆柱绕流相比}

\section{流动状态从稳态向非稳态的转变——与稳态圆柱绕流相比}

\section{本章小结}
